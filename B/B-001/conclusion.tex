\section{Conclusion}

In general, any of the RP-25 wheel profiles can be found by:
\begin{itemize}
	\item Looking up published values, in the RP-25 document, for a desired wheel code.
	\item Choosing a desired tread slope $\theta_s$.
	\item Calculating distance and angle values with equations \ref{eqn:x_s} through \ref{eqn:y_3}.
	\item Calculating point locations with equations \ref{eqn:p_g} through \ref{eqn:d_prime}.
\end{itemize}

Table \ref{table:mytable} shows calculated values: $p_g$, $\theta_g$, $p_s$, $p_1$, $p_2$, $p_3$, $p_d$ and $d'$; found from equations \ref{eqn:x_s} through \ref{eqn:d_prime}. 
The values are tabulated by wheel code and indexed by tread slopes, $\theta_s$, from $0\degree$ to $3\degree$.

Published values from RP-25 and tabulated values from Table \ref{table:mytable} can be used together to draw the profile shown in Figure \ref{fig:fig_f10}.
To draw a wheel profile:
\begin{itemize}
	\item First draw $x$ and $y$ axes at the origin.
	\item Then mark the tabulated locations for points: $p_g$, $p_s$, $p_1$, $p_2$, $p_3$ and $p_d$.
	\item Next draw arcs of radius $R1$, $R2$ and $R3$ centered at points $p_1$, $p_2$ and $p_3$.
	\item From point $p_s$ draw the tread slope at an angle $\theta_s$.
	\item Last, complete the drawing with vertical lines a distance $N'$ apart at the left and right sides of the profile.
\end{itemize}

Drawings for different wheel codes will have different relative spacing between points.
This is expected and is due to the discrete values published in the RP-25 document.

Upon inspecting Table \ref{table:mytable} you will notice the angle $\theta_g$ varies greatly for different wheel profile codes.
For example $\theta_g$ for $Code\ 110$ at $\theta_s = 3\degree$ is $16.5196\degree$, whereas it is $-0.0785\degree$ at $\theta_s = 3\degree$ for $Code\ 93$.

In Table \ref{table:mytable} you will notice that changing the slope angle $\theta_s$ within any particular wheel code has an effect upon location of $p_s$.
On the other hand, the effect on $\theta_g$ is slight.
Furthermore, changing $\theta_s$ has negligible effect upon other point locations.
Table \ref{table:mytable} is deliberately redundant to show that except $p_s$, when rounded to four decimal places, point locations are effectively independent of tread slope.

When using a CAD program to draw profiles from the tabulated data, you may notice some slight misalignments when zooming in closely.
This is due to the tabulated data being rounded to 4 decimal places.
If you need greater precision you can always perform calculations with equations \ref{eqn:x_s} through \ref{eqn:d_prime} to whatever precision you require.

Also, if you need point locations for an in-between value of slope angle, say for $\theta_s = 1.5\degree$, you should perform calculations equations \ref{eqn:x_s} through \ref{eqn:d_prime}.

As a final thought, the location of $p_d$ is at flange depth $d'$, which is well short of the flange depth $D'$ given in the RP-25 document.
The RP-25 document misleadingly draws the intersection of arcs $R2$ and $R3$ at flange depth $D'$.
You should realize $D'$ is the maximum flange depth specified by NMRA Standard S-4.2, therefore locating $p_d$ at depth $d'$ meets the standard.
