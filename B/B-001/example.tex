\section{Example}
We now know how to algebraically locate all the features of the RP-25 wheel profile.

As a test we should apply what we know to a concrete example.
Let's calculate the critical: points, angles and distances; for a code 110 wheel profile with a $3^{\circ}$ tread slope.

\paragraph{Given:}
\begin{itemize}
	\item A wheel profile where $code = 110$
	\item Tread with $slope = \theta_s = 3^{\circ}$
\end{itemize}
	

\paragraph{Find:}
\begin{itemize}
	\item Published values for: $N'$, $T$, $W$, $D'$, $P$, $R1$, $R2$ and $R3$
	\item Calculated values for: $x_s$, $y_s$, $x_g$, $y_g$, $\theta_g$, $x_2$, $y_2$, $x_d$, $y_d$, $x_3$ and $y_3$
	\item Calculated values for: $p_g$, $p_1$, $p_s$, $p_2$, $p_d$, $p_3$ and $d'$	
\end{itemize}


\paragraph{Steps:}
\begin{enumerate}
\item From the RP-25 document lookup the row with published values for code = 110 :
\begin{align*}
N' &= 0.110 \\
T &= 0.030 \\
W &= 0.080 \\
D' &= 0.025 \\
P &= 0.010 \\		
R1 &= 0.014 \\		
R2 &= 0.018 \\		
R3 &= 0.018				
\end{align*}		

\item Substitute values for $R1$ and $slope = \theta_g$ into equation \ref{eqn:x_s} to find $x_s$.
\begin{align*}
	x_s &= R1 \cdot \sin(\theta_s) \\
	x_s &= 0.014 \cdot \sin(3^{\circ}) \\
	x_s &= 0.014 \cdot 0.0523359562 \\
	x_s &= 0.0007327034
\end{align*}		

\item Calculate $y_s$ by substituting values for $R1$ and $slope = \theta_g$ into equation \ref{eqn:y_s}.
\begin{align*}
	y_s &= R1 \cdot \cos(\theta_s) \\
	y_s &= 0.014 \cdot \cos(3^{\circ}) \\
	y_s &= 0.014 \cdot 0.9986295348	\\
	y_s &= 0.0139808135
\end{align*}		

\item Now plug in the calculated value of $y_s$ and the published value for $P$ into equation \ref{eqn:y_g} to find $y_g$.
\begin{align*}
	y_g &= y_s - P \\
	y_g &= 0.0139808135 - 0.01 \\
	y_g &= 0.0039808135
\end{align*}		

\item Equation \ref{eqn:x_g} with values substituted in for $R1$ and $y_g$ yields $x_g$.
\begin{align*}
	x_g &= \sqrt{(R1)^2 - y_g^2} \\
	x_g &= \sqrt{(0.014)^2 - (0.0039808135)^2} \\
	x_g &= \sqrt{0.000196 - 0.000015846876015} \\
	x_g &= \sqrt{0.000180153123985} \\
	x_g &= 0.0134221132
\end{align*}		

\item To calculate angle $\theta_g$ simply apply values for $x_g$ and $R1$ to equation \ref{eqn:theta_g}.
\begin{align*}  
	\theta_g &= \arccos \left( \frac{x_g}{R1} \right) \\
	\theta_g &= \arccos \left( \frac{0.0134221132}{0.014} \right) \\
	\theta_g &= \arccos (0.9587223747) \\
	\theta_g &= 16.5196295818^\circ
\end{align*}

\item Distance $x_2$ can be found with equation \ref{eqn:x_2} and the values for $R2$ and $\theta_g$.
\begin{align*}
	x_2 &= R2 \cdot \cos(\theta_g) \\
	x_2 &= 0.018 \cdot \cos(16.5196295818^\circ) \\
	x_2 &= 0.018 \cdot 0.9587223747	\\
	x_2 &= 0.0172570027
\end{align*}

\item Likewise $y_2$ is calculated with $R2$, $\theta_g$ and equation \ref{eqn:y_2}.
\begin{align*}
	y_2 &= R2 \cdot \sin(\theta_g) \\
	y_2 &= 0.018 \cdot \sin(16.5196295818^\circ) \\
	y_2 &= 0.018 \cdot 0.2843438205 \\
	y_2 &= 0.0051181888
\end{align*}

\item To calculate $x_d$, substitute the value $x_2$ and the published $T$ value into equation \ref{eqn:x_d}.
\begin{align*}
	x_d &= x_2 - T/2 \\
	x_d &= 0.0172570027 - (0.03)/2 \\
	x_d &= 0.0172570027 - 0.015	\\
	x_d &= 0.0022570027
\end{align*}


\item Distance $y_d$ is calculated with equation \ref{eqn:y_d} and the values for $R2$ and $x_d$.
\begin{align*}
	y_d &= \sqrt{(R2)^2 - x_d^2} \\
	y_d &= \sqrt{(0.018)^2 - (0.0022570027)^2} \\
	y_d &= \sqrt{0.000324 - 0.000005094061390} \\
	y_d &= \sqrt{0.0003189059} \\
	y_d &= 0.0178579377
\end{align*}


\item From equation \ref{eqn:x_3}, find $x_3$ by plugging in the value for $x_d$.
\begin{align*}
	x_3 &= 2 \cdot x_d \\
	x_3 &= 2 \cdot 0.0022570027 \\
	x_3 &= 0.0045140055
\end{align*}

\item Equation \ref{eqn:y_3} simply states $y_3$ will always be zero.
\begin{align*}
	y_3 &= 0
\end{align*}


%%%%%%%

\item Equation \ref{eqn:p_g} states gage point $p_g$ will always be at the origin by definition.
\begin{equation*}
	p_g = (0, 0)
\end{equation*}


\item To locate point $p_1$, plug in $x_g$ and $y_g$ into equation \ref{eqn:p_1}.
\begin{align*}
	p_1 &= (-x_g, -y_g) \\
	p_1 &= (-0.0134221132, -0.0039808135) \\
	p_1 &= (-0.0134, -0.0040)
\end{align*}

\item Point $p_s$ is found by applying values for: $x_s$, $x_g$, $y_s$ and $y_g$; to equation \ref{eqn:p_s}.
\begin{align*}
	p_s =\ &(x_s - x_g, y_s - y_g) \\
	p_s =\ &(0.0007327034 - 0.0134221132, \\ &\ 0.0139808135 - 0.0039808135) \\
	p_s =\ &(-0.0126894098, 0.01) \\
	p_s =\ &(-0.0127, 0.0100)
\end{align*}

\item Use equation \ref{eqn:p_2} and the values for $x_2$ and $y_2$ to find point $p_2$.
\begin{align*}
	p_2 &= (x_2, y_2) \\
	p_2 &= (0.0172570027, 0.0051181888) \\
	p_2 &= (0.0173, 0.0051)
\end{align*}

\item Now use equation \ref{eqn:p_d} and values: $x_2$, $x_d$, $y_2$ and $y_d$; to yield point $p_d$.
\begin{align*}
	p_d =\ &(x_2 - x_d, y_2 - y_d) \\
	p_d =\ &(0.0172570027 - 0.0022570027, \\ &\ 0.0051181888 - 0.0178579377) \\
	p_d =\ &(0.015, -0.0127397489) \\
	p_d =\ &(0.015, -0.0127)
\end{align*}

\item The location of point $p_3$ can by discovered by substituting values for: $x_2$, $x_3$ and $y_2$; into equation \ref{eqn:p_3}.
\begin{align*}
	p_3 =\ &(x_2 - x_3, y_2) \\
	p_3 =\ &(0.0172570027 - 0.0045140055, \\ &\ 0.0051181888) \\
	p_3 =\ &(0.0127429972, 0.0051181888) \\
	p_3 =\ &(0.0127, 0.0051)
\end{align*}

\item The distance $d'$ can easily be calculated with equation \ref{eqn:d_prime} and values for: $P$, $y_2$ and $y_d$.
\begin{align*}
	d' &= P - y_2 + y_d \\
	d' &= 0.01 - 0.0051181888 + 0.0178579377 \\
	d' &= 0.022739748 \\
	d' &= 0.0227
\end{align*}

\item Finally, all the published and calculated values can be used to draw the profile shown in Figure \ref{fig:fig_f10}.
\begin{itemize}
	\item First draw $x$ and $y$ axes.
	\item Then mark the calculated locations for the points.
	\item Next draw arcs of radius $R1$, $R2$ and $R3$ centered at points $p_1$, $p_2$ and $p_3$.
	\item From point $p_s$ draw the tread slope at an angle $\theta_s$.
	\item Last, complete the drawing with vertical lines a distance $N'$ apart at the left and right sides of the profile.
\end{itemize}

\end{enumerate}
