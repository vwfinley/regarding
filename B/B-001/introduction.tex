\section{Introduction}
While concise, the RP-25 document can be perplexing.
Discussion of the wheel-contour construction process is absent.
Locations of circle centers are omitted.
Calculations of circle radius values are conspicuously missing.
Important information is implied or excluded.

Understanding 1960s technology aids in understanding the RP-25 document.
Most of the omitted information may have been obvious to technical people, of the early 1960s, when RP-25 was written.
Technology was analog for most of the population.
Systems were designed and drawn with simple drafting tools.
Secondary school graduates would have been well-trained in constructive geometry.
Mechanical design was mainly accomplished by applying Euclidean techniques, rather than algebraic or trigonometric calculation.
Calculations were performed by hand, when required, or perhaps with a sliderule.

Prior to inexpensive digital personal computers and computer aided drawing (CAD), most design work involved analog drafting tools.
Digital calculators may have substituted later for sliderules, however, designs and drawings were still analog.
Although the current RP-25 document is typeset as a *.pdf document, the design is still vintage 1960s analog.


