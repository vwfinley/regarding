% VWF NOTE: Trim down commentary
% Mention of ogives and references to prior VWF works
% Reference NMRA standard  S-4.2 2019.01.04. Contour is non-critial.  Other contours could work so long as the standard is observed.  Non-static Operational effects: In fact as a rail wheels wear the countour changes dimensions.  Also buildup of dirt and crud changes contour dimensions.
% Source of confusion: ideal has 3 constraints (requirements) but, implementation is under-contrained only 2 of 3 
% Relaxing requirements vs. relaxing recommendations


\section{Introduction}
The RP-25 document\endnote{Recommended Practice RP-25, July 2009, Copyright 1986-2009, National Model Railroad Association, Inc.,  Designed by: Olesen, Mortimer and Bradley, Updated by D.A. Voss, https://www.nmra.org} is perplexing.
\begin{itemize}
\item It ignores discussion of how to accurately draw the wheel contour.
\item It excludes how the recommended circle/arc radii values were determined.
\item It omits critical information, like the position of circle/arc centers.
\item It overlooks manufacturing tolerances.
\end{itemize}

Furthermore, when reviewing dimensions defined in neighboring wheel codes, it becomes obvious that the values do not scale proportionally from one code to the next.
\begin{itemize}
\item Perhaps the dimensions were not developed from a mathematical formula?
\item Perhaps the dimensions were based upon common practice at the time a particular rail/wheel/gauge combination was developed?
\item Perhaps the dimensions were a compromise amongst competing equipment manufacturers?
\end{itemize}

You should understand, specific contour of a wheel is relatively unimportant.
Many wheel contours have been tried with varying success.\endnote{cite Model Railroader article}
In fact you are free to invent your own contour if you wish.
What is actually important is that a wheelset (pair of wheels mounted on an axle):
\begin{itemize}
\item Conforms to NMRA\textsuperscript{\textregistered} standard S-4.2 for wheelsets.\endnote{Standard S-4.2, January 2019, National Model Railroad Association, Inc., https://www.nmra.org}
\item Passes inspection with a NMRA\textsuperscript{\textregistered} RP-2 test gauge.\endnote{Recommended Practice RP-2, Standards Gauge, January 15, 2024, National Model Railroad Association, Inc., https://www.nmra.org}
\end{itemize}

Keep in mind, RP-25 is only a recommended practice, rather than a standard.
However, it still would be helpful to understand what that recommended practice is, and how to apply it.

If you can draw a RP-25 contour accurately, then you can understand the ideal you are attempting to achieve.

The following describes an approach to drawing the RP-25 contour.
