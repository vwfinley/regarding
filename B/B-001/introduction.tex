% VWF NOTE: Trim down commentary
% Mention of ogives and references to prior VWF works
% Reference NMRA standard  S-4.2 2019.01.04. Contour is non-critial.  Other contours could work so long as the standard is observed.  Non-static Operational effects: In fact as a rail wheels wear the countour changes dimensions.  Also buildup of dirt and crud changes contour dimensions.
% Source of confusion: ideal has 3 constraints (requirements) but, implementation is under-contrained only 2 of 3 
% Relaxing requirements vs. relaxing recommendations


\section{Introduction}
The RP-25 document\endnote{Recommended Practice RP-25, July 2009, Copyright 1986-2009, National Model Railroad Association, Inc.,  Designed by: Olesen, Mortimer and Bradley, Updated by D.A. Voss, https://www.nmra.org} is perplexing.
It could be improved by:
\begin{itemize}
\item Discussing of how to accurately draw the wheel contour.
\item Providing critical information, like the location of circle/arc centers.
\item Describing how recommended circle/arc radii values were determined.
\item Suggesting reasonable manufacturing tolerances.
\end{itemize}

When reviewing dimensions defined by adjacent RP-25 wheel codes, it becomes obvious that the published values scale disproportionally from one code to the next.
Answers to the following questions would help explain the published values.
\begin{itemize}
\item Were dimensions developed from a mathematical formula?
\item Were dimensions were based upon common practices at the time a particular rail/wheel/gauge combination was developed?
\item Were dimensions a compromise amongst competing equipment manufacturers?
\end{itemize}

Rigorous mathematical treatment of the RP-25 wheel contour in an official NMRA\textsuperscript{\textregistered} publication would be helpful.
Unfortunately, historical NMRA\textsuperscript{\textregistered}: bulletins, magazines, standards and recommended practices; omit mathematical basis for the contour. 
Alternative sources, like the article in Model Railroader magazine\endnote{Article: NMRA Recommended Practice Wheel contour - RP25, Model Railroader Magazine, January 1962, page 69, Kalmbach Publishing}, only discusses history of wheel profile development and restates technical information found in the RP-25 document; but does not discuss derivation of profile geometry.

Please understand, specific contour profile of a wheel is relatively unimportant.
Over the years many wheel contours have been proposed and tried with varying success.
In fact you are free to invent your own contours.
What is truly important is that a wheelset (pair of wheels mounted on an axle):
\begin{itemize}
\item Conforms to NMRA\textsuperscript{\textregistered} standard S-4.2 for wheelsets.\endnote{Standard S-4.2, January 2019, National Model Railroad Association, Inc., https://www.nmra.org}
\item Passes inspection with a NMRA\textsuperscript{\textregistered} RP-2 test gauge.\endnote{Recommended Practice RP-2, Standards Gauge, January 15, 2024, National Model Railroad Association, Inc., https://www.nmra.org}
\end{itemize}

Remember, RP-25 is only a recommended practice rather than a standard.
Nonetheless, it still would be helpful to understand what that recommended practice is, and how to apply it.

The good news is: if you can draw a RP-25 contour accurately, then you can understand the ideal you are attempting to achieve.
The following sections describe an approach to:
\begin{itemize}
    \item Draw the RP-25 contour
    \item Analyze the RP-25 contour mathematically
    \item Calculate and tabulate critical points
\end{itemize}

