\section{Constructive Solution}

%\begin{figure*}[ht!]
%	\centering
%	\begin{subfigure}{1.0\textwidth}
%		\centering
%		
\begin{tikzpicture}[x=1in, y=1in]
	\begin{scope}[xscale=.10, yscale=.10]	

%		\coordinate (origin) at (0,0);
%		\coordinate (P1) at (-0.0134, -0.0040);
%		\coordinate (P2) at (.0214, .0064);
%		\coordinate (P3) at (0.0086, 0.0064);
%		\coordinate (Pf) at (0.015, -0.015);
%		\coordinate (Pb) at (0.030, 0);
%		\coordinate (Pm) at (0.0075, -0.0075);		

		\clip (-21, -10) rectangle (10, 17);

		% axes
		\draw[lightgray, -{Latex[scale=1.5]}] (0, -30.0) -- (0, 15.0)  node [above] {$y$};	
		\draw[lightgray, -{Latex[scale=1.5]}] (-40.0, 0) -- (8.0, 0)  node [right] {$x$};		

		% vertical
		\draw[lightgray] (-16.0097, -35) -- (-16.0097, 35); % left

		% horizontal
		\draw[lightgray] (-50, 12.5) -- (90, 12.5) node[shift={(179.3 : 8.2)}]{$L_H$}; % L_H
		\draw[lightgray] (-19.0097, -4.0024) -- (-9.0097, -4.0024);	 % R center mark

		% sloped
		\draw[lightgray] (194.04 : 58) -- (14.04 : 98) node[shift={(-166.6 : 9.0)}]{$L_S$}; % L_S

		% bisector
		\draw[lightgray, dashed] (142.02 : -4) -- (142.02 : 24); % dashed line
		\draw[lightgray] ([shift=(135 :  20.3116)] 0, 0)  arc (135 : 150 : 20.3116); % mark

		% angle labels
		\draw[black] ([shift=(180 : 6)] 0, 0)  arc (180 : 194.04 : 6) node[shift={(-0.11, 0.06)}]{$\alpha$} ; % alpha
		\draw[black] ([shift=(194.04 : 2)] 0, 0)  arc (194.04 : 270 : 2) node[shift={(-0.17, -0.05)}]{$\beta$}; % beta
		\draw[black] ([shift=(142.02 : 4)] 0, 0)  arc (142.02 : 194.04 : 4) node[shift={(-0.06, 0.24)}]{$\gamma$}; % gamma

\draw[black] ([shift=(194.04 : 2.5)] 0, 0)  arc (194.04 : 90 : 2.5) node[shift={(-0.18, 0.02)}]{$\delta$}; % gamma

		\draw[black] ([shift=(0 : 6)] 194.04 : 16.5024)  arc (0 : 14.04 : 6) node[shift={(0.1, -0.07)}]{$\alpha$}; % alpha
		\draw[black] ([shift=(14.04 : 2)] 194.04 : 16.5024)  arc (14.04 : 90 : 2) node[shift={(0.17, 0.05)}]{$\beta$}; % beta		
		\draw[black] ([shift=(142.02 : 20.3116)] 270 : 3)  arc (270 : 322.02 : 3) node[shift={(-0.07, -0.17)}]{$\gamma$}; % gamma


		% locate arc
%		\draw[black] ([shift=(-40 : 16.5024)] 194.04 : 16.5024) arc (-40  : 105 : 16.5024); % The arc
		\draw[black, thick] (194.04 : 16.5024) circle (16.5024); % The arc
		
		% radius lines
		\draw[black, -{Latex[scale=1.0]}] (194.04 : 16.5024) -- +(90 : 16.5024) node[shift={(-94 : 0.9)}]{$r$};
		\draw[black, -{Latex[scale=1.0]}] (194.04 : 16.5024) -- +(14.04 : 16.5024) node[shift={(200 : 0.75)}]{$r$};

		% Labels
		\draw[black] (197 : 17.5)  node{$A$};
		\draw[black] (50 : 1.3)  node{$B$};
		\draw[black] ((139.6 : 22)  node{$C$};
\draw[black] ((86 : 13.4)  node{$D$};

	\end{scope}
		





\end{tikzpicture}
%		\caption{Fig A1}
%		\label{fig:fig_A1}
%	\end{subfigure}
%	\begin{subfigure}{1.0\textwidth}
%		\centering
%		
\begin{tikzpicture}[x=1in, y=1in]
	\begin{scope}[xscale=.10, yscale=.10]	

%		\coordinate (origin) at (0,0);
%		\coordinate (P1) at (-0.0134, -0.0040);
%		\coordinate (P2) at (.0214, .0064);
%		\coordinate (P3) at (0.0086, 0.0064);
%		\coordinate (Pf) at (0.015, -0.015);
%		\coordinate (Pb) at (0.030, 0);
%		\coordinate (Pm) at (0.0075, -0.0075);		

		\clip (-21, -10) rectangle (10, 17);

		% axes
		\draw[lightgray, -{Latex[scale=1.5]}] (0, -30.0) -- (0, 15.0)  node [above] {$y$};	
		\draw[lightgray, -{Latex[scale=1.5]}] (-40.0, 0) -- (8.0, 0)  node [right] {$x$};		

		% vertical
		\draw[lightgray] (-16.0097, -35) -- (-16.0097, 35); % left

		% horizontal
		\draw[lightgray] (-50, 12.5) -- (90, 12.5) node[shift={(179.3 : 8.2)}]{$L_H$}; % L_H
		\draw[lightgray] (-19.0097, -4.0024) -- (-9.0097, -4.0024);	 % R center mark

		% sloped
		\draw[lightgray] (194.04 : 58) -- (14.04 : 98) node[shift={(-166.6 : 9.0)}]{$L_S$}; % L_S

		% bisector
		\draw[lightgray, dashed] (142.02 : -4) -- (142.02 : 24); % dashed line
		\draw[lightgray] ([shift=(135 :  20.3116)] 0, 0)  arc (135 : 150 : 20.3116); % mark

		% angle labels
		\draw[black] ([shift=(180 : 6)] 0, 0)  arc (180 : 194.04 : 6) node[shift={(-0.11, 0.06)}]{$\alpha$} ; % alpha
		\draw[black] ([shift=(194.04 : 2)] 0, 0)  arc (194.04 : 270 : 2) node[shift={(-0.17, -0.05)}]{$\beta$}; % beta
		\draw[black] ([shift=(142.02 : 4)] 0, 0)  arc (142.02 : 194.04 : 4) node[shift={(-0.06, 0.24)}]{$\gamma$}; % gamma

\draw[black] ([shift=(194.04 : 2.5)] 0, 0)  arc (194.04 : 90 : 2.5) node[shift={(-0.18, 0.02)}]{$\delta$}; % gamma

		\draw[black] ([shift=(0 : 6)] 194.04 : 16.5024)  arc (0 : 14.04 : 6) node[shift={(0.1, -0.07)}]{$\alpha$}; % alpha
		\draw[black] ([shift=(14.04 : 2)] 194.04 : 16.5024)  arc (14.04 : 90 : 2) node[shift={(0.17, 0.05)}]{$\beta$}; % beta		
		\draw[black] ([shift=(142.02 : 20.3116)] 270 : 3)  arc (270 : 322.02 : 3) node[shift={(-0.07, -0.17)}]{$\gamma$}; % gamma


		% locate arc
%		\draw[black] ([shift=(-40 : 16.5024)] 194.04 : 16.5024) arc (-40  : 105 : 16.5024); % The arc
		\draw[black, thick] (194.04 : 16.5024) circle (16.5024); % The arc
		
		% radius lines
		\draw[black, -{Latex[scale=1.0]}] (194.04 : 16.5024) -- +(90 : 16.5024) node[shift={(-94 : 0.9)}]{$r$};
		\draw[black, -{Latex[scale=1.0]}] (194.04 : 16.5024) -- +(14.04 : 16.5024) node[shift={(200 : 0.75)}]{$r$};

		% Labels
		\draw[black] (197 : 17.5)  node{$A$};
		\draw[black] (50 : 1.3)  node{$B$};
		\draw[black] ((139.6 : 22)  node{$C$};
\draw[black] ((86 : 13.4)  node{$D$};

	\end{scope}
		





\end{tikzpicture}
%		\caption{Fig A2}
%		\label{fig:fig_A2}
%	\end{subfigure}
%	\begin{subfigure}{1.0\textwidth}
%		\centering
%		
\begin{tikzpicture}[x=1in, y=1in]
	\begin{scope}[xscale=.10, yscale=.10]	

%		\coordinate (origin) at (0,0);
%		\coordinate (P1) at (-0.0134, -0.0040);
%		\coordinate (P2) at (.0214, .0064);
%		\coordinate (P3) at (0.0086, 0.0064);
%		\coordinate (Pf) at (0.015, -0.015);
%		\coordinate (Pb) at (0.030, 0);
%		\coordinate (Pm) at (0.0075, -0.0075);		

		\clip (-21, -10) rectangle (10, 17);

		% axes
		\draw[lightgray, -{Latex[scale=1.5]}] (0, -30.0) -- (0, 15.0)  node [above] {$y$};	
		\draw[lightgray, -{Latex[scale=1.5]}] (-40.0, 0) -- (8.0, 0)  node [right] {$x$};		

		% vertical
		\draw[lightgray] (-16.0097, -35) -- (-16.0097, 35); % left

		% horizontal
		\draw[lightgray] (-50, 12.5) -- (90, 12.5) node[shift={(179.3 : 8.2)}]{$L_H$}; % L_H
		\draw[lightgray] (-19.0097, -4.0024) -- (-9.0097, -4.0024);	 % R center mark

		% sloped
		\draw[lightgray] (194.04 : 58) -- (14.04 : 98) node[shift={(-166.6 : 9.0)}]{$L_S$}; % L_S

		% bisector
		\draw[lightgray, dashed] (142.02 : -4) -- (142.02 : 24); % dashed line
		\draw[lightgray] ([shift=(135 :  20.3116)] 0, 0)  arc (135 : 150 : 20.3116); % mark

		% angle labels
		\draw[black] ([shift=(180 : 6)] 0, 0)  arc (180 : 194.04 : 6) node[shift={(-0.11, 0.06)}]{$\alpha$} ; % alpha
		\draw[black] ([shift=(194.04 : 2)] 0, 0)  arc (194.04 : 270 : 2) node[shift={(-0.17, -0.05)}]{$\beta$}; % beta
		\draw[black] ([shift=(142.02 : 4)] 0, 0)  arc (142.02 : 194.04 : 4) node[shift={(-0.06, 0.24)}]{$\gamma$}; % gamma

\draw[black] ([shift=(194.04 : 2.5)] 0, 0)  arc (194.04 : 90 : 2.5) node[shift={(-0.18, 0.02)}]{$\delta$}; % gamma

		\draw[black] ([shift=(0 : 6)] 194.04 : 16.5024)  arc (0 : 14.04 : 6) node[shift={(0.1, -0.07)}]{$\alpha$}; % alpha
		\draw[black] ([shift=(14.04 : 2)] 194.04 : 16.5024)  arc (14.04 : 90 : 2) node[shift={(0.17, 0.05)}]{$\beta$}; % beta		
		\draw[black] ([shift=(142.02 : 20.3116)] 270 : 3)  arc (270 : 322.02 : 3) node[shift={(-0.07, -0.17)}]{$\gamma$}; % gamma


		% locate arc
%		\draw[black] ([shift=(-40 : 16.5024)] 194.04 : 16.5024) arc (-40  : 105 : 16.5024); % The arc
		\draw[black, thick] (194.04 : 16.5024) circle (16.5024); % The arc
		
		% radius lines
		\draw[black, -{Latex[scale=1.0]}] (194.04 : 16.5024) -- +(90 : 16.5024) node[shift={(-94 : 0.9)}]{$r$};
		\draw[black, -{Latex[scale=1.0]}] (194.04 : 16.5024) -- +(14.04 : 16.5024) node[shift={(200 : 0.75)}]{$r$};

		% Labels
		\draw[black] (197 : 17.5)  node{$A$};
		\draw[black] (50 : 1.3)  node{$B$};
		\draw[black] ((139.6 : 22)  node{$C$};
\draw[black] ((86 : 13.4)  node{$D$};

	\end{scope}
		





\end{tikzpicture}
%		\caption{Fig A2}
%		\label{fig:fig_A2}
%	\end{subfigure}
%	\begin{subfigure}{1.0\textwidth}
%		\centering
%		
\begin{tikzpicture}[x=1in, y=1in]
	\begin{scope}[xscale=.10, yscale=.10]	

%		\coordinate (origin) at (0,0);
%		\coordinate (P1) at (-0.0134, -0.0040);
%		\coordinate (P2) at (.0214, .0064);
%		\coordinate (P3) at (0.0086, 0.0064);
%		\coordinate (Pf) at (0.015, -0.015);
%		\coordinate (Pb) at (0.030, 0);
%		\coordinate (Pm) at (0.0075, -0.0075);		

		\clip (-21, -10) rectangle (10, 17);

		% axes
		\draw[lightgray, -{Latex[scale=1.5]}] (0, -30.0) -- (0, 15.0)  node [above] {$y$};	
		\draw[lightgray, -{Latex[scale=1.5]}] (-40.0, 0) -- (8.0, 0)  node [right] {$x$};		

		% vertical
		\draw[lightgray] (-16.0097, -35) -- (-16.0097, 35); % left

		% horizontal
		\draw[lightgray] (-50, 12.5) -- (90, 12.5) node[shift={(179.3 : 8.2)}]{$L_H$}; % L_H
		\draw[lightgray] (-19.0097, -4.0024) -- (-9.0097, -4.0024);	 % R center mark

		% sloped
		\draw[lightgray] (194.04 : 58) -- (14.04 : 98) node[shift={(-166.6 : 9.0)}]{$L_S$}; % L_S

		% bisector
		\draw[lightgray, dashed] (142.02 : -4) -- (142.02 : 24); % dashed line
		\draw[lightgray] ([shift=(135 :  20.3116)] 0, 0)  arc (135 : 150 : 20.3116); % mark

		% angle labels
		\draw[black] ([shift=(180 : 6)] 0, 0)  arc (180 : 194.04 : 6) node[shift={(-0.11, 0.06)}]{$\alpha$} ; % alpha
		\draw[black] ([shift=(194.04 : 2)] 0, 0)  arc (194.04 : 270 : 2) node[shift={(-0.17, -0.05)}]{$\beta$}; % beta
		\draw[black] ([shift=(142.02 : 4)] 0, 0)  arc (142.02 : 194.04 : 4) node[shift={(-0.06, 0.24)}]{$\gamma$}; % gamma

\draw[black] ([shift=(194.04 : 2.5)] 0, 0)  arc (194.04 : 90 : 2.5) node[shift={(-0.18, 0.02)}]{$\delta$}; % gamma

		\draw[black] ([shift=(0 : 6)] 194.04 : 16.5024)  arc (0 : 14.04 : 6) node[shift={(0.1, -0.07)}]{$\alpha$}; % alpha
		\draw[black] ([shift=(14.04 : 2)] 194.04 : 16.5024)  arc (14.04 : 90 : 2) node[shift={(0.17, 0.05)}]{$\beta$}; % beta		
		\draw[black] ([shift=(142.02 : 20.3116)] 270 : 3)  arc (270 : 322.02 : 3) node[shift={(-0.07, -0.17)}]{$\gamma$}; % gamma


		% locate arc
%		\draw[black] ([shift=(-40 : 16.5024)] 194.04 : 16.5024) arc (-40  : 105 : 16.5024); % The arc
		\draw[black, thick] (194.04 : 16.5024) circle (16.5024); % The arc
		
		% radius lines
		\draw[black, -{Latex[scale=1.0]}] (194.04 : 16.5024) -- +(90 : 16.5024) node[shift={(-94 : 0.9)}]{$r$};
		\draw[black, -{Latex[scale=1.0]}] (194.04 : 16.5024) -- +(14.04 : 16.5024) node[shift={(200 : 0.75)}]{$r$};

		% Labels
		\draw[black] (197 : 17.5)  node{$A$};
		\draw[black] (50 : 1.3)  node{$B$};
		\draw[black] ((139.6 : 22)  node{$C$};
\draw[black] ((86 : 13.4)  node{$D$};

	\end{scope}
		





\end{tikzpicture}
%		\caption{Fig A2}
%		\label{fig:fig_A2}
%	\end{subfigure}
%	\caption{Stack of Subfigures}
%	\label{fig:fig_aa}
%\end{figure*}


\begin{figure}
	\centering
	\input{python/draw/output/fig1.tikz}
	\caption{\\Establishing $p_s$}
	\label{fig:fig_f1}
\end{figure}
\begin{figure}
	\centering
	\input{python/draw/output/fig2.tikz}
	\caption{\\Locating $p_1$}
	\label{fig:fig_f2}
\end{figure}

\begin{figure}
	\centering
	\input{python/draw/output/fig3.tikz}
	\caption{\\Locating $p_g$}
	\label{fig:fig_f3}
\end{figure}

\begin{figure}
	\centering
	\input{python/draw/output/fig4.tikz}
	\caption{\\Establishing coordinate system}
	\label{fig:fig_f4}
\end{figure}

\begin{figure}
	\centering
	\input{python/draw/output/fig5.tikz}
	\caption{\\Locating $p_2$}
	\label{fig:fig_f5}
\end{figure}

\begin{figure}
	\centering
	\input{python/draw/output/fig6.tikz}
	\caption{\\Drawing R2}
	\label{fig:fig_f6}
\end{figure}

\begin{figure}
	\centering
	\input{python/draw/output/fig7.tikz}
	\caption{\\Drawing vertical guidelines}
	\label{fig:fig_f7}
\end{figure}

\begin{figure}
	\centering
	\input{python/draw/output/fig8.tikz}
	\caption{\\Locating $p_3$}
	\label{fig:fig_f8}
\end{figure}

\begin{figure}
	\centering
	\input{python/draw/output/fig9.tikz}
	\caption{\\Drawing R3}
	\label{fig:fig_f9}
\end{figure}

\begin{figure*}
	\centering
	\input{python/draw/output/fig10.tikz}
	\caption{\\Completing Tread}
	\label{fig:fig_f10}
\end{figure*}

%\begin{figure}
%	\centering
%	
\begin{tikzpicture}[x=1in, y=1in]
	\begin{scope}[xscale=.10, yscale=.10]	

%		\coordinate (origin) at (0,0);
%		\coordinate (P1) at (-0.0134, -0.0040);
%		\coordinate (P2) at (.0214, .0064);
%		\coordinate (P3) at (0.0086, 0.0064);
%		\coordinate (Pf) at (0.015, -0.015);
%		\coordinate (Pb) at (0.030, 0);
%		\coordinate (Pm) at (0.0075, -0.0075);		

		\clip (-21, -10) rectangle (10, 17);

		% axes
		\draw[lightgray, -{Latex[scale=1.5]}] (0, -30.0) -- (0, 15.0)  node [above] {$y$};	
		\draw[lightgray, -{Latex[scale=1.5]}] (-40.0, 0) -- (8.0, 0)  node [right] {$x$};		

		% vertical
		\draw[lightgray] (-16.0097, -35) -- (-16.0097, 35); % left

		% horizontal
		\draw[lightgray] (-50, 12.5) -- (90, 12.5) node[shift={(179.3 : 8.2)}]{$L_H$}; % L_H
		\draw[lightgray] (-19.0097, -4.0024) -- (-9.0097, -4.0024);	 % R center mark

		% sloped
		\draw[lightgray] (194.04 : 58) -- (14.04 : 98) node[shift={(-166.6 : 9.0)}]{$L_S$}; % L_S

		% bisector
		\draw[lightgray, dashed] (142.02 : -4) -- (142.02 : 24); % dashed line
		\draw[lightgray] ([shift=(135 :  20.3116)] 0, 0)  arc (135 : 150 : 20.3116); % mark

		% angle labels
		\draw[black] ([shift=(180 : 6)] 0, 0)  arc (180 : 194.04 : 6) node[shift={(-0.11, 0.06)}]{$\alpha$} ; % alpha
		\draw[black] ([shift=(194.04 : 2)] 0, 0)  arc (194.04 : 270 : 2) node[shift={(-0.17, -0.05)}]{$\beta$}; % beta
		\draw[black] ([shift=(142.02 : 4)] 0, 0)  arc (142.02 : 194.04 : 4) node[shift={(-0.06, 0.24)}]{$\gamma$}; % gamma

\draw[black] ([shift=(194.04 : 2.5)] 0, 0)  arc (194.04 : 90 : 2.5) node[shift={(-0.18, 0.02)}]{$\delta$}; % gamma

		\draw[black] ([shift=(0 : 6)] 194.04 : 16.5024)  arc (0 : 14.04 : 6) node[shift={(0.1, -0.07)}]{$\alpha$}; % alpha
		\draw[black] ([shift=(14.04 : 2)] 194.04 : 16.5024)  arc (14.04 : 90 : 2) node[shift={(0.17, 0.05)}]{$\beta$}; % beta		
		\draw[black] ([shift=(142.02 : 20.3116)] 270 : 3)  arc (270 : 322.02 : 3) node[shift={(-0.07, -0.17)}]{$\gamma$}; % gamma


		% locate arc
%		\draw[black] ([shift=(-40 : 16.5024)] 194.04 : 16.5024) arc (-40  : 105 : 16.5024); % The arc
		\draw[black, thick] (194.04 : 16.5024) circle (16.5024); % The arc
		
		% radius lines
		\draw[black, -{Latex[scale=1.0]}] (194.04 : 16.5024) -- +(90 : 16.5024) node[shift={(-94 : 0.9)}]{$r$};
		\draw[black, -{Latex[scale=1.0]}] (194.04 : 16.5024) -- +(14.04 : 16.5024) node[shift={(200 : 0.75)}]{$r$};

		% Labels
		\draw[black] (197 : 17.5)  node{$A$};
		\draw[black] (50 : 1.3)  node{$B$};
		\draw[black] ((139.6 : 22)  node{$C$};
\draw[black] ((86 : 13.4)  node{$D$};

	\end{scope}
		





\end{tikzpicture}
%	\caption{\\Locating R2, Case 2: constraints A \& C satisfied}
%	\label{fig:fig_A3}
%\end{figure}
%\begin{figure}
%	\centering
%	
\begin{tikzpicture}[x=1in, y=1in]
	\begin{scope}[xscale=.10, yscale=.10]	

%		\coordinate (origin) at (0,0);
%		\coordinate (P1) at (-0.0134, -0.0040);
%		\coordinate (P2) at (.0214, .0064);
%		\coordinate (P3) at (0.0086, 0.0064);
%		\coordinate (Pf) at (0.015, -0.015);
%		\coordinate (Pb) at (0.030, 0);
%		\coordinate (Pm) at (0.0075, -0.0075);		

		\clip (-21, -10) rectangle (10, 17);

		% axes
		\draw[lightgray, -{Latex[scale=1.5]}] (0, -30.0) -- (0, 15.0)  node [above] {$y$};	
		\draw[lightgray, -{Latex[scale=1.5]}] (-40.0, 0) -- (8.0, 0)  node [right] {$x$};		

		% vertical
		\draw[lightgray] (-16.0097, -35) -- (-16.0097, 35); % left

		% horizontal
		\draw[lightgray] (-50, 12.5) -- (90, 12.5) node[shift={(179.3 : 8.2)}]{$L_H$}; % L_H
		\draw[lightgray] (-19.0097, -4.0024) -- (-9.0097, -4.0024);	 % R center mark

		% sloped
		\draw[lightgray] (194.04 : 58) -- (14.04 : 98) node[shift={(-166.6 : 9.0)}]{$L_S$}; % L_S

		% bisector
		\draw[lightgray, dashed] (142.02 : -4) -- (142.02 : 24); % dashed line
		\draw[lightgray] ([shift=(135 :  20.3116)] 0, 0)  arc (135 : 150 : 20.3116); % mark

		% angle labels
		\draw[black] ([shift=(180 : 6)] 0, 0)  arc (180 : 194.04 : 6) node[shift={(-0.11, 0.06)}]{$\alpha$} ; % alpha
		\draw[black] ([shift=(194.04 : 2)] 0, 0)  arc (194.04 : 270 : 2) node[shift={(-0.17, -0.05)}]{$\beta$}; % beta
		\draw[black] ([shift=(142.02 : 4)] 0, 0)  arc (142.02 : 194.04 : 4) node[shift={(-0.06, 0.24)}]{$\gamma$}; % gamma

\draw[black] ([shift=(194.04 : 2.5)] 0, 0)  arc (194.04 : 90 : 2.5) node[shift={(-0.18, 0.02)}]{$\delta$}; % gamma

		\draw[black] ([shift=(0 : 6)] 194.04 : 16.5024)  arc (0 : 14.04 : 6) node[shift={(0.1, -0.07)}]{$\alpha$}; % alpha
		\draw[black] ([shift=(14.04 : 2)] 194.04 : 16.5024)  arc (14.04 : 90 : 2) node[shift={(0.17, 0.05)}]{$\beta$}; % beta		
		\draw[black] ([shift=(142.02 : 20.3116)] 270 : 3)  arc (270 : 322.02 : 3) node[shift={(-0.07, -0.17)}]{$\gamma$}; % gamma


		% locate arc
%		\draw[black] ([shift=(-40 : 16.5024)] 194.04 : 16.5024) arc (-40  : 105 : 16.5024); % The arc
		\draw[black, thick] (194.04 : 16.5024) circle (16.5024); % The arc
		
		% radius lines
		\draw[black, -{Latex[scale=1.0]}] (194.04 : 16.5024) -- +(90 : 16.5024) node[shift={(-94 : 0.9)}]{$r$};
		\draw[black, -{Latex[scale=1.0]}] (194.04 : 16.5024) -- +(14.04 : 16.5024) node[shift={(200 : 0.75)}]{$r$};

		% Labels
		\draw[black] (197 : 17.5)  node{$A$};
		\draw[black] (50 : 1.3)  node{$B$};
		\draw[black] ((139.6 : 22)  node{$C$};
\draw[black] ((86 : 13.4)  node{$D$};

	\end{scope}
		





\end{tikzpicture}
%	\caption{\\Locating R2, Case 3: constraints B \& C satisfied}
%	\label{fig:fig_A4}
%\end{figure}

%
\begin{tikzpicture}[x=1in, y=1in]
	\begin{scope}[xscale=.10, yscale=.10]	

%		\coordinate (origin) at (0,0);
%		\coordinate (P1) at (-0.0134, -0.0040);
%		\coordinate (P2) at (.0214, .0064);
%		\coordinate (P3) at (0.0086, 0.0064);
%		\coordinate (Pf) at (0.015, -0.015);
%		\coordinate (Pb) at (0.030, 0);
%		\coordinate (Pm) at (0.0075, -0.0075);		

		\clip (-21, -10) rectangle (10, 17);

		% axes
		\draw[lightgray, -{Latex[scale=1.5]}] (0, -30.0) -- (0, 15.0)  node [above] {$y$};	
		\draw[lightgray, -{Latex[scale=1.5]}] (-40.0, 0) -- (8.0, 0)  node [right] {$x$};		

		% vertical
		\draw[lightgray] (-16.0097, -35) -- (-16.0097, 35); % left

		% horizontal
		\draw[lightgray] (-50, 12.5) -- (90, 12.5) node[shift={(179.3 : 8.2)}]{$L_H$}; % L_H
		\draw[lightgray] (-19.0097, -4.0024) -- (-9.0097, -4.0024);	 % R center mark

		% sloped
		\draw[lightgray] (194.04 : 58) -- (14.04 : 98) node[shift={(-166.6 : 9.0)}]{$L_S$}; % L_S

		% bisector
		\draw[lightgray, dashed] (142.02 : -4) -- (142.02 : 24); % dashed line
		\draw[lightgray] ([shift=(135 :  20.3116)] 0, 0)  arc (135 : 150 : 20.3116); % mark

		% angle labels
		\draw[black] ([shift=(180 : 6)] 0, 0)  arc (180 : 194.04 : 6) node[shift={(-0.11, 0.06)}]{$\alpha$} ; % alpha
		\draw[black] ([shift=(194.04 : 2)] 0, 0)  arc (194.04 : 270 : 2) node[shift={(-0.17, -0.05)}]{$\beta$}; % beta
		\draw[black] ([shift=(142.02 : 4)] 0, 0)  arc (142.02 : 194.04 : 4) node[shift={(-0.06, 0.24)}]{$\gamma$}; % gamma

\draw[black] ([shift=(194.04 : 2.5)] 0, 0)  arc (194.04 : 90 : 2.5) node[shift={(-0.18, 0.02)}]{$\delta$}; % gamma

		\draw[black] ([shift=(0 : 6)] 194.04 : 16.5024)  arc (0 : 14.04 : 6) node[shift={(0.1, -0.07)}]{$\alpha$}; % alpha
		\draw[black] ([shift=(14.04 : 2)] 194.04 : 16.5024)  arc (14.04 : 90 : 2) node[shift={(0.17, 0.05)}]{$\beta$}; % beta		
		\draw[black] ([shift=(142.02 : 20.3116)] 270 : 3)  arc (270 : 322.02 : 3) node[shift={(-0.07, -0.17)}]{$\gamma$}; % gamma


		% locate arc
%		\draw[black] ([shift=(-40 : 16.5024)] 194.04 : 16.5024) arc (-40  : 105 : 16.5024); % The arc
		\draw[black, thick] (194.04 : 16.5024) circle (16.5024); % The arc
		
		% radius lines
		\draw[black, -{Latex[scale=1.0]}] (194.04 : 16.5024) -- +(90 : 16.5024) node[shift={(-94 : 0.9)}]{$r$};
		\draw[black, -{Latex[scale=1.0]}] (194.04 : 16.5024) -- +(14.04 : 16.5024) node[shift={(200 : 0.75)}]{$r$};

		% Labels
		\draw[black] (197 : 17.5)  node{$A$};
		\draw[black] (50 : 1.3)  node{$B$};
		\draw[black] ((139.6 : 22)  node{$C$};
\draw[black] ((86 : 13.4)  node{$D$};

	\end{scope}
		





\end{tikzpicture}
%\caption{Tangent Ogive Alternate Construction}
%\label{fig:fig_a}

\subsection{Axes and Layout lines}

\subsection{Locate P1}
- Discussion of R1 location.
- Gage point should be referenced from 0deg treadline.  
- Otherwise relationship between tread and flange changes.
- Discussion of taper angle.  
- Inspection gage designed to make contact at gage point Pg.  Fillet radius bears against the fillet.  Until taper point.  Beyond the tread taper rises.
- 3deg layout lines

\subsection{Locate P2}
Find R2 center given Constraints(Conditions) 
- Draw
- Problem
- Ambiguous.  
- Draw R2, no tangent with R1.  Centers unconnected.  Differs with the note.
- How are centers of R2 and R3 determined.  Notes text, Doesn't agree with drawing.

- Approaches fall in to two categories
	- ignore the point gage point (Pg) requirement
	- ignore the published values of R2 and R3, calculate new values R2' and R3'
- Three suggested solutions
	- None guaranteed to pass NMRA gage test.  
	- Performance may vary.  
	- Selection depends on materials (metal vs. plastic), process (casting, forging, machining), tooling vs. moulding
1) Vertical connecting line.  Pros: simple to draw, Cons: discontinuity
2) Tangent connecting line.  Pros: smooth transition, Cons: harder to find tangent points
3) Change published values of R2, R3.  Pros: easy to draw, complies with (Pg) tangent requirement,  Cons: R3 discontinuity at wheelback, must be calculated.
  - How to locate centerpoint.  bi-sect isosceles triangle method
4) Custom solution flange=square, round, angled

\subsubsection{Constraints}
\begin{itemize}
\item Constraint A: Radius is R2=0.018
\item Constraint B: Tangent to R1 at Pg (passes thru Pg, Pg is on R2 circle)
\item Constraint C: Passes thru ogive tip (tip is on R2 circle)
\end{itemize}

All three constrains cannot be satisfied!  Thus the source of confusion.

\subsubsection{Case1}
 Case 1 (cyan): Constraints AB, C not satisfied 
\begin{itemize}
\item (Pg is on circle, but not tangent, discontinuity at Pg, furthermore ogive tip not on circle(bad!))
\item Disagrees with RP25 note 3
\item Flange too short, too flat and broad.
\item Some have suggested to drop R2 circle down and connect R1 and R2 with a vertical line.  
	\item Results in 2 discontinuities where R1 intersects vert line and where R2 intersects vert line
\item A sloped connecting tangent line may work.
\item  Eitherway R2 center must be lowered (hackish!) and R2 won't be tangent to R1.
\end{itemize}

\subsubsection{Case2}
Case 2 (yellow): Constraints AC, B not satisfied (Pg is on circle, but not tangent, discontinuity at Pg, ogive tip on circle (better! no hacks required))
		- Disagrees with RP25 note 3

\subsubsection{Case3}
Case 3 (white): Constraints BC, A not satisfied (although a new R2' can be calculated, best!)
\begin{itemize}
\item R2 and R1 will be tangent at Pg.  No discontinuity at Pg. (ok)
\item Ogive tip is on R2 (ok)
\item Results in narrower/sharper ogive (ok, looks better, more clearance when gaged)
\item Method to determine R2' center must be found (bi-sect isosceles triangle)
\item New values for R2' and R3' must be calculated (ok, no problem)
\end{itemize}



\subsection{Locate P3}


- The goal is to get a profile what fits inside envelope of the s4.2 standard.
- In theory many solutions to the problem if stated as profile must fit standard.  
	- Many profiles have been proposed and manufactured.  
	- Reference MR article.
- In practice many solutions fit the NMRA test gage profile.
- Profile could have 



\subsection{Taper}
% 0 - 3 deg












\section{Analytic Solution}

Analytic Geometry
Equations


Tables
- Solution 1,2,3 inch/mm
- Calculate values for R2 and R3.
- Locate all critical points.  Circle centers, tangent points, ogive tip, etc.








While it isIt o as it omits 
- Disambiguation
- How to you draw accurately with the given dimensions?
- How to construct?
- Parametric CAD?




%# S1 Motivation
- Create dimensioned drawing
- Clear drawing prior to fabrication
- Tools, dies, molds

How do I draw RP-25?  Not an easy answer.  But a very rich learning exercise.
Must be understood in context of fuller system of standards and RPs.
Interchangability.  Interchangeable parts from different producers, in different places at different times.
Really how do I deal with inperfection?
Tolerances are difference between math and engineering.
Perfection vs. real world.
Real world is not perfect.
Even if you could measure with infinite you would eventually measure a deviation from ideal.
Manufacturing variation.  Between manufacturers, but also batches, parts.
Variation in rail height, bumpts and dips.  Rail spread and constrictions.  Twists.
Level (lack of) of maintance.
Weather.  Freeze/thaw cycle heaving.  Erosion.  Saturation of sub bed.
Dimensional change.  Materials change with temperature.  Temperature change is not uniform everywhere for all materials. Rails and wheels (tires) are wear 
parts, they change dimension over service-life as they wear.
What is needed, is the "idealized" design AND a system to deal with variations over time and space.
How much inperfection can be tolerated?  What is the acceptable worst-case scenario that still works?

Quite an engineering achievement.  Solution to a difficult problem.  More so to align interests of many different vendors, modelers, clubs, etc. over many 
decades.

To understand RP-25 look at nmra RP-2 standards gage.
On wheels side, gage is really a proxy for the rails (and all it's variations).
On: track, points, flangeways sides, gage is really a proxy for the wheelset.(and all it's variations).



RP-15.1  note railhead flat on top.  Railhead has two points-of-contact with wheel top (along C/R) and side (along H).  Wheel has two points-of-contact with 
railhead (flange tangent) and (gage point).

In truth there are many solutions to the wheel profile problem.
Gage acts as an envelop of a few critical dimensions and tolerances.
Standard gage (wheels): Does wheel fit in the slot(s)?
Primitive box profile.  Tread and flange.  Does it fit?
Point of tread contact.  Clearance (fillet rolloff vs. railhead radius).  Gage point and gage point point-of-contact
S-4.2 Dmax.  Flange can be shorter but not taller.
S-4.2 tolerance of K, B and Tmax requirement.  Almost like a budget, you can spend tolerance on K and/or B but total tolerance cannot exceed Tmax.
Gage has rounded bottom wheel slot.  Flange must fit in this envelope.  Otherwise box flange would work.  Okay, round off bottom of flange so now standard is 
satisfied.













\section{Construction}
 Constructive Geometry
Todo: Write me.


%\begin{figure}[htbp]
%  \centering
%  \includesvg{drawing.svg}
%  \caption{svg image}
%\end{figure}

\begin{figure}[ht!]
\centering

\begin{tikzpicture}[x=1in, y=1in]
	\begin{scope}[xscale=.10, yscale=.10]	

%		\coordinate (origin) at (0,0);
%		\coordinate (P1) at (-0.0134, -0.0040);
%		\coordinate (P2) at (.0214, .0064);
%		\coordinate (P3) at (0.0086, 0.0064);
%		\coordinate (Pf) at (0.015, -0.015);
%		\coordinate (Pb) at (0.030, 0);
%		\coordinate (Pm) at (0.0075, -0.0075);		

		\clip (-21, -10) rectangle (10, 17);

		% axes
		\draw[lightgray, -{Latex[scale=1.5]}] (0, -30.0) -- (0, 15.0)  node [above] {$y$};	
		\draw[lightgray, -{Latex[scale=1.5]}] (-40.0, 0) -- (8.0, 0)  node [right] {$x$};		

		% vertical
		\draw[lightgray] (-16.0097, -35) -- (-16.0097, 35); % left

		% horizontal
		\draw[lightgray] (-50, 12.5) -- (90, 12.5) node[shift={(179.3 : 8.2)}]{$L_H$}; % L_H
		\draw[lightgray] (-19.0097, -4.0024) -- (-9.0097, -4.0024);	 % R center mark

		% sloped
		\draw[lightgray] (194.04 : 58) -- (14.04 : 98) node[shift={(-166.6 : 9.0)}]{$L_S$}; % L_S

		% bisector
		\draw[lightgray, dashed] (142.02 : -4) -- (142.02 : 24); % dashed line
		\draw[lightgray] ([shift=(135 :  20.3116)] 0, 0)  arc (135 : 150 : 20.3116); % mark

		% angle labels
		\draw[black] ([shift=(180 : 6)] 0, 0)  arc (180 : 194.04 : 6) node[shift={(-0.11, 0.06)}]{$\alpha$} ; % alpha
		\draw[black] ([shift=(194.04 : 2)] 0, 0)  arc (194.04 : 270 : 2) node[shift={(-0.17, -0.05)}]{$\beta$}; % beta
		\draw[black] ([shift=(142.02 : 4)] 0, 0)  arc (142.02 : 194.04 : 4) node[shift={(-0.06, 0.24)}]{$\gamma$}; % gamma

\draw[black] ([shift=(194.04 : 2.5)] 0, 0)  arc (194.04 : 90 : 2.5) node[shift={(-0.18, 0.02)}]{$\delta$}; % gamma

		\draw[black] ([shift=(0 : 6)] 194.04 : 16.5024)  arc (0 : 14.04 : 6) node[shift={(0.1, -0.07)}]{$\alpha$}; % alpha
		\draw[black] ([shift=(14.04 : 2)] 194.04 : 16.5024)  arc (14.04 : 90 : 2) node[shift={(0.17, 0.05)}]{$\beta$}; % beta		
		\draw[black] ([shift=(142.02 : 20.3116)] 270 : 3)  arc (270 : 322.02 : 3) node[shift={(-0.07, -0.17)}]{$\gamma$}; % gamma


		% locate arc
%		\draw[black] ([shift=(-40 : 16.5024)] 194.04 : 16.5024) arc (-40  : 105 : 16.5024); % The arc
		\draw[black, thick] (194.04 : 16.5024) circle (16.5024); % The arc
		
		% radius lines
		\draw[black, -{Latex[scale=1.0]}] (194.04 : 16.5024) -- +(90 : 16.5024) node[shift={(-94 : 0.9)}]{$r$};
		\draw[black, -{Latex[scale=1.0]}] (194.04 : 16.5024) -- +(14.04 : 16.5024) node[shift={(200 : 0.75)}]{$r$};

		% Labels
		\draw[black] (197 : 17.5)  node{$A$};
		\draw[black] (50 : 1.3)  node{$B$};
		\draw[black] ((139.6 : 22)  node{$C$};
\draw[black] ((86 : 13.4)  node{$D$};

	\end{scope}
		





\end{tikzpicture}
\caption{Tangent Ogive Alternate Construction}
\label{fig:fig_a}
\end{figure}


\subsection{Strategy}
Reduce problem to it's essence.
% Tread, fillet and flange
% The essence of the solution is locating the centers of 3 circles.
- Drawing order suggested by the spec authors
- R1, R2, R3
- or R3, R2, R1???  Results in treadline above P. 

- Constructive geometry
- Then analytic geometry to make drawing on CAD easier.
- Draw tread, Draw flange
- Reduce problem to essence.  Trying to find the centers of 3 circles such they are arranged in a specific way.  We are just trying to arrange 3 circles.
- Find R2 center given Constraints(Conditions) 
	- Constraint A: Radius is R2=0.018
	- Constraint B: Tangent to R1 at Pg (passes thru Pg, Pg is on R2 circle)
	- Constraint C: Passes thru ogive tip (tip is on R2 circle)



\subsection{Layout lines}
\subsection{Locating R1}
- Discussion of R1 location.
- Gage point should be referenced from 0deg treadline.  
- Otherwise relationship between tread and flange changes.
- Discussion of taper angle.  
- Inspection gage designed to make contact at gage point Pg.  Fillet radius bears against the fillet.  Until taper point.  Beyond the tread taper rises.
- 3deg layout lines


\subsection{Locating R2}
Find R2 center given Constraints(Conditions) 

- Draw
- Problem
- Ambiguous.  
- Draw R2, no tangent with R1.  Centers unconnected.  Differs with the note.
- How are centers of R2 and R3 determined.  Notes text, Doesn't agree with drawing.

- Approaches fall in to two categories
	- ignore the point gage point (Pg) requirement
	- ignore the published values of R2 and R3, calculate new values R2' and R3'
- Three suggested solutions
	- None guaranteed to pass NMRA gage test.  
	- Performance may vary.  
	- Selection depends on materials (metal vs. plastic), process (casting, forging, machining), tooling vs. moulding
1) Vertical connecting line.  Pros: simple to draw, Cons: discontinuity
2) Tangent connecting line.  Pros: smooth transition, Cons: harder to find tangent points
3) Change published values of R2, R3.  Pros: easy to draw, complies with (Pg) tangent requirement,  Cons: R3 discontinuity at wheelback, must be calculated.
  - How to locate centerpoint.  bi-sect isosceles triangle method
4) Custom solution flange=square, round, angled





\subsubsection{Constraints}
\begin{itemize}
\item Constraint A: Radius is R2=0.018
\item Constraint B: Tangent to R1 at Pg (passes thru Pg, Pg is on R2 circle)
\item Constraint C: Passes thru ogive tip (tip is on R2 circle)
\end{itemize}

All three constrains cannot be satisfied!  Thus the source of confusion.






\subsubsection{Case1}
 Case 1 (cyan): Constraints AB, C not satisfied 
\begin{itemize}
\item (Pg is on circle, but not tangent, discontinuity at Pg, furthermore ogive tip not on circle(bad!))
\item Disagrees with RP25 note 3
\item Flange too short, too flat and broad.
\item Some have suggested to drop R2 circle down and connect R1 and R2 with a vertical line.  
	\item Results in 2 discontinuities where R1 intersects vert line and where R2 intersects vert line
\item A sloped connecting tangent line may work.
\item  Eitherway R2 center must be lowered (hackish!) and R2 won't be tangent to R1.
\end{itemize}

\subsubsection{Case2}
Case 2 (yellow): Constraints AC, B not satisfied (Pg is on circle, but not tangent, discontinuity at Pg, ogive tip on circle (better! no hacks required))
		- Disagrees with RP25 note 3

\subsubsection{Case3}
Case 3 (white): Constraints BC, A not satisfied (although a new R2' can be calculated, best!)
\begin{itemize}
\item R2 and R1 will be tangent at Pg.  No discontinuity at Pg. (ok)
\item Ogive tip is on R2 (ok)
\item Results in narrower/sharper ogive (ok, looks better, more clearance when gaged)
\item Method to determine R2' center must be found (bi-sect isosceles triangle)
\item New values for R2' and R3' must be calculated (ok, no problem)
\end{itemize}



\subsection{Locating R3}


- The goal is to get a profile what fits inside envelope of the s4.2 standard.
- In theory many solutions to the problem if stated as profile must fit standard.  
	- Many profiles have been proposed and manufactured.  
	- Reference MR article.
- In practice many solutions fit the NMRA test gage profile.
- Profile could have 



\subsection{Taper}
% 0 - 3 deg


%\begin{flushright}   
%Q.E.D.
%\end{flushright}

\section{Analysis}
Analytic Geometry
Equations


Tables
- Solution 1,2,3 inch/mm
- Calculate values for R2 and R3.
- Locate all critical points.  Circle centers, tangent points, ogive tip, etc.





