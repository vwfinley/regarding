\section{Geometric Theorems}
The geometry of Euclid of Alexandria is the key to unlocking RP-25.

Three helpful geometric theorems are declared here.
They will be used throughout the Solution.
They are stated without proof; however, the endnotes section provides referenced sources for the curious.

\paragraph{Theorem 1: Circle Center}
\begin{quote}
The center of a circle (or arc), of radius $r$, can be found by 
constructing arcs of radius $r$ centered about any two points on the circle.
The circle's center shall coincide with intersection
of constructed arcs.\endnote{Machinery's Handbook 25th Edition,
Robert E. Green, Editor
Copyright 1996,
Industrial Press Inc.,
ISBN 0-8311-2424-5,
Page 51, 
Paraphrased from first construction on page}
%\subsection{Theorem 2: Mutually tangent circles}
%\begin{quote}
%If two circles (or arcs) are tangent to each other externally or internally, the line of centers
%passes through the point of tangency.
%\endnote{
%Practical Shop Mathematics,
%Volume 1 Elementary, 
%John H. Wolfe and Everett R. Phelps, Ph.D.
%Copyright 1935 \& 1939,
%McGraw-Hill Book Company, Inc.,
%Page 144,
%Proposition 43}
%\end{quote}
\end{quote}
\paragraph{Theorem 2: Mutually tangent circles}
\begin{quote}
If two circles touch each other at any point, the centres and 
that point are collinear.\endnote{
The First Six Books of the Elements of Euclid, 
3rd Edition,
John Casey and Euclid,
1885,
Hodges, Figgis \& Co., Dublin,
Longmans, Green \& Co., London,
Page 78,
Observation: Propositions XI, XII
https://www.gutenberg.org/files/21076/21076-pdf.pdf
}
\end{quote}
\paragraph{Theorem 3: Bisection of isosceles triangle}
\begin{quote}
The bisector of the vertical angle of an isosceles triangle is perpendicular to the base 
and bisects the base.\endnote{
Practical Shop Mathematics,
Volume 1 Elementary, 
John H. Wolfe and Everett R. Phelps, Ph.D.
Copyright 1935 \& 1939,
McGraw-Hill Book Company, Inc.,
Page 105,
Corollary 1 to Proposition 19}
\end{quote}