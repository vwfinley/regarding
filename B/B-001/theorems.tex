% VWF NOTE: Add drawings to theorems
% VWF NOTE: Add discussion of gage point as origin.


\section{Geometric Theorems}
Three helpful plane geometry theorems are declared here.
They will be used throughout the solution.
They are quoted from their original source, which does provide proofs.
A curious reader will find the source referenced in the endnotes section.


% VWF NOTE: Arcs or chords?  Any two points?  Or points such that they intersect?
\paragraph{Theorem 1: Points on a circle}
\begin{quote}
A circle is a plane figure bounded by a line called the circumference,
all points of which are equidistant from a point within called the center.
\endnote{
Practical Shop Mathematics,
Volume 1,
Definition 35,
Page 126,
John H. Wolfe and Everett R. Phelps,
Copyright 1939,
McGraw-Hill Book Company, Inc.
}
\end{quote}


\paragraph{Theorem 2: Tangent line perpendicular to radius}
\begin{quote}
A straight line perpendicular to a radius (of a circle) at its extremity is tangent to the circle;
conversely, the tangent at the extremity of a radius is perpendicular to the radius.
\endnote{
Practical Shop Mathematics,
Volume 1,
Proposition 36,
Page 132,
John H. Wolfe and Everett R. Phelps,
Copyright 1939,
McGraw-Hill Book Company, Inc.
}
\end{quote}

\paragraph{Theorem 3: Mutually tangent circles}
\begin{quote}
If two circles are tangent to each other externally or
internally, the line of centers passes through the point of tangency.
\endnote{
Practical Shop Mathematics,
Volume 1,
Proposition 43,
Page 144,
John H. Wolfe and Everett R. Phelps,
Copyright 1939,
McGraw-Hill Book Company, Inc.
}
\end{quote}
