% VWF NOTE: Add drawings to theorems
% VWF NOTE: Add discussion of gage point as origin.


\section{Geometric Theorems}
Three helpful plane geometric theorems are declared here.
They will be used throughout the solution.
They are paraphrased from their original sources and are stated here without proof.
A curious reader will find those sources referenced in the endnotes section.


% VWF NOTE: Arcs or chords?  Any two points?  Or points such that they intersect?
\paragraph{Theorem 1: Points on a circle}
\begin{quote}
Every point on a circle of radius $R$ is distance $R$ from the circle center.
As a corollary, the center of a circle is distance $R$ from every point on the circle.
\endnote{tbd}
\end{quote}


\paragraph{Theorem 2: Tangent line perpendicular to centerline}
\begin{quote}
A line that is tangent to some point on a circle, is perpendicular to the line that passes through the tangent point and the circle center.
\endnote{tbd}
\end{quote}

\paragraph{Theorem 3: Mutually tangent circles}
\begin{quote}
If two circles contact each other at a single tangent point, the circle centers and 
the tangent point are collinear.\endnote{
The First Six Books of the Elements of Euclid, 
3rd Edition,
John Casey and Euclid,
1885,
Hodges, Figgis \& Co., Dublin,
Longmans, Green \& Co., London,
Page 78,
Observation: Propositions XI, XII
https://www.gutenberg.org/files/21076/21076-pdf.pdf
}
\end{quote}
