\section{Solution}
The right triangle shown in Figure \ref{fig:fig_a} has equal sides $a$ and $b$ of some length $L$
\begin{equation}
    \label{eqn:a_b_L}
    a = b = L
\end{equation}

Since lengths of $a$ and $b$ are equal, $\triangle{abc}$ is an isosceles triangle.
Furthermore, angles $\theta$ will be: 
\begin{equation}
\theta = \frac{180\degree - 90\degree}{2} = 45\degree
\end{equation}

For a right triangle, side $c$ is related to $a$ and $b$ by the Pythagorean theorem.
By substituting $L = a = b$ from equation \ref{eqn:a_b_L} into the theorem, $L$ is found to be.
\begin{align}
c^2 &= a^2 + b^2 \notag \\
c^2 &= L^2 + L^2 \notag \\
c^2 &= 2 L^2 \notag \\
\sqrt{c^2}   &= \sqrt{2 L^2} \notag \\
c   &= L\sqrt{2} \notag \\
L   &= \frac{c}{\sqrt{2}} \label{eqn:L_c_sqrt2} 
\end{align}

A second smaller right isosceles triangle is formed by $d$ and $c/2$, with hypotenuse $b$. 
Since $d$ bisects $c$, the length of $d$ will be equal to half of $c$.
\begin{align}
     d &= \frac{c}{2} \notag \\ 
     c &= 2 d \label{eqn:c_d} 
\end{align}

Substituting $c$ from equation \ref{eqn:c_d} into equation \ref{eqn:L_c_sqrt2} yields:
\begin{align}
     L &= \frac{c}{\sqrt{2}} \notag \\ 
     L &= \frac{2d}{\sqrt{2}} \notag \\
     L &= \frac{d(\sqrt{2})^2}{\sqrt{2}} \notag \\
     L &= d \sqrt{2} \label{eqn:L_d_sqrt2}
\end{align}


Since the midpoint of hypotenuse $c$ is at the origin, the starting position on the x-axis will be the midpoint of $b$.
Therefore, the travel distance $t_x$ will be half the length of $b$.
Substituting $b = L$ from equation \ref{eqn:a_b_L}, it is found.
\begin{align}
t_x &= \frac{b}{2} \notag \\
t_x &= \frac{L}{2} \label{eqn:t_x_L_2}
\end{align}

Likewise, the midpoint of hypotenuse $c$ is at the origin, so the starting position on the z-axis will be the midpoint of $a$.
Therefore, the travel distance $t_z$ will be half the length of $a$:
Substituting $a = L$ from equation \ref{eqn:a_b_L}, it is found.
\begin{align}
t_z &= \frac{a}{2} \notag \\
t_z &= \frac{L}{2} \label{eqn:t_z_L_2}
\end{align}

From equations \ref{eqn:t_x_L_2} and \ref{eqn:t_z_L_2} it can be seen that $t_x$ and $t_z$ are equal to $L/2$, and are therefore equal to each other. 
We can write:
\begin{align}
t_x = t_z &= \frac{L}{2} \label{eqn:t_x_z_L_2} \\
t_x = t_z &= \left(\frac{1}{2}\right)L \notag
\end{align}
\begin{equation}
     \boxed{
          t_x = t_z = 0.5 \cdot L \label{eqn:t_x_z_05_L}
     }
\end{equation}

Substituting $L$ from equation \ref{eqn:L_c_sqrt2} into equation \ref{eqn:t_x_z_L_2} yields:
\begin{align*}
t_x = t_z &= \frac{L}{2} \notag \\
t_x = t_z &= \frac{\frac{c}{\sqrt{2}}}{2} \notag \\
t_x = t_z &= \frac{c}{2\sqrt{2}} \notag \\
t_x = t_z &= \left(\frac{1}{2\sqrt{2}}\right) c \notag \\
\end{align*}
\begin{equation}
     \boxed{
          t_x = t_z = 0.353553 \cdot c  \label{eqn:t_x_z_035_c}
     }
\end{equation}

Furthermore, substituting $L$ from equation \ref{eqn:L_d_sqrt2} into equation \ref{eqn:t_x_z_L_2} yields:
\begin{align*}
t_x = t_z &= \frac{L}{2} \\
t_x = t_z &= \frac{d\sqrt{2}}{2} \\
t_x = t_z &= \left(\frac{\sqrt{2}}{2}\right)d
\end{align*}
\begin{equation}
     \boxed{
          t_x = t_z = 0.707107 \cdot d \label{eqn:t_x_z_07_d}
     }
\end{equation}

Symmetry of the initial setup can be used to create a symmetrical v-grove in the work by passing the rotating tool along the y-axis.

After each pass, the tool should be stepped slightly leftward until the tool has traveled a distance $t_x$ along the y-axis.
Then the tool should be advanced downward after each pass, along the z-axis, until distance $t_z$ has been traveled. 
Thus the groove is formed.

The machinist should be advised that the diameter $D$ of the cutting tool is relatively unimportant.
However if $D \leq 2 \cdot t_x$ then some leftover material may need to be cleaned up before the start point. 
