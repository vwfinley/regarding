\section{Solution}
As given, the right triangle shown in Figure \ref{fig:fig_b} has equal sides $a$ and $b$ of some length $L$
\begin{equation}
    \label{eqn:a_b_L}
    a = b = L
\end{equation}

Since lengths of $a$ and $b$ are equal, $\triangle{abc}$ is an isosceles triangle.
Furthermore, angles $\theta$ will be: 
\begin{equation}
\theta = \frac{180\degree - 90\degree}{2} = 45\degree
\end{equation}

Side $c$ is related to $a$ and $b$ by the Pythagorean theorem.
Substituting in equation \ref{eqn:a_b_L}, $L$ is found to be.
\begin{align}
c^2 &= a^2 + b^2 \\
c^2 &= L^2 + L^2 \notag \\
c^2 &= 2 L^2 \notag \\
\sqrt{c^2}   &= \sqrt{2 L^2} \notag \\
c   &= L\sqrt{2} \notag \\
L   &= \frac{c}{\sqrt{2}}
\end{align}

Cut depth $d$ is found by applying Pythagorean to the right triangle formed by $d$, half of $c$, and with $b$ as the hypotenuse. 



\begin{align}
     d &= \frac{c}{2} \notag \\ 
     c &= 2 d
\end{align}


\begin{align}
     L &= \frac{c}{\sqrt{2}} \notag \\ 
     L &= \frac{2d}{\sqrt{2}} \notag \\
     L &= \frac{d(\sqrt{2})^2}{\sqrt{2}} \notag \\
     L &= d \sqrt{2}
\end{align}


Since the midpoint of hypotenuse $c$ is at the origin, the starting position on the x-axis will be the midpoint of $b$.
Therefore, the travel distance $t_x$ will be half the length of $b$.
Substituting $b = L$ from equation \ref{eqn:a_b_L}, it is found.
\begin{align}
t_x &= \frac{b}{2} \notag \\
t_x &= \frac{L}{2}
\end{align}

The midpoint of hypotenuse $c$ is at the origin, so the starting position on the z-axis will be the midpoint of $a$.
Therefore, the travel distance $t_z$ will be half the length of $a$:
Substituting $a = L$ from equation \ref{eqn:a_b_L}, it is found.
\begin{align}
t_z &= \frac{a}{2} \notag \\
t_z &= \frac{L}{2}
\end{align}

\begin{align*}
t_x = t_z &= \frac{L}{2} \notag \\
t_x = t_z &= \left(\frac{1}{2}\right)L\notag \\
\end{align*}
\begin{equation}
     \boxed{
          t_x = t_z = 0.5 \cdot L
     }
\end{equation}


\begin{align*}
t_x = t_z &= \frac{L}{2} \notag \\
t_x = t_z &= \frac{\frac{c}{\sqrt{2}}}{2} \notag \\
t_x = t_z &= \frac{c}{2\sqrt{2}} \notag \\
t_x = t_z &= \left(\frac{1}{2\sqrt{2}}\right) c \notag \\
\end{align*}
\begin{equation}
     \boxed{
          t_x = t_z = 0.353553 \cdot c
     }
\end{equation}


\begin{align*}
t_x = t_z &= \frac{L}{2} \\
t_x = t_z &= \frac{d\sqrt{2}}{2} \\
t_x = t_z &= \left(\frac{\sqrt{2}}{2}\right)d
\end{align*}
\begin{equation}
     \boxed{
          t_x = t_z = 0.707107 \cdot d
     }
\end{equation}

Symmetry of the initial setup can be used to create a symmetrical v-grove in the work.

The diameter $D$ of the cutting tool is relatively unimportant since removal of material from the v-grove is accomplished by the travel of the cutting tool.
By gradually advancing the cutting tool downward a distance $L/2$ along the z-axis, and shifting leftward a distance $L/2$ along the y-axis, the groove is formed.

Follow these steps to create a symmetrical v-grove in the work.  

\subsection{Method 1}
This method starts cutting on the surface of the v-groove centerline.
It is easy to setup.
An advanage to this method is the machinst can choose to increase the value of $L$ while machining, without disturbing the zero position of Digital Readouts or dial indicators.

\begin{enumerate}
\item Determine the diameter $D$ of the cutting tool.
\item Mount the cutting tool in the milling machine.
\item Choose the desired length $L$ for the depth of side $a$ and width of side $b$, such that equation \ref{eqn:a_b_L} is satisfied.
\item Calculate $L/2$, this will determine the travel of the cutting tool.
\item Locate and scribe the place on surface of the work where the v-grove will be centered.
\item Mount the work at a $\theta = 45\degree$ angle in a vise in the milling machine, with the scribed line parallel to the y-axis.
\item Position the milling machine table so that the scribed line touches the bottom-left corner of the cutting tool.
\item Zero out any Digital Readouts or dial indicators.
\item Turn on the milling machine, make an initial pass along the y-axis.
\item On subsequent passes, advance the cutter leftward in the x-axis until it has traveled a maximum of $L/2$ to the left.
\item Advance the depth of cutter downward along the z-axis until it has sunk a maximum of $L/2$.
\end{enumerate}

\subsection{Method 2}
This method takes a little longer to setup, but it will be easier to machine.
For a cutting tool where $D > L$, the machinist only needs to advance the cut depth along the z-axis until the maximum $L$ depth (not $L/2$) is reached.
For a cutting tool where $D \leq L$, the machinist will need to make extra passes to remove missed material.

\begin{enumerate}
\item Determine the diameter $D$ of the cutting tool.
\item Mount the cutting tool in the milling machine.
\item Choose the desired length $L$ for the depth of side $a$ and width of side $b$.
\item Calculate $L/2$, this will determine the SETUP travels of the cutting tool.
\item Locate and scribe the place on surface of the work where the v-grove will be centered.
\item Mount the work at a $\theta = 45\degree$ angle in a vise in the milling machine, with the scribed line parallel to the y-axis.
\item Position the milling machine table so that the scribed line touches the bottom-left corner of the cutting tool.
\item Zero out any Digital Readouts or dial indicators.
\item Now raise the cutting tool a distance $L/2$ along the z-axis.
\item Move the cutting tool leftward a distance $L/2$ along the x-axis.
\item Again, zero out any Digital Readouts or dial indicators.
\item Turn on the milling machine, make an initial pass along the y-axis.
\item On subsequent passes, advance the depth of cutter downward along the z-axis until it has sunk a maximum of $L$ (not $L/2$ here)!!!
\end{enumerate}