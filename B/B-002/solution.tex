\section{Solution}
As given, the right triangle shown in Figure \ref{fig:fig_b} has sides $a = b = L$.
Therefore $\triangle{abc}$ is an isosceles triangle.

Furthermore, angles $\theta$ are 
\begin{equation}
\theta = \frac{180\degree - 90\degree}{2} = 45\degree
\end{equation}
This is expected since the mounting angle of the work relative to the cutting tool was given as $\theta = 45\degree$.

Symmetry of the initial setup can be used to create a symmetrical v-grove in the work.

The diameter $D$ is relatively unimportant since removal of material from the v-grove is accomplished by the travel of the cutting tool.
By gradually plunging the cutting tool downward a distance $L/2$ along the $z-axis$, and shifting leftward a distance $L/2$ along the $y-axis$, the groove is formed.

Ideally the machinist should use a cutting tool where $D > L$, but this is not strictly necessary.
This will avoid the need to return past the initial cutting pass.

Choosing $L \geq D$, or using a cutting tool with diameter $D \leq L$, is possible but will leave missed material before the starting cut.
The machinist will need to make extra passes to remove the missed material.

Follow these steps to create a symmetrical v-grove in the work.  

\subsection{Method 1}
This method starts cutting on the surface of the v-groove centerline.
It is easy to setup.

\begin{enumerate}
\item Determine the diameter $D$ of the cutting tool.
\item Mount the cutting tool in the milling machine.
\item Choose the desired length $L$ for the depth of side $a$ and width of side $b$.
\item Calculate $L/2$, this will determine the travel of the cutting tool.
\item Locate and scribe the place on surface of the work where the v-grove will be centered.
\item Mount the work at a $\theta = 45\degree$ angle in a vise in the milling machine, with the scribed line parallel to the y-axis.
\item Position the milling machine table so that the scribed line touches the bottom-left corner of the cutting tool.
\item Zero out any Digital Readouts or dial indicators.
\item Turn on the milling machine, make an initial pass along the y-axis.
\item On subsequent passes, advance the cutter leftward in the x-axis until it has traveled a maximum of $L/2$ to the left.
\item Advance the depth of cutter downward along the z-axis until it has sunk a maximum of $L/2$.
\end{enumerate}

\subsection{Method 2}
This method takes a little longer to setup, but it will be easier to machine.
If $D > L$, then it will only require the machinist to advance the cut depth along the z-axis until the maximum $L$ depth (not $L/2$) is reached.

\begin{enumerate}
\item Determine the diameter $D$ of the cutting tool.
\item Mount the cutting tool in the milling machine.
\item Choose the desired length $L$ for the depth of side $a$ and width of side $b$.
\item Calculate $L/2$, this will determine the SETUP travels of the cutting tool.
\item Locate and scribe the place on surface of the work where the v-grove will be centered.
\item Mount the work at a $\theta = 45\degree$ angle in a vise in the milling machine, with the scribed line parallel to the y-axis.
\item Position the milling machine table so that the scribed line touches the bottom-left corner of the cutting tool.
\item Zero out any Digital Readouts or dial indicators.
\item Now raise the cutting tool a distance $L/2$ along the z-axis.
\item Move the cutting tool leftward a distance $L/2$ along the x-axis.
\item Again, zero out any Digital Readouts or dial indicators.
\item Turn on the milling machine, make an initial pass along the y-axis.
\item On subsequent passes, advance the depth of cutter downward along the z-axis until it has sunk a maximum of $L$ (not $L/2$ here)!!!
\end{enumerate}