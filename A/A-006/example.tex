\section{Example}

\paragraph{Given:}
\begin{itemize}
\item A tangent ogive whose radius is 4.5 units and height is 2.3 units.
\end{itemize}

\paragraph{Find:}
\begin{itemize}
\item Values for $a$ and $b$.
\item Locations of points $P$ through $P_c$.
\item Locations of points $P'$ through $P'_c$ for the right half.
\item Locations of points $P''$ through $P''_c$ for the left half.
\item Distances of $\overline{P'_cP'}$ and $\overline{P'_cP''}$.
\end{itemize}

\paragraph{Steps:}
\begin{enumerate}
\item
Let: 
\begin{align*}
	r &= 4.5 \\
	c &= 2.3
\end{align*}

\item
From Equation \ref{eq:eqB5}, line-segment $b$ is found by substituting values for $r$ and $c$:
\begin{align*}
	b &= r - \sqrt{r^2 - c^2}\\
	b &= 4.5 - \sqrt{4.5^2 - 2.3^2}\\
	b &= 4.5 - \sqrt{20.25 - 5.29}\\
	b &= 4.5 - \sqrt{14.96}\\
	b &= 4.5 - 3.8678159212
\end{align*}
\begin{equation*}
\boxed{
	b = 0.6321840788
}
\end{equation*}

\item
The value for $a$ is easily found by rearranging Equaton \ref{eq:eqR1}, solving for $a$, and substituting values for $r$ and $b$.
\begin{align*}
	r &= a+b\\
	a &= r-b\\
	a &= 4.5-0.6321840788
\end{align*}
\begin{equation*}
\boxed{
	a = 3.8678159212
}
\end{equation*}

\item
Points $P$ through $P_c$ are found by substituting values of $a$, $b$, $c$ and $r$ into 
Equations \ref{eq:P} through \ref{eq:P_c}.
\begin{align*}
	P &= (0, 0)\\
	P_a &= (3.8678159212, 0)\\
	P_b &= (4.5, 0)\\
	P_c &= (3.8678159212, 2.3)
\end{align*}

\item
Points $P'$ through $P'_c$ are found by substituting values of $a$, $b$, $c$ and $r$ into 
Equations \ref{eq:PPrime} through \ref{eq:PPrime_c}.
\begin{align*}
	P' &= (-3.8678159212, 0)\\
	P'_a &= (0, 0)\\
	P'_b &= (0.6321840788, 0)\\
	P'_c &= (0, 2.3)
\end{align*}

\item
Points $P''$ through $P''_c$ are found by substituting values of $a$, $b$, $c$ and $r$ into 
Equations \ref{eq:PPrimePrime} through \ref{eq:PPrimePrime_c}.
\begin{align*}
P'' &= (3.8678159212, 0)\\
P''_a &= (0, 0)\\
P''_b &= (-0.6321840788, 0)\\
P''_c &= (0, 2.3)
\end{align*}

\item
Use the method of Section \ref{check} to check the correctness of the calculations.

The distance of $\overline{P'_cP'}$ is just the distance between point $P'_c$ and point $P'$.

From above:
\begin{align*}
	P'_c &= (0, 2.3)\\
	P' &= (-3.8678159212, 0)
\end{align*}
By using the Pythagorean theorem, the distance is found to be: 
\begin{align*}
	d &= \sqrt{(x_2-x_1)^2 + (y_2-y_1)^2}\\
	\overline{P'_cP'} &= \sqrt{(P'_x-P'_{cx})^2 + (P'_y-P'_{cy})^2}\\
	\overline{P'_cP'} &= \sqrt{(-3.8678159212 - 0)^2 + (0 - 2.3)^2}\\
	\overline{P'_cP'} &= \sqrt{14.96 + 5.29}\\
	\overline{P'_cP'} &= \sqrt{20.25}\\
	\overline{P'_cP'} &= 4.5
\end{align*}
As expected, the distance $\overline{P'_cP'}$ is equal to $r$ thereby confirming correctness of calculations.

Furthermore, absolute values $|P'_x| = |P''_x|$, and $P'_x$ or $P''_x$
are squared when either is substituted.
\begin{equation*}
	\overline{P'_cP'} = \overline{P'_cP''} = r = 4.5
\end{equation*}

\item
Use Equation \ref{eq:eqR1} as another check.
\begin{align*}
	r &= a + b\\
	r &= 3.8678159212 + 0.6321840788\\
	r &= 4.5
\end{align*}
Since all restrictions of Section \ref{restrictions} are satisfied, correctness of the calculations is confirmed.
\end{enumerate}