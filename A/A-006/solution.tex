\section{Solution}

\subsection{(Stage 1) Inscribing ogive section on semi-circle} 
Refer to Figure \ref{fig:fig_aa} (Stage 1).
Let a radius $r$ sweep out a semi-circle centered about a point $P$ which coincides with the origin.

The semi-circle intersects the x-axis at a point $P_b$ located on the x-axis at distance $r$ away from point $P$.

Let $c$ be the \textbf{inscribed} vertical line which intersects the semi-circle at some point $P_c$, and intersects the x-axis, at right angles, at a point $P_a$.  Point $P_a$ lies on the x-axis some distance $a$ away from $P$.  Finally, let $b$ be the line-segment between points $P_a$ and $P_b$.

Select the area bounded by sides $b$, $c$, and the arc between $P_b$ and $P_c$.  This area will be used in subsequent stages of the construction. 

Distances between points can be listed as:
\begin{align*}
\overline{PP_a} &= a \\
\overline{P_aP_b} &= b \\
\overline{P_aP_c} &= c \\
\overline{PP_b} &= r
\end{align*}

Furthermore, locations of the points can also be listed:
\begin{subequations}\label{eq:Points}
\begin{align}
P &= (0, 0)\label{eq:P}\\
P_a &= (a, 0)\label{eq:P_a}\\
P_b &= (r, 0)\label{eq:P_b}\\
P_c &= (a, c)\label{eq:P_c}
\end{align}
\end{subequations}
%\Stagecounter{equation}
Because points $P$, $P_a$ and $P_b$ are collinear, the distance $\overline{PP_b}$ is just the sum of $\overline{PP_a}$
and $\overline{P_aP_b}$.
\begin{equation}
\boxed{
	r = a+b \label{eq:eqR1}
}
\end{equation}

Now since points $P$, $P_a$ and $P_c$ form a right triangle, the Pythagorean theorem can be used to relate hypotenuse $r$ with sides $a$ and $c$.
\begin{equation}
\boxed{
	r = \sqrt{a^2+c^2} \label{eq:eqR2}
}
\end{equation}

To find length of side $a$ given lengths of $b$ and $c$, first equate variable $r$ in equations \ref{eq:eqR1} 
and \ref{eq:eqR2}, then reduce.
\begin{align*}
	a+b &= \sqrt{a^2+c^2} \\
	(a+b)^2 &= a^2+c^2 \\
	a^2 + 2ab + b^2 &= a^2 + c^2 \\
	2ab + b^2 &= c^2 \\
	2ab &= c^2 - b^2
\end{align*}
\begin{equation}
\boxed{
	a = \frac{c^2 - b^2}{2b} \label{eq:eqA3}
}
\end{equation}

\begin{figure}[ht!]
\centering
\begin{tikzpicture}[x=1in,y=1in]

	\coordinate (origin) at (0,0);
	\coordinate (I) at (0,0);
	\coordinate (t1) at (30 : 0.8660254038);
	\coordinate (t2) at (-30 : 0.8660254038);
	\coordinate (t1n) at (210 : 0.8660254038);
	\coordinate (t2n) at (150 : 0.8660254038);
	\coordinate (t1s) at (30:2);
	\coordinate (t2s) at (-30:2);

%	\coordinate (Origin0) at (0, -0*0.6);
%	\coordinate (Origin1) at (0, -1*1.6);

	\draw[lightgray, -{Latex[scale=1.5]}] (0, -6.0) -- (0, 1.5)  node [above] {$y$};





		\begin{scope} [shift={(-0.34202, 0)}]
			\begin{scope} [shift={(0.34202, 0)}]	
%				\node [anchor=west] at (-75:0.4) {Stage 3: Reflect};
			\end{scope}
		
			\begin{scope} [shift={(0.34202, 0)}]	
				\draw[lightgray, -{Latex[scale=1.5]}] (-1.5,0) -- (1.5,0)  node [right] {$x$};		
			\end{scope}
			
%			\draw[lightgray] (0:1) arc [start angle=0, delta angle=360, radius=1]; % full-circle
%			\draw[lightgray] (0.34202,0) circle [radius=1]; % full-circle
			\draw[lightgray] (0,0) circle [radius=1]; % full-circle
			
			\draw[fill=lightgray!50] (0.34202,0) -- (0:1) arc [start angle=0, delta angle=70, radius=1] -- (0.34202,0.0); % wedge-region

			\begin{scope} [shift={(0.68404, 0)}]		
			\begin{scope}[xscale=-1,yscale=1]
				\draw[lightgray] (0,0) circle [radius=1]; % full-circle
				\draw[fill=lightgray!15] (0.34202,0) -- (0:1) arc [start angle=0, delta angle=70, radius=1] -- (0.34202,0.0); % wedge-region
			\end{scope}
			\end{scope}

			\begin{scope} [shift={(0, 0)}]
				\draw  node [shift={(-0.07, -0.13)}]{$P'$};
			\end{scope}
			
			\begin{scope} [shift={(0.68404, 0)}]
				\draw  node [shift={(-0.0, -0.13)}]{$P''$};
			\end{scope}

			\begin{scope} [shift={(1, 0 )}]
				\draw [fill=black] circle (0.02) node [shift={(0.0, -0.15)}]{$P'_b$};
			\end{scope}

			\begin{scope} [shift={(70:1.0)}]
				\draw [fill=black] circle (0.02) node [shift={(0.05, 0.11)}]{$P'_c$};
			\end{scope}

			\begin{scope} [shift={(0.68404, 0)}]
			\begin{scope}[xscale=-1,yscale=1]		
				\draw[lightgray, -{Latex[scale=1.0]}] (0,0) -- (45:1);% node [shift={(0.2, -0.4)}]{$r$}; % radius and label r
			\end{scope}
			\end{scope}

			\begin{scope} [shift={(0, 0)}]
				\draw[black, -{Latex[scale=1.0]}] (0,0) -- (45:1) node [shift={(-0.49, -0.4)}]{$r$}; % radius and label r
			\end{scope}

%			\draw node [shift={(0.15, 0.056)}]{$a$}; %label a
%			\draw node [shift={(0.67, 0.073)}]{$b$}; %label b
%			\draw node [shift={(0.41, 0.35)}]{$c$}; %label c

			\begin{scope} [shift={(0, 0)}]
				\draw (-0.025,0) -- (0.025, 0); % centermark
				\draw (0, -0.025) -- (0, 0.025);
			\end{scope}

			\begin{scope} [shift={(0.68404, 0)}]
				\draw (-0.025,0) -- (0.025, 0); % centermark
				\draw (0, -0.025) -- (0, 0.025);
			\end{scope}

			\begin{scope} [shift={(0.34202, 0 )}]
%				\draw (90: 0.1) -- ++(0: -0.1) -- ++(-90: 0.1) ; % square
				\draw [fill=black] circle (0.02) node [shift={(0.0, -0.15)}]{$P'_a$};
			\end{scope}
		\end{scope}



\end{tikzpicture}
\caption{Construction of Tangent Ogive}
\label{fig:fig_aa}
\end{figure}

Next, from equation \ref{eq:eqR2}, it is convenient to express $r^2$ in terms of $a^2$ and $c^2$ 
\begin{align*}
	r &= \sqrt{a^2+c^2} \\
	r^2 &= a^2+c^2
\end{align*}
Then to substitute $a = r - b$ from equation \ref{eq:eqR1}. 
\begin{equation}
	r^2 = (r - b)^2+c^2 \label{eq:eqRSQ4}
\end{equation}

Using equation \ref{eq:eqRSQ4}, $b$ can be found in terms of $r$ and $c$.
\begin{align*}
	r^2 &= (r - b)^2+c^2 \\
	(r - b)^2 &= r^2 - c^2 \\
	\sqrt{(r - b)^2} &= \sqrt{r^2 - c^2} \\
	r - b &= \sqrt{r^2 - c^2}
\end{align*}
\begin{equation}
\boxed{
	b = r - \sqrt{r^2 - c^2} \label{eq:eqB5}
}
\end{equation}

Next, by expanding equation \ref{eq:eqRSQ4} and simplifying, $c$ can be found in terms of $r$ and $b$.
\begin{align*}
	r^2 &= (r - b)^2+c^2 \\
	r^2 &= r^2 - 2rb +  b^2 + c^2 \\
	c^2 &= 2rb - b^2
\end{align*}
\begin{equation}
\boxed{
	c = \sqrt{2rb - b^2}  \label{eq:eqC6}
}
\end{equation}	

\subsection{(Stage 2) Translate ogive half-section to y-axis} 
The ogive section is easily translated to the origin and y-axis by subtracting length $a$ from the $x$-coordinates in points $P$ through $P_c$ from equations \ref{eq:P}
through \ref{eq:P_c}, as follows:
\begin{subequations}\label{eq:PointsPrime}
\begin{align}
P' &= (-a, 0)\label{eq:PPrime}\\
P'_a &= (0, 0)\label{eq:PPrime_a}\\
P'_b &= (b, 0)\label{eq:PPrime_b}\\
P'_c &= (0, c)\label{eq:PPrime_c}
\end{align}
\end{subequations}
%\Stagecounter{equation}
Prime notation is used here to show how the new points are derived from the original points. 

\subsection{(Stage 3) Reflect ogive half-section about y-axis}
Reflecting the right ogive half-section across the y-axis is easily accomplished by applying a negative to all $x$-coordinates in points $P'$ through $P'_c$ from equations \ref{eq:PPrime} through \ref{eq:PPrime_c}, as follows:
\begin{subequations}\label{eq:PointsPrimePrime}
\begin{align}
P'' &= (a, 0)\label{eq:PPrimePrime}\\
P''_a &= (0, 0)\label{eq:PPrimePrime_a}\\
P''_b &= (-b, 0)\label{eq:PPrimePrime_b}\\
P''_c &= (0, c)\label{eq:PPrimePrime_c}
\end{align}
\end{subequations}
%\Stagecounter{equation}
Double-prime notation is used here to show how the new points are derived from the original points. 
Reflecting the half-section about y-axis yields a full ogive cross-section.

Note: points $P''_a$, $P''_b$ and $P''_c$ are omitted for clarity in Figure \ref{fig:fig_aa} (Stage 3).  However point $P''$ is shown.

\subsection{(Stage 4) Check ogive cross-section for accuracy}\label{check}
Once the ogive cross-section has been calculated and constructed, checking for correctness and accuracy is recommended.

Checking the completed cross-section is a simple matter of recognizing the relationship between: $P'_c$, $P'$ and $P''$.  Refer to Figure \ref{fig:fig_aa} (Stage 3).  The lines centered at either $P'$ or $P''$, drawn to point $P'_c$, have length $r$ and sweep arcs of radius $r$.  The arcs meet at a common point $P'_c$. 

Refer to Figure \ref{fig:fig_aa} (Stage 4). Lines centered at $P'_c$ drawn back, in the opposite direction, to either $P'$ or $P''$ will also have length $r$.  So, an arc of
radius $r$, centered at point $P'_c$, will include $P'$ and $P''$.

Therefore, a quick algebraic check for correctness is to measure the distance of lines $\overline{P'_cP'}$ and $\overline{P'_cP''}$.  Both should equal $r$.

A corresponding geometric check is to sweep an arc of radius $r$ centered at $P'_c$.  If points $P'$ and $P''$ lie on the arc, then the ogive cross-section is correctly constructed.

\subsection{Restrictions and valid values}\label{restrictions}
The three most important variables in equations \ref{eq:eqR1} through \ref{eq:eqC6} are: $a$, $b$ and $r$.  They serve as inputs to the ogive geometry and
are subject to some restrictions.  These equations are undefined for values outside the valid ranges.

Restrictions:
\begin{itemize}
\item $r > 0$ : Otherwise the ogive is undefined.
\item $0 < a < r$ : Otherwise the ogive is undefined
\item $0 < b < r$ : Otherwise the ogive is undefined
\item $0 < c < r$ : Otherwise the ogive is undefined
\item Equation \ref{eq:eqR1} must be satisfied.  Otherwise, the ogive is undefined. 
\end{itemize}

\begin{flushright}   
Q.E.D.
\end{flushright}   


