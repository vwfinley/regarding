%% List of Steps
\section{Solution}
The strategy here is to adapt the circle-pair cases from prior article\footnotemark[\ref{noteA003}] for the condition of equal circles.  Next, the effect of equal circles upon tangent classes is examined.
Then, how $I$ tangent intersection point is affected.  Finally, classification for equal circles is performed.

\subsection{Effect upon Circle-Pair Cases\\} 
Refer to Figure \ref{fig:fig_b} for the discussion that follows.

For each of the cases defined in prior work\footnotemark[\ref{noteA003}] set the radius of circle $B$ to be equal to the radius of circle $A$, such that
$r_B = r_A$.  For each equation or inequality substitute $r_A$ in place of all occurrences of $r_B$.  Combine terms and simplify the equation or inequality.  Each of the ten cases follow:

\begin{figure}[ht!]
\centering

\begin{tikzpicture}[x=1in, y=1in]
	\begin{scope}[xscale=.10, yscale=.10]	

		\clip (-24, -10) rectangle (7, 17);

		% axes
		\draw[lightgray, -{Latex[scale=1.5]}] (-40.0, 0) -- (5.0, 0)  node [right] {$x$};		
		\draw[lightgray, -{Latex[scale=1.5]}] (0, -30.0) -- (0, 15.0)  node [above] {$y$};	

		% vertical
		\draw[lightgray] (-16.942, -35) -- (-16.942, 35); % left

		% horizontal
		\draw[lightgray] (-50, 13.0) -- (90, 13.0) node[shift={(179.4 : 8.5)}]{$L_h$};    % node[shift={(179.3 : 8.2)}]{$L_h$}; % L_H
%		\draw[lightgray] (-19.0097, -4.0024) -- (-9.0097, -4.0024);	 % A center mark

%		\draw[lightgray] (-16.0097, -0.75) -- ++(0  : 0.75) -- ++(90 : 0.75) ;	 % A center mark

		% sloped
		\draw[lightgray] (195 : 100) -- (15 : 100) node[shift={(-165.6 : 9.45)}]{$L_s$}; % L_S

		% circle
		\draw[lightgray] ([shift=(135 :  21.355)] 0, 0)  arc (135 : 150: 21.355); % mark
		
		% bisector
		\draw[lightgray, dashed] (142.50 : -3) -- (142.50 : 25); % dashed line

		% angle labels
%		\draw[black] ([shift=(180 : 5)] 0, 0)  arc (180 : 194.04 : 5) node[shift={(-0.10, 0.05)}]{$\alpha$} ; % alpha
%		\draw[black] ([shift=(194.04 : 2)] 0, 0)  arc (194.04 : 270 : 2) node[shift={(-0.17, -0.05)}]{$\beta$}; % beta
%		\draw[black] ([shift=(142.02 : 4)] 0, 0)  arc (142.02 : 194.04 : 4) node[shift={(-0.05, 0.23)}]{$\gamma$}; % gamma
%		\draw[black] ([shift=(194.04 : 3)] 0, 0)  arc (194.04 : 90 : 3) node[shift={(-0.21, 0.003)}]{$\delta$}; % delta

%		\draw[black] ([shift=(0 : 5)] 194.04 : 16.5024)  arc (0 : 14.04 : 5) node[shift={(0.09, -0.05)}]{$\alpha$}; % alpha
%		\draw[black] ([shift=(14.04 : 2)] 194.04 : 16.5024)  arc (14.04 : 90 : 2) node[shift={(0.17, 0.05)}]{$\beta$}; % beta		
%		\draw[black] ([shift=(142.02 : 20.3116)] 270 : 4)  arc (270 : 322.02 : 4) node[shift={(-0.09, -0.19)}]{$\gamma$}; % gamma

		% dimensions
%		\draw[lightgray, latex - latex] (0 : -22.5) -- +(90 : 12.5) node[shift={(-84 : 0.65)}]{$h$};
%		\draw[lightgray, latex - latex] (0 : -22.5) -- +(-90 : 4.0024) node[shift={(75 : 0.24)}]{$i$};
%		\draw[lightgray] (-23.5, -4.0024) -- (-21.5, -4.0024);	 % A center mark

		% locate arc
		\draw[black, thick] (-16.942, -4.54) circle (17.54); % The arc
		
		% radius lines
%		\draw[black, -latex] (194.04 : 16.5024) -- +(90 : 16.5024) node[shift={(-94 : 0.9)}]{$r$};
%		\draw[black, -latex] (194.04 : 16.5024) -- +(14.04 : 16.5024) node[shift={(200 : 0.75)}]{$r$};




		% point labels
		\draw[black] (192 : 18.2)  node{$A$};
		\draw[black] (50 : 1.4)  node{$B$};
		\draw[black] (140.2 : 23)  node{$C$};
%		\draw[black] (183 : 16.8)  node{$D$};
		\draw[black] (85 : 12.1)  node{$E$};

	\end{scope}
		





\end{tikzpicture}
\caption{Locations of $C_A$ relative to $C_B$}
\label{fig:fig_b}
\end{figure}

\begin{itemize}
\item \textbf{case 0:}
$x_A = 0$

Identical to case 4.  When the center of $A$ is at the origin, $A$ and $B$ are the \emph{same} circle.

\item \textbf{case 1:}
$0 < x_A < r_A$

Identical to case 5.  
When $C_A$ lies less than one $r_A$ radius from the origin, $A$ will overlap left and right regions of $B$.

\item \textbf{case 2:}
$ x_A = r_A$

Identical to case 6.  
When $C_A$ lies exactly than one $r_A$ radius from the origin, circle $A$ will pass through center of  $C_B$ at the origin.
Furthermore, $C_A$ will lie on $B$.

\item \textbf{case 3:}
This case is UNDEFINED.

Taking case 3 inequality from prior article\footnotemark[\ref{noteA003}], substituting $r_A$ in place of $r_B$, and reducing yields:
\begin{align*}
r_A &< x_A < r_B - r_A\\
r_A &< x_A < r_A - r_A\\
r_A &< x_A < 0
\end{align*}
$x_A$ cannot be less than zero (negative), and simultaneously greater than non-negative values of $r_A$.  Therefore the inequality is not satisfied, and case 3
is undefined.

\item \textbf{case 4:} $ x_A = 0$

Taking case 4 equation from prior article\footnotemark[\ref{noteA003}], substituting $r_A$ in place of $r_B$, and reducing yields:
\begin{align*}
 x_A &= r_B - r_A\\
 x_A &= r_A - r_A\\
 x_A &= 0
\end{align*}
Since the center of $A$ is at the origin, $A$ and $B$ are the \emph{same} circle. This is identical to case 0 above.

\item \textbf{case 5:}
$ 0 < x_A < r_A$

Taking case 5 inequality from prior article\footnotemark[\ref{noteA003}], substituting $r_A$ in place of $r_B$, and reducing yields:
\begin{align*}
r_B - r_A &< x_A < r_B\\
r_A - r_A &< x_A < r_A\\
0 &< x_A < r_A
\end{align*}
This is identical to case 1 above.

\item \textbf{case 6:} 
$ x_A = r_A$

Taking case 6 equation from prior article\footnotemark[\ref{noteA003}], substituting $r_A$ in place of $r_B$, and reducing yields:
\begin{align*}
x_A &= r_B\\
x_A &= r_A
\end{align*}
This is identical to case 2 above.

\item \textbf{case 7:}
$ r_A < x_A <2 \cdot r_A$

Taking case 7 inequality from prior article\footnotemark[\ref{noteA003}], substituting $r_A$ in place of $r_B$, and reducing yields:
\begin{align*}
r_B &< x_A < r_B + r_A\\
r_A &< x_A < r_A + r_A\\
r_A &< x_A <2 \cdot r_A
\end{align*}
The center of $A$ is beyond the edge of circle $B$.  Circle $A$ partially overlaps the right side of $B$.

\item \textbf{case 8:}
$ x_A = 2  \cdot  r_A$

Taking case 8 equation from prior article\footnotemark[\ref{noteA003}], substituting $r_A$ in place of $r_B$, and reducing yields:
\begin{align*}
x_A &= r_B + r_A\\
x_A &= r_A + r_A\\
x_A &= 2 \cdot r_A
\end{align*}
Circles $A$ an $B$ are tangent to each other with no overlap.

\item \textbf{case 9:}
$ x_A > 2  \cdot r_A$

Taking case 9 inequality from prior article\footnotemark[\ref{noteA003}], substituting $r_A$ in place of $r_B$, and reducing yields:
\begin{align*}
x_A &> r_B + r_A\\
x_A &> r_A + r_A\\
x_A &> 2  \cdot r_A
\end{align*}
Circle $A$ is entirely outside $B$. $A$ and $B$ are non-overlapping.
\end{itemize}

\subsection{Effect upon Tangent Classes\\} 
For each tangent class defined in prior work\footnotemark[\ref{noteA004}] consider how tangents are affected by setting the radii of circles $A$ and $B$ to be equal.

\begin{figure}[hb!]
\centering
\begin{tikzpicture}[x=1in,y=1in]

	\coordinate (origin) at (0,0);
	\coordinate (I) at (0,0);
	\coordinate (t1) at (30 : 0.8660254038);
	\coordinate (t2) at (-30 : 0.8660254038);
	\coordinate (t1n) at (210 : 0.8660254038);
	\coordinate (t2n) at (150 : 0.8660254038);
	\coordinate (t1s) at (30:2);
	\coordinate (t2s) at (-30:2);
	\coordinate (C) at (0:1);

	\coordinate (B) at (0,0);
	
	\coordinate (A0) at (0*0.05,0);
 	\coordinate (A1) at (1*0.05,0);
	\coordinate (A2) at (2*0.05,0);
	\coordinate (A3) at (3*0.05,0);
	\coordinate (A4) at (4*0.05,0);

	\coordinate (A5) at (5*0.05,0);
	\coordinate (A6) at (6*0.05,0);
	\coordinate (A7) at (7*0.05,0);

	\coordinate (A8) at (8*0.05,0);
	\coordinate (A9) at (9*0.05,0);


	\coordinate (Origin0) at (0, -0*0.6);
	\coordinate (Origin1) at (0, -1*0.6);

%	\draw[-{Latex[scale=1.5]}] (0, 9.75 * -0.65) -- (0, 0.5)  node [above] {$y$};
	\draw[lightgray, -{Latex[scale=1.5]}] (0, 5.5 * -0.65) -- (0, 0.5)  node [above] {$y$};




	\foreach \i in {0,...,0}
	{
		\begin{scope} [shift={(0, \i * -0.65 )}]
			\draw[lightgray, -{Latex[scale=1.5]}] (-0.5,0) -- (1.3,0)  node [right] {$x$};	
			\draw node [shift={(0.65,-0.45)}, anchor=west] {Class A: Null};			
%			\draw[lightgray] circle (0.3) node [shift={(0.30, 0.25)}] {$B$};
			\draw node [shift={(0.45, 0.08)}] {$Undefined$};
			\begin{scope} [shift={(\i * 0.05, 0)}]
%				\draw[lightgray] circle (0.1) node [shift={(0.15, 0.10)}] {$A$};
%				\draw (-0.025,0) -- (0.025, 0); % centermark
%				\draw (0, -0.025) -- (0, 0.025);				
			\end{scope}
		\end{scope}
	}

%	\foreach \i in {4,7,9}
%	{
%		\begin{scope} [shift={(0, \i * -0.65 )}]
%			\draw[lightgray, -{Latex[scale=1.5]}] (-0.5,0) -- (0.75,0);	
%			\draw node [shift={(0.52,-0.24)}] {case \i};
%			\draw[lightgray] circle (0.3);
%			\begin{scope} [shift={(\i * 0.05, 0)}]
%				\draw circle (0.1);				
%				\draw (-0.025,0) -- (0.025, 0);
%				\draw (0, -0.025) -- (0, 0.025);
%			\end{scope}
%		\end{scope}
%	}
	declare function={ Nprime(x) = x; 
			}
		]
  	\foreach \i\j\k in {4/1/Class B: Single}
	{
			\begin{scope} [shift={(0, \j * -1.0 )}]
			\draw[lightgray, -{Latex[scale=1.5]}] (-0.5,0) -- (1.3,0); % x-axis
			\draw node [shift={(0.65,-0.45)}, anchor=west] {\k};  % label
			\draw[lightgray] circle (0.3) node [shift={(-0.2, 0.35)}] {$B$}; % circle B
			\begin{scope} [shift={(-4  * 0.15, 0)}]		
				\begin{scope} [shift={(\i * 0.15, 0)}]
					\draw[lightgray] circle (0.3) node [shift={(0.2, 0.35)}] {$A$}; % circle A				
					\draw (-0.025,0) -- (0.025, 0); % centermark
					\draw (0, -0.025) -- (0, 0.025);
				\end{scope}
			\end{scope}
		\end{scope}
	}
  	\foreach \i\j\k in {7/2/Class C: Outer,9/3/Class D: Inner}
	{
			\begin{scope} [shift={(0, \j * -1.0 )}]
			\draw[lightgray, -{Latex[scale=1.5]}] (-0.5,0) -- (1.3,0); % x-axis
			\draw node [shift={(0.65,-0.45)}, anchor=west] {\k};  % label
			\draw[lightgray] circle (0.3) node [shift={(-0.35, 0.2)}] {$B$}; % circle B
			\begin{scope} [shift={(-4  * 0.15, 0)}]		
				\begin{scope} [shift={(\i * 0.15, 0)}]
					\draw[lightgray] circle (0.3) node [shift={(0.35, 0.2)}] {$A$}; % circle A				
					\draw (-0.025,0) -- (0.025, 0); % centermark
					\draw (0, -0.025) -- (0, 0.025);
				\end{scope}
			\end{scope}
		\end{scope}
	}
	

	\begin{scope} [shift={(0, 1 * -1.0 )}]
		\begin{scope} [shift={(0.3, 0 )}]
			\draw (90.0 : 0.5) -- (-90.0 : 0.5); % common tangent
		\end{scope}		
	\end{scope}


	\begin{scope} [shift={(0, 2 * -1.0 )}]
		\begin{scope} [shift={(90 : 0.3 )}]
			\draw (180 : 0.4) -- (0 : 0.9); % outer tangent (I)
		\end{scope}		
		\begin{scope} [shift={(-90 : 0.3 )}]
			\draw (180 : 0.4) -- (0 : 0.9); % outer tangent (I)
		\end{scope}		
	\end{scope}



	\begin{scope} [shift={(0, 3 * -1.0 )}]	
		\begin{scope} [shift={(0.375, 0 )}]
			\draw (-53.13 : 0.45) -- (180.0 - 53.13 : 0.45); % outer tangent (I)
			\draw (53.13 : 0.45) -- (-180.0 + 53.13 : 0.45); % outer tangent (I)
		\end{scope}			
	\end{scope}


% centermark
%VWF_FIGA	
%	\begin{scope}  [shift={(C)}]	
%		\draw (-0.1, 0.0) -- (0.1, 0.0);
%		\draw (0.0, -0.1) -- (0.0, 0.1);
%	\end{scope}	
%
	% axes
% VWF_FIGB
%	\draw[-{Latex[scale=1.5]}] (-0.5,0) -- (0.75,0)  node [right] {$x$};
%	\draw[-{Latex[scale=1.5]}] (0,-0.5) -- (0,0.5)  node [above] {$y$};


%	\begin{scope} [shift={(B)}]
%		\draw circle (0.3);
%	\end{scope}

%	\begin{scope} [shift={(A0)}]
%		\draw[-{Latex[scale=1.5]}] (-0.5,0) -- (0.75,0)  node [right] {$x$};
%		\draw circle (0.1);
%	\end{scope}

%	\begin{scope} [shift={(A1)}]
%		\draw circle (0.1);
%	\end{scope}
	
%	\begin{scope} [shift={(A2)}]
%		\draw circle (0.1);
%	\end{scope}

%	\begin{scope} [shift={(A3)}]
%		\draw circle (0.1);
%	\end{scope}

%	\begin{scope} [shift={(A4)}]
%		\draw circle (0.1);
%	\end{scope}

%	\begin{scope} [shift={(A5)}]
%		\draw circle (0.1);
%	\end{scope}

%	\begin{scope} [shift={(A6)}]
%		\draw circle (0.1);
%	\end{scope}
	
%	\begin{scope} [shift={(A7)}]
%		\draw circle (0.1);
%	\end{scope}
	
%	\begin{scope} [shift={(A8)}]
%		\draw circle (0.1);
%	\end{scope}
	
%	\begin{scope} [shift={(A9)}]
%		\draw circle (0.1);
%	\end{scope}	
	

%	\draw[dotted] (t1n) -- (t1);
%	\draw[dotted] (t2n) -- (t2);

%	\draw (t1) -- (t1s);
%	\draw (t2) -- (t2s);
	
% VWF_FigA	
%	\draw (t1) arc[start angle=120, end angle=240, radius=0.5];
% VWF_FIGB
 %	\draw [dashed] (t1) arc[start angle=120, end angle=-120, radius=0.5];
%	\draw (t1) arc[start angle=120, end angle=-120, radius=0.5];
%	

% VWF_FigA	
%	 \draw[-{Latex}] (C) -- (t1);
%

% VWF_FIGB
	% \draw (C) ellipse (0.5);
%	 \draw (C) -- (t1);
%
	
	
% VWF_FIGB
%	\begin{scope}  [shift={(t1)}]
%		\draw (210:0.1) -- ++(-60:0.1); 
%		\draw (-60:0.1) -- ++(210:0.1); 
%	\end{scope}	
	

	% point I
%	\begin{scope} [shift={(I)}]
%		\draw [fill=black] circle (0.02);
%	\end{scope}

	% point C
% VWF_FIGB	
%	\begin{scope} [shift={(C)}]
%		\draw [fill=black] circle (0.02);
%	\end{scope}
	
	% point P
% VWF_FIGB
%	\begin{scope} [shift={(t1)}]
%		\draw [fill=black] circle (0.02);
%	\end{scope}

% VWF_FIGB
%	\begin{scope} [shift={(I)}]
%		\draw (0.25,0) arc[start angle=0, end angle=30, radius=0.25];
%	\end{scope}


%	\begin{scope} [shift={(I)}]
%		\draw (150:0.25) arc[start angle=150, end angle=210, radius=0.25];
%	\end{scope}
	
	
	%\node [label={Hello}] ;
%	\node [label={[shift={(0.1,-0.25)}]$C$}] at (C) {};
%	\node [label={[shift={(-0.2,-0.1)}]$L_1$}] at (t1s) {};
%	\node [label={[shift={(-0.2,0.1)}]$L_2$}] at (t2s) {};
%	\node [label={[shift={(-0.05,0.1)}]$r$}] at (C) {};
%	\node at (165:0.35) {$\theta$};
%	\node [label={[shift={(0.05,-0.28)}]$I$}] at (I) {};
	
% VWF_FIGB	
%	\node [label={[shift={(0.0,-0.04)}]$P$}] at (t1) {};
%	\node [label={[shift={(0.65,-0.25)}]$\Delta x$}] at (I) {};	
%	\node [label={[shift={(0.33,0.15)}]$d$}] at (I) {};	
%	\node at (15:0.35) {$\alpha$};
%


\end{tikzpicture}
\caption{Tangent Classes}
\label{fig:fig_c}
\end{figure}

\paragraph{Class A: Null\\}
The Null class covers those cases where circles $A$ and $B$ have no common tangent line, as seen in cases 0-3 in prior article\footnotemark[\ref{noteA003}].

Null tangent class arises when circle $A$ is wholly contained by circle $B$, and there is no contact between the circles.  This can only occur when $r_A < r_B$.

However, when radii of circles $A$ and $B$ are equal, there can be no case where circle $A$ is wholly contained by circle $B$.  In fact, even when $r_A = r_B$ and the 
centers $C_A = C_B$ coincide, circle $A$ will not be contained by circle $B$.  This is because for this condition $A$ and $B$ will actually be the \emph{same} circle.

When, $A$ and $B$ are the \emph{same} circle they will have an infinite number of common tangents.

Since $A$ can never be fully contained by $B$ when their radii are equal, the Null tangent class does not apply (is undefined) for $r_A = r_B$.  

\paragraph{Class B: Single\\} 
When there is a tangent at the point of contact between circles $A$ and $B$, the Single tangent class is satisfied.

Inspecting Figure \ref{fig:fig_b}, shows case 8 has a single tangent at the point where $A$ and $B$ contact.

Furthermore when $r_A = r_B$, cases 0 and 4 degenerate from case 4 in prior article\footnotemark[\ref{noteA003}].  The point of contact for case 4 in prior article\footnotemark[\ref{noteA003}] is $x = r_B$.  For equal circles, the point of contact
is also equivalent to $x = r_A$.  Therefore $A$ and $B$ will have contact at $x = r_A$ since they are the \emph{same} circle.

In cases 0 and 4, $A$ and $B$ have infinite contact points and tangent lines since they are the \emph{same} circle.  Moreover, for cases 0 and 4, circles $A$ and $B$ reduce to the trivial case of a single circle with tangent lines
at every point on the circle, essentially infinite.

\paragraph{Class C: Outer\\} 
Outer tangents only occur when the centers of $A$ and $B$ are separated.  When $r_A = r_B$ separation will occur for cases: 1, 2, 5, 6, 7, 8, 9.

See Figure \ref{fig:fig_c}.  Because $r_A = r_B$ are equal, the Outer tangents will always be parallel to the x-axis.  Furthermore, they will always be parallel to each other and never intersect.

This can be confirmed by attempting to calculate the Outer tangent intersection point $I$ from equations 3 and 4 in prior work\footnotemark[\ref{noteA004}].

\begin{align*}
x_I &= \frac{r_B \cdot x_A}{r_B - r_A}\\
x_I &= \frac{r_A \cdot x_A}{r_A - r_A}\\
x_I &= \frac{r_A \cdot x_A}{0}\\
x_I &= \infty
\end{align*}
Substituting $x_I$ into equation 4 from prior work\footnotemark[\ref{noteA004}] yields:
\begin{equation*}
I = (x_I, 0)
\end{equation*}
\begin{equation}
\boxed
{
I = (\infty, 0)
}
\end{equation}
Therefore the intersection point of the Outer tangent lines is at somewhere at infinity along the x-axis.  In other words, the Outer tangent lines are parallel because they will never intersect.

\paragraph{Class D: Inner\\} 
See Figure \ref{fig:fig_c}.  Inner tangents only occur for case 9.  This is due to the fact that inner tangents will intersect each other along the x-axis, while not contacting circles $A$ or $B$ more than once.
Case 9 is the only to satisfy this condition for Inner class tangents.

Since the x-axis connects centers of $A$ and $B$, the point of intersection $I$ will occur along the x-axis at the midpoint between centers $C_A$ and $C_B$.  This can be confirmed by applying equation 4
and 9 from prior article\footnotemark[\ref{noteA003}].

 \begin{align*}
	x_I &= \frac{r_B \cdot x_{A'}}{ r_B + r_A}\\
	x_I &= \frac{r_A \cdot x_{A'}}{ r_A + r_A}\\
	x_I &= \frac{r_A \cdot x_{A'}}{2 \cdot r_A}
\end{align*}

\begin{equation}
\boxed
{
	x_I = \frac{x_{A'}}{2} \label{eq:eqxI}
}
\end{equation}

Substituting equation \ref{eq:eqxI} here into equation 4 from prior work\footnotemark[\ref{noteA004}] yields:

\begin{equation}
\boxed
{
I = \left(\frac{x_{A'}}{2}, 0\right)  \label{eq:IResult}
}
\end{equation}
Since the center of circle $B$ is at the origin, it can be seen from equation \ref{eq:IResult} that $I$ is located at the midpoint between $A$ and $B$.

The angle $\theta$ between the inner tangents can be found by taking equation 5 from prior work\footnotemark[\ref{noteA004}], and substituting $r_A$ in place of $r_B$:
\begin{align}
	\theta &= 2 \cdot \arcsin{\left( \frac{r_B}{x_I}  \right)}\\
	\theta &= 2 \cdot \arcsin{\left( \frac{r_A}{x_I}  \right)} \label{eq:theta}
\end{align}

%%%%%%%
\begin{table*}[ht!]
\centering
\begin{tabular}{|c||c|c|c|c|c|c|} 
 \hline
  \multirow{3}{*}{Case} & \multicolumn{5}{c|}{Number of Tangents} & \multirow{3}{*}{Description} \\
\cline{2-6}
   & \makecell{(A)\\Null}  & \makecell{(B)\\Single} & \makecell{(C)\\Outer} & \makecell{(D)\\Inner} & Total & \\
  \hline
  0 (4) & - & 1 & - & - & 1 & Same circle\\ 
  \hline
  1 (5) & - & - & 2 & - & 2 & Parallel outer tangents\\ 
  \hline
  2 (6) & - & - & 2 & - & 2 & Parallel outer tangents\\ 
  \hline
  3 & - & - & - & - & x & Undefined case\\ 
  \hline
  4 (0) & - & 1 & - & - & 1 & Same circle\\ 
  \hline
  5 (1) & - & - & 2 & - & 2 & Parallel outer tangents\\ 
  \hline
  6 (2) & - & - & 2 & - & 2 & Parallel outer tangents\\ 
  \hline
  7 & - & - & 2 & - & 2 & Parallel outer tangents\\ 
  \hline
  8 & - & 1 & 2 & - & 3 & Circles tangent to each other, and parallel outer tangents\\ 
  \hline
  9 & - & - & 2 & 2 & 4 & Parallel outer tangents and two inner tangents\\ 
  \hline
\end{tabular}
\caption{Classification of Equal Coplanar Circle Cases}
\label{table:tab_tangents}
\end{table*}
%%%%%%%
Now, equation \ref{eq:eqxI} can be substituted into equation \ref{eq:theta} to find the angle between inner tangents.
\begin{equation}
\boxed
{
	\theta = 2 \cdot \arcsin{\left( \frac{2 \cdot r_A}{x_{A'}}  \right)} \label{eq:thetaarcsin}
}
\end{equation}

\subsection{Effect upon Classification\\} 
Increasing the radius of $A$ to be equal with radius of $B$ eliminates the possibility of circle $A$ being wholly contained by circle $B$.  As the center of $A$
moves along the positive x-axis from the origin, the rightmost edge of $A$ will not contact any part of $B$.  

Cases 0-2 become redundant with cases 4-6.  Moreover, case 3 becomes undefined.

There is no Null tangent class defined when circles $A$ and $B$ are equal, this is because there will always be at least one tangent between the circles.

Single tangents are defined for cases 0 and 4 because all points of $A$ and $B$ are in contact.  Single tangent is also defined for case 8 because $A$ and $B$ have one point of contact.

Pairs of Outer tangents will always be parallel to each other.  Pairs of Inner tangents will always intersect halfway between $C_A$ and $C_B$.


\begin{flushright}   
Q.E.D.
\end{flushright}   


