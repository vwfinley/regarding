%- Another way to view ogive is half of a circle intersection.+
%- Alternate way to construct Ogives (intersection of semi circles)
%- Relate to A-005 which cases?  what are regions of overlap?
%- A-006 stage4 check is dead give away that drawing is an ogive. 
%- Describe the math
%- relate x separation of circles to b on ogive.  distance (of x circle intersections) = 2*b
%- give radius and distance between x-intersections ogive can be calculated.
\section{Solution}
\subsection{Alternate Construction}
An alternate tangent ogive construction method is to first draw an intersection of equal circles, then retain half of the intersection.
This method is presented in figure \ref{fig:fig_a}.
It is identical to that of stage 3 of figure 1 in the previous 
article\footnote{\label{noteA006}A-006, Regarding Tangent Ogives, Vincent W. Finley, September 2022}.
The similarity between these two figures is readily seen.  
\begin{figure}[ht!]
\centering

\begin{tikzpicture}[x=1in, y=1in]
	\begin{scope}[xscale=.10, yscale=.10]	

%		\coordinate (origin) at (0,0);
%		\coordinate (P1) at (-0.0134, -0.0040);
%		\coordinate (P2) at (.0214, .0064);
%		\coordinate (P3) at (0.0086, 0.0064);
%		\coordinate (Pf) at (0.015, -0.015);
%		\coordinate (Pb) at (0.030, 0);
%		\coordinate (Pm) at (0.0075, -0.0075);		

		\clip (-21, -10) rectangle (10, 17);

		% axes
		\draw[lightgray, -{Latex[scale=1.5]}] (0, -30.0) -- (0, 15.0)  node [above] {$y$};	
		\draw[lightgray, -{Latex[scale=1.5]}] (-40.0, 0) -- (8.0, 0)  node [right] {$x$};		

		% vertical
		\draw[lightgray] (-16.0097, -35) -- (-16.0097, 35); % left

		% horizontal
		\draw[lightgray] (-50, 12.5) -- (90, 12.5) node[shift={(179.3 : 8.2)}]{$L_H$}; % L_H
		\draw[lightgray] (-19.0097, -4.0024) -- (-9.0097, -4.0024);	 % R center mark

		% sloped
		\draw[lightgray] (194.04 : 58) -- (14.04 : 98) node[shift={(-166.6 : 9.0)}]{$L_S$}; % L_S

		% bisector
		\draw[lightgray, dashed] (142.02 : -4) -- (142.02 : 24); % dashed line
		\draw[lightgray] ([shift=(135 :  20.3116)] 0, 0)  arc (135 : 150 : 20.3116); % mark

		% angle labels
		\draw[black] ([shift=(180 : 6)] 0, 0)  arc (180 : 194.04 : 6) node[shift={(-0.11, 0.06)}]{$\alpha$} ; % alpha
		\draw[black] ([shift=(194.04 : 2)] 0, 0)  arc (194.04 : 270 : 2) node[shift={(-0.17, -0.05)}]{$\beta$}; % beta
		\draw[black] ([shift=(142.02 : 4)] 0, 0)  arc (142.02 : 194.04 : 4) node[shift={(-0.06, 0.24)}]{$\gamma$}; % gamma

\draw[black] ([shift=(194.04 : 2.5)] 0, 0)  arc (194.04 : 90 : 2.5) node[shift={(-0.18, 0.02)}]{$\delta$}; % gamma

		\draw[black] ([shift=(0 : 6)] 194.04 : 16.5024)  arc (0 : 14.04 : 6) node[shift={(0.1, -0.07)}]{$\alpha$}; % alpha
		\draw[black] ([shift=(14.04 : 2)] 194.04 : 16.5024)  arc (14.04 : 90 : 2) node[shift={(0.17, 0.05)}]{$\beta$}; % beta		
		\draw[black] ([shift=(142.02 : 20.3116)] 270 : 3)  arc (270 : 322.02 : 3) node[shift={(-0.07, -0.17)}]{$\gamma$}; % gamma


		% locate arc
%		\draw[black] ([shift=(-40 : 16.5024)] 194.04 : 16.5024) arc (-40  : 105 : 16.5024); % The arc
		\draw[black, thick] (194.04 : 16.5024) circle (16.5024); % The arc
		
		% radius lines
		\draw[black, -{Latex[scale=1.0]}] (194.04 : 16.5024) -- +(90 : 16.5024) node[shift={(-94 : 0.9)}]{$r$};
		\draw[black, -{Latex[scale=1.0]}] (194.04 : 16.5024) -- +(14.04 : 16.5024) node[shift={(200 : 0.75)}]{$r$};

		% Labels
		\draw[black] (197 : 17.5)  node{$A$};
		\draw[black] (50 : 1.3)  node{$B$};
		\draw[black] ((139.6 : 22)  node{$C$};
\draw[black] ((86 : 13.4)  node{$D$};

	\end{scope}
		





\end{tikzpicture}
\caption{Tangent Ogive Alternate Construction}
\label{fig:fig_a}
\end{figure}
In this article the ogive is drawn directly in a single step, whereas, the previous article extends the process with laborious detail.
Each article has something useful to teach, as each approaches tangent ogive construction from a different direction.
The prior article emphasizes the ogive geometry.
However, this article emphasizes relationship of equal coplanar circles.
It describes tangent ogives and intersection of equal circles are related.
Furthermore, it makes connections with another previous article.

\subsection{Relationship between tangent ogives and circle intersections. Comparing to A-005 and A-006.}
Figure 2 of article A-005\footnote{\label{noteA005}A-005, Regarding Relationship and Classification of Equal Coplanar Circles, Vincent W. Finley, May 2022}, shows cases where equal coplanar circles will have regions of intersection.
These are cases 5-7 specifically, with cases 1 and 2 being redundant.
It is worth noting for all cases resulting in regions of intersecion, only a pair of Class C outer tangent lines are possible.

Recall the conditions for cases 5-7:
\begin{itemize}
\item Case 5: $0 < x_A < r_A$
\item Case 6: $x_A = r_A$
\item Case 7: $r_A < x_A < 2 \cdot r_A$
\end{itemize}

Now, combining the three conditions yields a general condition, which circle pairs must satisfiy in order to have a region of intersection:
\begin{equation}
\boxed{
0 < x_A  < 2 \cdot r_A
}
\end{equation}

Since circles $C_B$ and $C_A$ are equal, their radii $r_A = r_B = r$ are equal. Furthermore, the value $x_A$ is simply the
distance beween the centers of circles $C_B$ and $C_A$ as
described by A-004\footnote{\label{noteA004}A-004, Regarding Classification of Coplanar Circle Tangents, Vincent W. Finley, April 2022} and A-005.
In those articles $C_B$ was fixed at the origin and $C_A$ was permitted to vary some
distance $x_A$ along the non-negative x-axis.

On Stages 2 and 3 in Figure 1 from article A-006, it is seen $C_B$ is centered about point $P'$ and $C_A$ is centered about $P''$.
These center points lie along the x-axis some distance $a$ from the origin.  This is confirmed by equations 8a and 9a in A-006, where
locatons $P'$ and $P''$ are given as $P'=(-a,0)$ and $P''=(a,0)$ respectively.

Therefore the relationship between variable $x_A$ in A-005 and variable $a$ in A-006 can be described as:
% x_A &= 2 \cdot a;
\begin{equation}
x_A = 2 \cdot a
\end{equation}
where $x_A$ is the distance between centers, and:
\begin{equation}
\boxed{
a = \frac{x_A}{2}\label{eq:a}
}
\end{equation}

This relationship bridges articles A-005 and A-006 together.
Bridging of these articles is expected since both merely deal with overlapping equal coplanar circles whose centers are separated by some distance. 

To further strengthen the bridge between articles, substitute equation \ref{eq:a} above into the ogive equations: 2, 3 and 4; from A-006.
The subsitution yields the respective ogive equation in terms of circle separation distance, as follows:
\begin{equation}
\boxed{
r = \frac{x_A}{2} + b
}
\end{equation}
\begin{equation}
\boxed{
r = \sqrt{\left(\frac{x_A}{2}\right)^2+c^2}
}
\end{equation}
\begin{equation}
\boxed{
x_A = \frac{c^2 - b^2}{b}
}
\end{equation}
Note: equations: 6 and 7; from A-006 are unaffected since they are independant of $a$, and are therefore independent of $x_A$.

\subsection{Finding area of ogive cross-section} 
The area of the ogive cross-section is found by doubling the area under the circle between points $P_a$ and $P_b$ in figure \ref{fig:fig_a}.

By remembering figure \ref{fig:fig_a} is just Stage 3 from Figure 1 in article A-006, one can work backward to Stage 1.
Conceptualizing the half-ogive as shown in Stage 1 makes the integration that follows more apparent.

%The simplist way to find an integral of a circle with rectangular limits of integration is to refer to an integral table.
From a table of integrals\footnote{ISBN 0-471-85045-4, Calculus, 3rd Edition, Howard Anton, Drexel University, Copyright 1988 Anton Textbooks, Inc., John Wiley \& Sons, See integral \#40 inside front cover.} the 
definite integral: with respect to $x$, of the circle with having radius $r$, over limits $a$ and $r$; is found to be:
\begin{align*}
\left. F(x) \right]^r_a &= \int^r_a \sqrt{r^2 - x^2}\,dx \\
&=\left. \frac{x}{2}\sqrt{r^2 - x^2} + \frac{r^2}{2}sin^{-1}\left(\frac{x}{r}\right)\right]^r_a \\
&=\frac{r}{2}\sqrt{r^2 - r^2} + \frac{r^2}{2}sin^{-1}\left(\frac{r}{r}\right) \\ 
&\phantom{{}=1}- \frac{a}{2}\sqrt{r^2 - a^2} - \frac{r^2}{2}sin^{-1}\left(\frac{a}{r}\right) \\
&= \frac{\pi r^2}{4} - \frac{a}{2}\sqrt{r^2 - a^2} - \frac{r^2}{2}sin^{-1}\left(\frac{a}{r}\right)
\end{align*}
The area $A$ of the ogive cross-section is twice the area of definite integral.
\begin{equation*}
A = 2 \left[\frac{\pi r^2}{4} - \frac{a}{2}\sqrt{r^2 - a^2} - \frac{r^2}{2}sin^{-1}\left(\frac{a}{r}\right)\right]
\end{equation*}
\begin{equation}
\boxed{
A = \frac{\pi r^2}{2} - a \sqrt{r^2 - a^2} - r^2 sin^{-1}\left(\frac{a}{r}\right)
}
\end{equation}

\subsection{Finding volume of tangent ogive} 
Starting with the standard equation of a circle, substitute in $P'=(-a, 0)$, for $(h, k)$.
Doing so will give the translated circle as shown in A-006, Figure 1, Stage 2.
Solve the equation of $x$, thereby yielding a function of $y$.
\begin{align*}
r^2 &= (x-h)^2 + (y-k)^2 \\
r^2 &= (x-(-a))^2 + (y-0)^2\\
r^2 &= (x+a)^2 + y^2\\
(x+a)^2 &= r^2 - y^2 \\
 x+a &= \sqrt{r^2 - y^2}
\end{align*}
\begin{equation}
u(y) = x = -a + \sqrt{r^2 - y^2} \label{eq:uy}
\end{equation}
Using the ``Volumes by Disks Perpendicular to
the y-Axis''\footnote{ISBN 0-471-85045-4, Calculus, 3rd Edition, Howard Anton, Drexel University, Copyright 1988 Anton Textbooks, Inc., John Wiley \& Sons, page 380, equation \#7.}
method to integrate equation \ref{eq:uy}.
\begin{equation}
V = \int^d_c\pi[u(y)]^2\,dy \label{eq:V}
\end{equation}
Substituting equation \ref{eq:uy} into \ref{eq:V} yields:
\begin{align*}
V &= \int^c_0\pi\left[-a + \sqrt{r^2 - y^2}\right]^2\,dy\\
&= \pi \int^c_0\left[a^2 - 2a\sqrt{r^2 - y^2} + r^2 - y^2\right]\,dy\\
&= \pi \left[a^2y - 2ay\sqrt{r^2 - y^2} - 2ar^2 sin^{-1}\left(\frac{y}{r}\right)+ r^2y - \frac{y^3}{3}\right]^c_0\\
&= \pi \left[a^2c - 2ac\sqrt{r^2 - c^2} - 2ar^2 sin^{-1}\left(\frac{c}{r}\right)+ r^2c - \frac{c^3}{3}\right]\\
&= \pi \left[a^2c - 2a^2c - 2ar^2 sin^{-1}\left(\frac{c}{r}\right)+ r^2c - \frac{c^3}{3}\right]\\
&= \pi \left[a^2c + r^2c - \frac{c^3}{3} - 2ar^2 sin^{-1}\left(\frac{c}{r}\right)\right]
\end{align*}
Finally, substitute in $c = \sqrt{r^2 - a^2}$ and multiply through by $\pi$.
\begin{equation}
\boxed{
\begin{split}
V = \pi r^2\sqrt{r^2 - a^2} - \frac{\pi (r^2 - a^2)^{3/2}}{3} \\
- \pi ar^2 sin^{-1}\left(\frac{\sqrt{r^2 - a^2}}{r}\right)
\end{split}
}
\end{equation}

\subsection{Area of circle intersection}
- Ogive is half of intersection, intersection is twice ogive
- set operations
- Intersection, union, xor
- Venn diagrams, set operations

\subsection{Volume of circle intersection}




\begin{flushright}   
Q.E.D.
\end{flushright}   


