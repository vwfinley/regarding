\section{Solution}
An alternate Tangent Ogive construction method is to first draw an intersection of equal circles, then retain half of the intersection.
This method is presented in figure \ref{fig:fig_a}.
It is identical to that of stage 3 of figure 1 in the previous 
article\footnote{\label{noteA006}A-006, Regarding Tangent Ogives, Vincent W. Finley, September 2022}.
The similarity between these two figures is readily seen.  
\begin{figure}[ht!]
\centering

\begin{tikzpicture}[x=1in, y=1in]
	\begin{scope}[xscale=.10, yscale=.10]	

%		\coordinate (origin) at (0,0);
%		\coordinate (P1) at (-0.0134, -0.0040);
%		\coordinate (P2) at (.0214, .0064);
%		\coordinate (P3) at (0.0086, 0.0064);
%		\coordinate (Pf) at (0.015, -0.015);
%		\coordinate (Pb) at (0.030, 0);
%		\coordinate (Pm) at (0.0075, -0.0075);		

		\clip (-21, -10) rectangle (10, 17);

		% axes
		\draw[lightgray, -{Latex[scale=1.5]}] (0, -30.0) -- (0, 15.0)  node [above] {$y$};	
		\draw[lightgray, -{Latex[scale=1.5]}] (-40.0, 0) -- (8.0, 0)  node [right] {$x$};		

		% vertical
		\draw[lightgray] (-16.0097, -35) -- (-16.0097, 35); % left

		% horizontal
		\draw[lightgray] (-50, 12.5) -- (90, 12.5) node[shift={(179.3 : 8.2)}]{$L_H$}; % L_H
		\draw[lightgray] (-19.0097, -4.0024) -- (-9.0097, -4.0024);	 % R center mark

		% sloped
		\draw[lightgray] (194.04 : 58) -- (14.04 : 98) node[shift={(-166.6 : 9.0)}]{$L_S$}; % L_S

		% bisector
		\draw[lightgray, dashed] (142.02 : -4) -- (142.02 : 24); % dashed line
		\draw[lightgray] ([shift=(135 :  20.3116)] 0, 0)  arc (135 : 150 : 20.3116); % mark

		% angle labels
		\draw[black] ([shift=(180 : 6)] 0, 0)  arc (180 : 194.04 : 6) node[shift={(-0.11, 0.06)}]{$\alpha$} ; % alpha
		\draw[black] ([shift=(194.04 : 2)] 0, 0)  arc (194.04 : 270 : 2) node[shift={(-0.17, -0.05)}]{$\beta$}; % beta
		\draw[black] ([shift=(142.02 : 4)] 0, 0)  arc (142.02 : 194.04 : 4) node[shift={(-0.06, 0.24)}]{$\gamma$}; % gamma

\draw[black] ([shift=(194.04 : 2.5)] 0, 0)  arc (194.04 : 90 : 2.5) node[shift={(-0.18, 0.02)}]{$\delta$}; % gamma

		\draw[black] ([shift=(0 : 6)] 194.04 : 16.5024)  arc (0 : 14.04 : 6) node[shift={(0.1, -0.07)}]{$\alpha$}; % alpha
		\draw[black] ([shift=(14.04 : 2)] 194.04 : 16.5024)  arc (14.04 : 90 : 2) node[shift={(0.17, 0.05)}]{$\beta$}; % beta		
		\draw[black] ([shift=(142.02 : 20.3116)] 270 : 3)  arc (270 : 322.02 : 3) node[shift={(-0.07, -0.17)}]{$\gamma$}; % gamma


		% locate arc
%		\draw[black] ([shift=(-40 : 16.5024)] 194.04 : 16.5024) arc (-40  : 105 : 16.5024); % The arc
		\draw[black, thick] (194.04 : 16.5024) circle (16.5024); % The arc
		
		% radius lines
		\draw[black, -{Latex[scale=1.0]}] (194.04 : 16.5024) -- +(90 : 16.5024) node[shift={(-94 : 0.9)}]{$r$};
		\draw[black, -{Latex[scale=1.0]}] (194.04 : 16.5024) -- +(14.04 : 16.5024) node[shift={(200 : 0.75)}]{$r$};

		% Labels
		\draw[black] (197 : 17.5)  node{$A$};
		\draw[black] (50 : 1.3)  node{$B$};
		\draw[black] ((139.6 : 22)  node{$C$};
\draw[black] ((86 : 13.4)  node{$D$};

	\end{scope}
		





\end{tikzpicture}
\caption{Tangent Ogive Alternate Construction}
\label{fig:fig_a}
\end{figure}
In this article the ogive is drawn directly in a single step, whereas, the previous article extends the process with laborious detail.
Each article has something useful to teach, as each approaches the problem from a different direction.
The prior article emphasizes the ogive geometry.
However, this article emphasizes relationship of equal coplanar circles.
It describes tangent ogives and intersection of equal circles are related.
Furthermore, it makes connections with another previous article\footnote{\label{noteA005}A-005, Regarding Relationship and Classification of Equal Coplanar Circles, Vincent W. Finley, May 2022}.

\subsection{Comparison to A-005. Overlapping equal coplanar circles}
In the article regarding equal coplanar circles, figure 2 show cases where equal coplanar circles will have regions of intersection.
These are cases 5-7 specifically, with cases 1 and 2 being redundant.
For all cases resulting in regions of intersecion, only a pair of Class C outer tangent lines are possible.

Recalling the conditions for cases 5-7:
\begin{itemize}
\item Case 5: $0 < x_A < r_A$
\item Case 6: $x_A = r_A$
\item Case 7: $r_A < x_A < 2 \cdot r_A$
\end{itemize}

Combining the three conditions yields a composite condition which circle pairs must satisfiy to have an intersection.
$0 < x_A  < 2 * r_A$

The cases 
- Another way to view ogive is half of a circle intersection.+
- Alternate way to construct Ogives (intersection of semi circles)
- Relate to A-005 which cases?  what are regions of overlap?
- A-006 stage4 check is dead give away that drawing is an ogive. 
- Describe the math
- relate x separation of circles to b on ogive.  distance (of x circle intersections) = 2*b
- give radius and distance between x-intersections ogive can be calculated.




\subsection{Integrate area} 

The area of the ogive cross section can be found by doubling the area under the circle
between points $P_a$ and $P_b$.

The easiest way to find the integral of a circle with rectangular limits of integration is to lookup the integral in a table of
integrals.

From the table of integrals\footnote{ISBN 0-471-85045-4, Calculus, 3rd Edition, Howard Anton, Drexel University, Copyright 1988 Anton Textbooks, Inc., John Wiley \& Sons, See integral \#40 inside front cover} it is found the indefinite integral 
with respect to $x$ of circle with radius $r$ is:
\begin{align*}
\int \sqrt{a^2 - u^2} \,du &= \frac{u}{2}\sqrt{a^2 - u^2} + \frac{a^2}{2}sin^{-1}\left(\frac{u}{a}\right) + C \\
\int \sqrt{r^2 - x^2} \,dx &= \frac{x}{2}\sqrt{r^2 - x^2} + \frac{r^2}{2}sin^{-1}\left(\frac{x}{r}\right) + C
\end{align*}
The definite integral with limits $a$ and $b$ would be:
\begin{align*}
\left. F(x) \right]^b_a &= \int^b_a \sqrt{r^2 - x^2} \,dx \\
&=\left. \frac{x}{2}\sqrt{r^2 - x^2} + \frac{r^2}{2}sin^{-1}\left(\frac{x}{r}\right)\right]^b_a \\
&=\frac{b}{2}\sqrt{r^2 - b^2} + \frac{r^2}{2}sin^{-1}\left(\frac{b}{r}\right) \\ 
&\phantom{{}=1}- \frac{a}{2}\sqrt{r^2 - a^2} - \frac{r^2}{2}sin^{-1}\left(\frac{a}{r}\right)
\end{align*}
Substituing in $a$ and $r$ as the limits of integration yields,
\begin{align*}
&=\frac{r}{2}\sqrt{r^2 - r^2} + \frac{r^2}{2}sin^{-1}\left(\frac{r}{r}\right) \\ 
&\phantom{{}=1}- \frac{a}{2}\sqrt{r^2 - a^2} - \frac{r^2}{2}sin^{-1}\left(\frac{a}{r}\right) \\
&= \frac{\pi r^2}{4} - \frac{a}{2}\sqrt{r^2 - a^2} - \frac{r^2}{2}sin^{-1}\left(\frac{a}{r}\right)
\end{align*}
The area $A$ of the ogive cross-section is twice the area of definite integral.
\begin{equation*}
A = 2 \left[\frac{\pi r^2}{4} - \frac{a}{2}\sqrt{r^2 - a^2} - \frac{r^2}{2}sin^{-1}\left(\frac{a}{r}\right)\right]
\end{equation*}
\begin{equation}
\boxed{
A = \frac{\pi r^2}{2} - a \sqrt{r^2 - a^2} - r^2 sin^{-1}\left(\frac{a}{r}\right)
}
\end{equation}

\subsection{Integrate volume} 


\subsection{Area and Volume of circle intersection}
- Ogive is half of intersection, intersection is twice ogive
- Intersection, union, xor


\begin{flushright}   
Q.E.D.
\end{flushright}   


