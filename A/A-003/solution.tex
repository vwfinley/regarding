%% List of Steps
\section{Solution}
\begin{enumerate}
\item
Refer to Figure \ref{fig:fig_b} for the discussion that follows.

\item
Since $A$ and $B$ are coplanar circles, center points $C_A$ and $C_B$ lie in the same plane with the circles.

\item
Two points define the line between them, therefore a line can be drawn between $C_A$ and $C_B$.  This line
shall be coplanar with circles $A$ and $B$, and their centers.

\item
A coordinate system can be drawn in the plane containing $A$ and $B$, and the line between their centers.
For convenience, the coordinate system origin can be arbitrarily chosen to coincide with $C_B$.

\begin{figure}[hb!]
\centering

% Code 145 wheel profile

\begin{tikzpicture}[x=1in, y=1in]
	\begin{scope}[xscale=45, yscale=45]	
		\coordinate (origin) at (0,0);
		\coordinate (P1) at (-0.0134, -0.0040);
		\coordinate (P2) at (.0214, .0064);
		\coordinate (P3) at (0.0086, 0.0064);
		\coordinate (Pf) at (0.015, -0.015);
		\coordinate (Pb) at (0.030, 0);
		\coordinate (Pm) at (0.0075, -0.0075);
		
		\clip (-.02, -.03) rectangle (.05,.05);

		% axes
		\draw[black, -{Latex[scale=1.5]}] (0, -0.030) -- (0, 0.030)  node [above] {$y$};	
		\draw[black, -{Latex[scale=1.5]}] (-0.090, 0) -- (0.04500, 0)  node [right] {$x$};		

		% vertical
		\draw[lightgray] (-.080, -.025) -- (-.080, .025);		% left
		\draw[lightgray] (.030, -.025) -- (.030, .025);		% right 
		\draw[lightgray] (.015, -.025) -- (.015, .025);		% middle 

		% horizontal
		\draw[lightgray] (-0.085, 0.010) -- (0.04, 0.010);		% above xaxis
		\draw[lightgray] (-0.085, -0.015) -- (0.04, -0.015);		% below xaxis

		% locate R1 center
%		\draw[lightgray] ([shift=(170 : 0.014)] 0, 0) arc (170 : 230 : 0.014); % construction mark
%		\draw[lightgray] (-0.017, -0.004) -- (-0.010, -0.004);  %P - R1 horiz mark
%		\draw[lightgray, -{Latex[scale=1.0]}] (origin) -- (210: 0.014)  node [shift={(0.35, .1)}] {$R1$};	

		% draw arc 1
%		\draw[black] ([shift=(-30 : 0.014)] 196.6 : 0.014) arc (-30 : 115 : 0.014); % arc
%		\draw[lightgray, -{Latex[scale=1.0]}] (P1) -- ++(45 : 0.014)  node [shift={(-0.3, -.15)}] {$R1$};	% R1 arrow

		% R1, R2' centers line and arc
%		\draw[lightgray] (196.6 : 0.024) -- (16.6 : 0.045);  % centers line
%		\draw[lightgray] ([shift=(65 : 0.0223)] Pf) arc (65 : 115 : 0.0223); % arc
%		\draw[lightgray, -{Latex[scale=1.0]}] (Pf) -- ++(100: 0.0223)  node [shift={(0.2, -0.55)}] {$R2'$};	

		% locate R2' center
%		\draw[lightgray] ([shift=(10 : 0.0223)] origin) arc (10 : 23 : 0.0223); % arc

		% locate R3 center
%		\draw[lightgray] ([shift=(157 : 0.0223)] Pb) arc (157 : 170 : 0.0223); % arc
		
		% Draw C2
%		\draw[black] ([shift=(180 : 0.0223)] P2) arc (180 : 290 : 0.0223); % arc
%		\draw[lightgray, -{Latex[scale=1.0]}] (P2) -- ++(-140 : 0.0223)  node [shift={(.20, 0.3)}] {$R2'$};	

		% Draw C3
%		\draw[black] ([shift=(-110 : 0.0223)] P3) arc (-110 : 0 : 0.0223); % arc
%		\draw[lightgray, -{Latex[scale=1.0]}] (P3) -- ++(-40 : 0.0223)  node [shift={(-.20, 0.3)}] {$R3'$};	

		% Bisection
%		\draw[lightgray] (origin) -- (Pf);
%		\draw[lightgray] (Pf) -- (P2);
%		\draw[lightgray, dashed] ([shift=(-135 : 0.04)] P2) -- ++(45 : 0.06);
		
%		\begin{scope} [shift={(Pm)}]
%		\begin{scope} [rotate=-45]
%			\draw[lightgray] (0,.0015) -- (.0015, .0015) -- ++(0, -.0015); 			
%		\end{scope}
%		\end{scope}

%	\end{scope}
		
		

		

\end{tikzpicture}
\caption{Locations of $C_A$ relative to $C_B$}
\label{fig:fig_b}
\end{figure}

\item
Furthermore, for convenience, the coordinate system can be oriented such that the x-axis aligns with the
line connecting $C_B$ and $C_A$.

\item
Since the origin is always relative to $C_B$, $C_B$ will always be at $C_B = (x_B, y_B) = (0, 0)$.

\item
Since the x-axis is aligned with the line connecting $C_B$ and $C_A$, $C_A$ will always lie along the x-axis, such
that:  $C_A = (x_A, y_A) = (x_A, 0)$.

\item
Furthermore, the distance between $C_A$ and $C_B$ will always be $\Delta x = x_A - x_B = x_A - 0 = x_A$

\item
Therefore the problem of finding critical locations of $C_A$ simplifies into the trivial problem of finding critical values of $x_A$.

\item
The locations of $C_A$ relative to $C_B$ can be generalized into the 10 cases shown in Figure \ref{fig:fig_b}.

\begin{itemize}
\item \textbf{case 0:}
$x_A = 0$

At the critical point where $C_A$ is at the origin, $C_A$ will be concentric with $C_B$.

\item \textbf{case 1:}
$0 < x_A < r_A$

When $C_A$ lies less than one $r_A$ radius from the origin, $A$ will overlap left and right regions of $B$.

\item \textbf{case 2:}
$ x_A = r_A$

When $C_A$ lies exactly than one $r_A$ radius from the origin, circle $A$ will pass through point $C_B$.

\item \textbf{case 3:}
$r_A < x_A < r_B - r_A$

In this case, $C_A$ overlaps with the right side region of $B$.

\item \textbf{case 4:}
$ x_A = r_B - r_A$

When $C_A$ lies at one $r_A$ radius less than $r_B$, $A$ and $B$ will share a tangent point.  $A$ will overlap with right region of $B$.

\item \textbf{case 5:}
$ r_B - r_A < x_A < r_B$

In this case, $A$ only partially overlaps $B$.  Furthermore it overlaps the right edge of $B$.

\item \textbf{case 6:}
$ x_A = r_B$

$C_A$, the center of $A$, lies on circle $B$.

\item \textbf{case 7:}
$ r_B < x_A < r_B + r_A$

The center of $A$ is beyond the edge of circle $B$.  Circle $A$ partially overlaps the right side of $B$.

\item \textbf{case 8:}
$ x_A = r_B + r_A$

Circles $A$ an $B$ are tangent to each other with no overlap.


\item \textbf{case 9:}
$ x_A > r_B + r_A$

Circle $A$ is entirely outside $B$.  $A$ and $B$ are non-overlapping.

\end{itemize}
\end{enumerate}


\begin{flushright}   
Q.E.D.
\end{flushright}   



