\section{Example}

\paragraph{Given:}
\begin{itemize}
\item Two circles $A$ and $B$ whose radii are $r_A = 1.0$ and $r_B = 3.0$ units respectively,
\item \emph{and} circle centers $C_A$ and $C_B$ separated by a distance of 6.5 units.
\end{itemize}

\paragraph{Find:}
\begin{itemize}
\item Which coplanar circle-pair case best matches the given arrangement.
\item Which tangent classes apply to the circle-pair case.
\item The location(s) where tangent lines intersect.
\item The angle of intersection between tangent lines.
\end{itemize}

\paragraph{Steps:}
\begin{enumerate}
\item
Let: 
\begin{align*}
	r_A &= 1.0\\
	r_B &= 3.0\\
	x_A &= 6.5
\end{align*}

\item
From the solution section of article\footnotemark[\ref{noteA003}], use the inequality for case 9:
\begin{equation*}
	x_A > r_B + r_A
\end{equation*}

\item
Plug in values for $r_A$ and $r_B$,
\begin{equation*}
	6.5 > 3.0 + 1.0
\end{equation*}
\begin{equation*}
	6.5 > 4.0
\end{equation*}

\item
Since $x_A = 6.5$ is greater than the sum of the radii, 3.0 and 1.0, the inequality for circle-pair case 9 is satisfied.  Therefore 
the relationship between circles $A$ and $B$ are classified by the case 9 circle-pair.

\item
From table \ref{table:tab_tangents} tangent classes (C) outer and (D) inner apply to circle-pair case 9.  Therefore circles $A$ and
$B$ are connected by two outer tangent lines and two inner tangent lines.

\item
Outer tangents:\\
The point of intersection $I_{outer}$ for the outer tangent pair is found by using Equation \ref{eq:xI} to get the $x$ coordinate.
Substituting values from above yields.
\begin{align*}
x_I &= \frac{3.0 \cdot 6.5}{3.0 - 1.0}   \\
x_I &= 9.75
\end{align*}

Then $x_I$ can be substituted into Equation \ref{eq:pointI} to find the point $I_{outer}$.
\begin{equation*}
\boxed
{
	I_{outer} = (9.75, 0)
}
\end{equation*}
The included angle between the outer tangent pair can be calculated from Equation \ref{eq:thetaarcsin}.
\begin{align*}
	\theta_{outer} &= 2 \cdot \arcsin{\left( \frac{3.0}{9.75}  \right)} \\
	\theta_{outer} &= 0.62553 \hspace{1 mm} rads
\end{align*}
\begin{equation*}
\boxed
{
	\theta_{outer} = 35.84^{\circ}
}
\end{equation*}

\item
Inner tangents:\\
The point of intersection $I_{inner}$ for the inner tangent pair is found by using Equation \ref{eq:xI2} to get the $x$ coordinate.
Taking $x_{A}$ to be $x_{A'}$ in this context, and substituting values from above yields.
\begin{align*}
x_I &= \frac{3.0 \cdot 6.5}{3.0 + 1.0}   \\
x_I &= 4.875
\end{align*}
Then $x_I$ can be substituted into Equation \ref{eq:pointI} to find the point $I_{inner}$.
\begin{equation*}
\boxed
{
	I_{inner} = (4.875, 0)
}
\end{equation*}
The included angle between the inner tangent pair can be calculated from Equation \ref{eq:thetaarcsin}.
\begin{align*}
	\theta_{inner} &= 2 \cdot \arcsin{\left( \frac{3.0}{4.875}  \right)} \\
	\theta_{inner} &= 1.32575  \hspace{1 mm} rads
\end{align*}
\begin{equation*}
\boxed
{
	\theta_{inner} = 75.96^{\circ}
}
\end{equation*}

\item
Finally, all tangent points can be found by referring to Equations 7 and 8 from article\footnotemark[\ref{noteA002}].  This is left as an exercise for the reader.
\end{enumerate}