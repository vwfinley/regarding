\section{Solution}
As shown in Figure \ref{fig:fig_a} paired circles, $A$ and $B$, can be gathered into 4 classes.
The classes are based upon the relationship between paired circles and their common tangent line(s), if any.

\begin{figure}[ht]
\centering
\begin{tikzpicture}[x=1in,y=1in]

	\coordinate (origin) at (0,0);
	\coordinate (I) at (0,0);
	\coordinate (t1) at (30 : 0.8660254038);
	\coordinate (t2) at (-30 : 0.8660254038);
	\coordinate (t1n) at (210 : 0.8660254038);
	\coordinate (t2n) at (150 : 0.8660254038);
	\coordinate (t1s) at (30:2);
	\coordinate (t2s) at (-30:2);
	\coordinate (C) at (0:1);
	

	% centermark
%VWF_FIGA	
	\begin{scope}  [shift={(C)}]	
		\draw (-0.1, 0.0) -- (0.1, 0.0);
		\draw (0.0, -0.1) -- (0.0, 0.1);
	\end{scope}	
%
	% axes
% VWF_FIGB
%	\draw[-{Latex[scale=1.5]}] (-1.0,0) -- (2,0)  node [right] {$x$};
%	\draw[-{Latex[scale=1.5]}] (0,-1.5) -- (0,1.5)  node [above] {$y$};


	\draw[dotted] (t1n) -- (t1);
	\draw[dotted] (t2n) -- (t2);

	\draw (t1) -- (t1s);
	\draw (t2) -- (t2s);
	
% VWF_FigA	
	\draw (t1) arc[start angle=120, end angle=240, radius=0.5];
% 	\draw [dashed] (t1) arc[start angle=120, end angle=-120, radius=0.5];
%	\draw (t1) arc[start angle=120, end angle=-120, radius=0.5];
%	

% VWF_FigA	
	 \draw[-{Latex}] (C) -- (t1);
%

% VWF_FIGB
	% \draw (C) ellipse (0.5);
	% \draw (C) -- (t1);
%
	
	
% VWF_FIGB
%	\begin{scope}  [shift={(t1)}]
%		\draw (210:0.1) -- ++(-60:0.1); 
%		\draw (-60:0.1) -- ++(210:0.1); 
%	\end{scope}	
	

	% point I
	\begin{scope} [shift={(I)}]
		\draw [fill=black] circle (0.02);
	\end{scope}

	% point C
% VWF_FIGB	
%	\begin{scope} [shift={(C)}]
%		\draw [fill=black] circle (0.02);
%	\end{scope}
	
	% point P
% VWF_FIGB
%	\begin{scope} [shift={(t1)}]
%		\draw [fill=black] circle (0.02);
%	\end{scope}

% VWF_FIGB
%	\begin{scope} [shift={(I)}]
%		\draw (0.25,0) arc[start angle=0, end angle=30, radius=0.25];
%	\end{scope}


	\begin{scope} [shift={(I)}]
		\draw (150:0.25) arc[start angle=150, end angle=210, radius=0.25];
	\end{scope}
	
	
	%\node [label={Hello}] ;
	\node [label={[shift={(0.1,-0.25)}]$C$}] at (C) {};
	\node [label={[shift={(-0.2,-0.1)}]$L_1$}] at (t1s) {};
	\node [label={[shift={(-0.2,0.1)}]$L_2$}] at (t2s) {};
	\node [label={[shift={(-0.05,0.1)}]$r$}] at (C) {};
	\node at (165:0.35) {$\theta$};
	\node [label={[shift={(0.05,-0.28)}]$I$}] at (I) {};
	
% VWF_FIGB	
%	\node [label={[shift={(0.0,-0.04)}]$P$}] at (t1) {};
%	\node [label={[shift={(0.65,-0.25)}]$\Delta x$}] at (I) {};	
%	\node [label={[shift={(0.33,0.15)}]$d$}] at (I) {};	
%	\node at (15:0.35) {$\alpha$};
%

\end{tikzpicture}
\caption{Tangent Classes}
\label{fig:fig_a}
\end{figure}

%%%
\paragraph{Class A: Null\\} 
No common tangent line between circles $A$ and $B$ is possible.  This is due to two reasons.

First, any line tangent to $A$ will contact $B$ at two points.  The points where the line contacts $B$ will
be points of intersection rather than points of tangency.

Second, a line tangent to $B$ cannot pass through $B$.  Since $A$ lies entirely within $B$ any line tangent 
to $B$ cannot contact $A$.

%Cases 0, 1, 2 and 3; from article\footnotemark[\ref{noteA003}] are examples of this class.

%%%
\paragraph{Class B: Single\\}
Where circle $A$ contacts $B$ at a single point, $A$ and $B$ will be tangent to each other.

Any line that is tangent to $B$ at the point of contact will also be tangent to $A$.

A line tangent to a circle will always intersect a radius line at right 
angles\footnote{ISBN 0-8311-2575-6, Machinery's Handbook, page 46}.
Any line that is tangent to $B$ will be at right angles to the radius
line from the center of $B$ to the tangent point.  

Since $r_B$ is given as the the radius of $B$, the tangent point of any line tangent to $B$ will lie a distance of
$r_B$ from the center of $B$.  A coordinate system can be arbitrarily constructed such that the origin coincides with
the center of $B$, and the x-axis coincides with the line between the centers of $A$ and $B$.  In such a construction the
tangent point will lie at a point ($r_B$, $0$).

Furthermore the center of $A$ must lie a distance $r_A$ from tangent point ($r_B$, $0$).  Therefore the center of $A$
can lie at locations:
\begin{equation}
\boxed
{
C_A=(r_B \pm r_A, 0)
}
\end{equation}

%These two positions correspond to cases 4 and 8 of article\footnotemark[\ref{noteA003}]. 

As the center of $A$ moves from positions covered by Class C and Class D, toward positions covered by Class B, pairs of tangents degenerate into the single tangent line detailed by Class B.

%Cases 4 and 8; from article\footnotemark[\ref{noteA003}] are examples of this class.

%%%
\paragraph{Class C: Outer\\}

The point of intersection for a pair of tangents lays outside the line segment connecting two circle centers, 
thus the name of the class. It should be noted: the arrangement for this tangent class is identical to the arrangement for 
fillets\footnote{\label{noteA002}A-002, Regarding Fillets, Vincent W. Finley, February 2022}.

However, there are two important difference between fillets and outer tangents.  First, when examining fillets the 
point and angle of intersection, $I$ and $\theta$, are usually known, while the center point of the fillet, $C$, is sought.  
The opposite is true for outer tangents.  Second, a fillet has only one circle.  Whereas the outer tangent class 
involves tangents common to two circles.

Due to these differences a geometric approach, based on similar triangles, is taken here.

\begin{figure}[ht]
\centering

% Code 145 wheel profile

\begin{tikzpicture}[x=1in, y=1in]
	\begin{scope}[xscale=45, yscale=45]	
		\coordinate (origin) at (0,0);
		\coordinate (P1) at (-0.0134, -0.0040);
		\coordinate (P2) at (.0214, .0064);
		\coordinate (P3) at (0.0086, 0.0064);
		\coordinate (Pf) at (0.015, -0.015);
		\coordinate (Pb) at (0.030, 0);
		\coordinate (Pm) at (0.0075, -0.0075);
		
		\clip (-.02, -.03) rectangle (.05,.05);

		% axes
		\draw[black, -{Latex[scale=1.5]}] (0, -0.030) -- (0, 0.030)  node [above] {$y$};	
		\draw[black, -{Latex[scale=1.5]}] (-0.090, 0) -- (0.04500, 0)  node [right] {$x$};		

		% vertical
		\draw[lightgray] (-.080, -.025) -- (-.080, .025);		% left
		\draw[lightgray] (.030, -.025) -- (.030, .025);		% right 
		\draw[lightgray] (.015, -.025) -- (.015, .025);		% middle 

		% horizontal
		\draw[lightgray] (-0.085, 0.010) -- (0.04, 0.010);		% above xaxis
		\draw[lightgray] (-0.085, -0.015) -- (0.04, -0.015);		% below xaxis

		% locate R1 center
%		\draw[lightgray] ([shift=(170 : 0.014)] 0, 0) arc (170 : 230 : 0.014); % construction mark
%		\draw[lightgray] (-0.017, -0.004) -- (-0.010, -0.004);  %P - R1 horiz mark
%		\draw[lightgray, -{Latex[scale=1.0]}] (origin) -- (210: 0.014)  node [shift={(0.35, .1)}] {$R1$};	

		% draw arc 1
%		\draw[black] ([shift=(-30 : 0.014)] 196.6 : 0.014) arc (-30 : 115 : 0.014); % arc
%		\draw[lightgray, -{Latex[scale=1.0]}] (P1) -- ++(45 : 0.014)  node [shift={(-0.3, -.15)}] {$R1$};	% R1 arrow

		% R1, R2' centers line and arc
%		\draw[lightgray] (196.6 : 0.024) -- (16.6 : 0.045);  % centers line
%		\draw[lightgray] ([shift=(65 : 0.0223)] Pf) arc (65 : 115 : 0.0223); % arc
%		\draw[lightgray, -{Latex[scale=1.0]}] (Pf) -- ++(100: 0.0223)  node [shift={(0.2, -0.55)}] {$R2'$};	

		% locate R2' center
%		\draw[lightgray] ([shift=(10 : 0.0223)] origin) arc (10 : 23 : 0.0223); % arc

		% locate R3 center
%		\draw[lightgray] ([shift=(157 : 0.0223)] Pb) arc (157 : 170 : 0.0223); % arc
		
		% Draw C2
%		\draw[black] ([shift=(180 : 0.0223)] P2) arc (180 : 290 : 0.0223); % arc
%		\draw[lightgray, -{Latex[scale=1.0]}] (P2) -- ++(-140 : 0.0223)  node [shift={(.20, 0.3)}] {$R2'$};	

		% Draw C3
%		\draw[black] ([shift=(-110 : 0.0223)] P3) arc (-110 : 0 : 0.0223); % arc
%		\draw[lightgray, -{Latex[scale=1.0]}] (P3) -- ++(-40 : 0.0223)  node [shift={(-.20, 0.3)}] {$R3'$};	

		% Bisection
%		\draw[lightgray] (origin) -- (Pf);
%		\draw[lightgray] (Pf) -- (P2);
%		\draw[lightgray, dashed] ([shift=(-135 : 0.04)] P2) -- ++(45 : 0.06);
		
%		\begin{scope} [shift={(Pm)}]
%		\begin{scope} [rotate=-45]
%			\draw[lightgray] (0,.0015) -- (.0015, .0015) -- ++(0, -.0015); 			
%		\end{scope}
%		\end{scope}

%	\end{scope}
		
		

		

\end{tikzpicture}
\caption{Arrangement of Outer Tangents}
\label{fig:fig_b}
\end{figure}

Similar triangles are are scaled versions of each other.  They are triangles whose corresponding angles are the congruent, yet whose corresponding sides are proportional.  In other words the respective angles of two similar triangles are equal.  However, the respective sides are multiples of each other.  That is, the ratio of corresponding sides will always be equal.

If Circles $A$ and $B$ are arranged as shown in Figure \ref{fig:fig_b}, a pair of tangent lines can be drawn to connect
them.  The tangent lines will be at right angles to radius lines from the circle center points to the tangent points.

The upper tangent line will contact $A$ and $B$ at tangent points $R_A$ and $R_B$ respectively.  It will lie at right angles
to the respective radius lines.

Each tangent line is a reflection of the other.  Together they are symmetric about the x-axis.  
Both intersect the x-axis at some point $I=(x_{I}, 0)$.  Included angle $\theta$, whose vertex is point $I$, results
from the intersection.

Triangles $\triangle IC_{A}R_{A}$ and $\triangle IC_{B}R_{B}$ are similar triangles, where side $\overline{IC_A}$ corresponds
to $\overline{IC_B}$, and side $\overline{C_{A}R_{A}}$ corresponds to $\overline{C_{B}R_{B}}$.  

Let $r_A$ be the length of $\overline{C_{A}R_{A}}$, such that $r_A = C_{A}R_{A}$.  Likewise: let $r_B = C_{B}R_{B}$ and
let  $x_I = IC_B$.  Moreover, the center of $A$, $C_A = (x_A, 0)$, and length of $\overline{IC_{A}}$ is $IC_A = x_I - x_A$.

Since triangles $\triangle IC_{A}R_{A}$ and $\triangle IC_{B}R_{B}$ are similar, the ratio of their corresponding sides are equal.
Therefore:

\begin{equation}
	\frac{IC_A}{IC_B} = \frac{ C_{A}R_{A}}{C_{B}R_{B}}  
\end{equation}

\begin{align*}
\frac{x_I - x_A}{x_I} &= \frac{r_A}{r_B}\\
r_B(x_I - x_A) &=	r_A \cdot x_I\\
r_B \cdot x_I - r_B \cdot x_A &= r_A \cdot x_I\\
r_B \cdot x_I - r_A \cdot x_I  &=	r_B \cdot x_A \\
x_I ( r_B - r_A) &= 	r_B \cdot x_A 
\end{align*}
\begin{equation}
\boxed
{
	x_I = \frac{r_B \cdot x_A}{r_B - r_A}  \label{eq:xI}
}
\end{equation}

If the radii of two circles are known, and the distance that separates circle centers is known, then Equation \ref{eq:xI} above
can be used to find the location where the tangent lines will intersect at point:
\begin{equation}
\boxed
{
	I = (x_I, 0)  \label{eq:pointI}
}
\end{equation}

By recognizing the fact that two tangent lines together with a circle form a fillet, the angle of intersection $\theta$ can then be found %by rearranging Equation 4 from article\footnotemark[\ref{noteA002}], and substituting variables for triangle $\triangle IC_{B}R_{B}$ as follows:  
\begin{equation*}
	\theta = 2 \cdot \arcsin{\left( \frac{r}{\Delta x}  \right)} 
\end{equation*}
\begin{equation}
\boxed
{
	\theta = 2 \cdot \arcsin{\left( \frac{r_B}{x_I}  \right)} \label{eq:thetaarcsin}
}
\end{equation}

The tangent points: $R_A$, $R_B$, and their reflections; can likewise be found using Equations 7 and 8 from
%article\footnotemark[\ref{noteA002}].

%Cases 5, 6, 7, 8 and 9; from article\footnotemark[\ref{noteA003}] are examples of this class.

%%%
\paragraph{Class D: Inner\\}

This class is characterized by the point of intersection laying inside the line segment connecting two circle centers.

By inspecting Figures \ref{fig:fig_b} and \ref{fig:fig_c} the relationship between Inner tangents and Outer tangents 
becomes immediately obvious.   

\begin{figure}[ht]
\centering
\begin{tikzpicture}[x=1in,y=1in]

	\coordinate (origin) at (0,0);
	\coordinate (I) at (0,0);
	\coordinate (t1) at (30 : 0.8660254038);
	\coordinate (t2) at (-30 : 0.8660254038);
	\coordinate (t1n) at (210 : 0.8660254038);
	\coordinate (t2n) at (150 : 0.8660254038);
	\coordinate (t1s) at (30:2);
	\coordinate (t2s) at (-30:2);
	\coordinate (C) at (0:1);

	\coordinate (B) at (0,0);
	
	\coordinate (A0) at (0*0.05,0);
 	\coordinate (A1) at (1*0.05,0);
	\coordinate (A2) at (2*0.05,0);
	\coordinate (A3) at (3*0.05,0);
	\coordinate (A4) at (4*0.05,0);

	\coordinate (A5) at (5*0.05,0);
	\coordinate (A6) at (6*0.05,0);
	\coordinate (A7) at (7*0.05,0);

	\coordinate (A8) at (8*0.05,0);
	\coordinate (A9) at (9*0.05,0);


	\coordinate (Origin0) at (0, -0*0.6);
	\coordinate (Origin1) at (0, -1*0.6);

%	\draw[-{Latex[scale=1.5]}] (0, 9.75 * -0.65) -- (0, 0.5)  node [above] {$y$};
	\draw[lightgray, -{Latex[scale=1.5]}] (0, 5.5 * -0.65) -- (0, 0.5)  node [above] {$y$};




	\foreach \i in {0,...,0}
	{
		\begin{scope} [shift={(0, \i * -0.65 )}]
			\draw[lightgray, -{Latex[scale=1.5]}] (-0.5,0) -- (1.3,0)  node [right] {$x$};	
			\draw node [shift={(0.65,-0.45)}, anchor=west] {Class A: Null};			
%			\draw[lightgray] circle (0.3) node [shift={(0.30, 0.25)}] {$B$};
			\draw node [shift={(0.45, 0.08)}] {$Undefined$};
			\begin{scope} [shift={(\i * 0.05, 0)}]
%				\draw[lightgray] circle (0.1) node [shift={(0.15, 0.10)}] {$A$};
%				\draw (-0.025,0) -- (0.025, 0); % centermark
%				\draw (0, -0.025) -- (0, 0.025);				
			\end{scope}
		\end{scope}
	}

%	\foreach \i in {4,7,9}
%	{
%		\begin{scope} [shift={(0, \i * -0.65 )}]
%			\draw[lightgray, -{Latex[scale=1.5]}] (-0.5,0) -- (0.75,0);	
%			\draw node [shift={(0.52,-0.24)}] {case \i};
%			\draw[lightgray] circle (0.3);
%			\begin{scope} [shift={(\i * 0.05, 0)}]
%				\draw circle (0.1);				
%				\draw (-0.025,0) -- (0.025, 0);
%				\draw (0, -0.025) -- (0, 0.025);
%			\end{scope}
%		\end{scope}
%	}
	declare function={ Nprime(x) = x; 
			}
		]
  	\foreach \i\j\k in {4/1/Class B: Single}
	{
			\begin{scope} [shift={(0, \j * -1.0 )}]
			\draw[lightgray, -{Latex[scale=1.5]}] (-0.5,0) -- (1.3,0); % x-axis
			\draw node [shift={(0.65,-0.45)}, anchor=west] {\k};  % label
			\draw[lightgray] circle (0.3) node [shift={(-0.2, 0.35)}] {$B$}; % circle B
			\begin{scope} [shift={(-4  * 0.15, 0)}]		
				\begin{scope} [shift={(\i * 0.15, 0)}]
					\draw[lightgray] circle (0.3) node [shift={(0.2, 0.35)}] {$A$}; % circle A				
					\draw (-0.025,0) -- (0.025, 0); % centermark
					\draw (0, -0.025) -- (0, 0.025);
				\end{scope}
			\end{scope}
		\end{scope}
	}
  	\foreach \i\j\k in {7/2/Class C: Outer,9/3/Class D: Inner}
	{
			\begin{scope} [shift={(0, \j * -1.0 )}]
			\draw[lightgray, -{Latex[scale=1.5]}] (-0.5,0) -- (1.3,0); % x-axis
			\draw node [shift={(0.65,-0.45)}, anchor=west] {\k};  % label
			\draw[lightgray] circle (0.3) node [shift={(-0.35, 0.2)}] {$B$}; % circle B
			\begin{scope} [shift={(-4  * 0.15, 0)}]		
				\begin{scope} [shift={(\i * 0.15, 0)}]
					\draw[lightgray] circle (0.3) node [shift={(0.35, 0.2)}] {$A$}; % circle A				
					\draw (-0.025,0) -- (0.025, 0); % centermark
					\draw (0, -0.025) -- (0, 0.025);
				\end{scope}
			\end{scope}
		\end{scope}
	}
	

	\begin{scope} [shift={(0, 1 * -1.0 )}]
		\begin{scope} [shift={(0.3, 0 )}]
			\draw (90.0 : 0.5) -- (-90.0 : 0.5); % common tangent
		\end{scope}		
	\end{scope}


	\begin{scope} [shift={(0, 2 * -1.0 )}]
		\begin{scope} [shift={(90 : 0.3 )}]
			\draw (180 : 0.4) -- (0 : 0.9); % outer tangent (I)
		\end{scope}		
		\begin{scope} [shift={(-90 : 0.3 )}]
			\draw (180 : 0.4) -- (0 : 0.9); % outer tangent (I)
		\end{scope}		
	\end{scope}



	\begin{scope} [shift={(0, 3 * -1.0 )}]	
		\begin{scope} [shift={(0.375, 0 )}]
			\draw (-53.13 : 0.45) -- (180.0 - 53.13 : 0.45); % outer tangent (I)
			\draw (53.13 : 0.45) -- (-180.0 + 53.13 : 0.45); % outer tangent (I)
		\end{scope}			
	\end{scope}


% centermark
%VWF_FIGA	
%	\begin{scope}  [shift={(C)}]	
%		\draw (-0.1, 0.0) -- (0.1, 0.0);
%		\draw (0.0, -0.1) -- (0.0, 0.1);
%	\end{scope}	
%
	% axes
% VWF_FIGB
%	\draw[-{Latex[scale=1.5]}] (-0.5,0) -- (0.75,0)  node [right] {$x$};
%	\draw[-{Latex[scale=1.5]}] (0,-0.5) -- (0,0.5)  node [above] {$y$};


%	\begin{scope} [shift={(B)}]
%		\draw circle (0.3);
%	\end{scope}

%	\begin{scope} [shift={(A0)}]
%		\draw[-{Latex[scale=1.5]}] (-0.5,0) -- (0.75,0)  node [right] {$x$};
%		\draw circle (0.1);
%	\end{scope}

%	\begin{scope} [shift={(A1)}]
%		\draw circle (0.1);
%	\end{scope}
	
%	\begin{scope} [shift={(A2)}]
%		\draw circle (0.1);
%	\end{scope}

%	\begin{scope} [shift={(A3)}]
%		\draw circle (0.1);
%	\end{scope}

%	\begin{scope} [shift={(A4)}]
%		\draw circle (0.1);
%	\end{scope}

%	\begin{scope} [shift={(A5)}]
%		\draw circle (0.1);
%	\end{scope}

%	\begin{scope} [shift={(A6)}]
%		\draw circle (0.1);
%	\end{scope}
	
%	\begin{scope} [shift={(A7)}]
%		\draw circle (0.1);
%	\end{scope}
	
%	\begin{scope} [shift={(A8)}]
%		\draw circle (0.1);
%	\end{scope}
	
%	\begin{scope} [shift={(A9)}]
%		\draw circle (0.1);
%	\end{scope}	
	

%	\draw[dotted] (t1n) -- (t1);
%	\draw[dotted] (t2n) -- (t2);

%	\draw (t1) -- (t1s);
%	\draw (t2) -- (t2s);
	
% VWF_FigA	
%	\draw (t1) arc[start angle=120, end angle=240, radius=0.5];
% VWF_FIGB
 %	\draw [dashed] (t1) arc[start angle=120, end angle=-120, radius=0.5];
%	\draw (t1) arc[start angle=120, end angle=-120, radius=0.5];
%	

% VWF_FigA	
%	 \draw[-{Latex}] (C) -- (t1);
%

% VWF_FIGB
	% \draw (C) ellipse (0.5);
%	 \draw (C) -- (t1);
%
	
	
% VWF_FIGB
%	\begin{scope}  [shift={(t1)}]
%		\draw (210:0.1) -- ++(-60:0.1); 
%		\draw (-60:0.1) -- ++(210:0.1); 
%	\end{scope}	
	

	% point I
%	\begin{scope} [shift={(I)}]
%		\draw [fill=black] circle (0.02);
%	\end{scope}

	% point C
% VWF_FIGB	
%	\begin{scope} [shift={(C)}]
%		\draw [fill=black] circle (0.02);
%	\end{scope}
	
	% point P
% VWF_FIGB
%	\begin{scope} [shift={(t1)}]
%		\draw [fill=black] circle (0.02);
%	\end{scope}

% VWF_FIGB
%	\begin{scope} [shift={(I)}]
%		\draw (0.25,0) arc[start angle=0, end angle=30, radius=0.25];
%	\end{scope}


%	\begin{scope} [shift={(I)}]
%		\draw (150:0.25) arc[start angle=150, end angle=210, radius=0.25];
%	\end{scope}
	
	
	%\node [label={Hello}] ;
%	\node [label={[shift={(0.1,-0.25)}]$C$}] at (C) {};
%	\node [label={[shift={(-0.2,-0.1)}]$L_1$}] at (t1s) {};
%	\node [label={[shift={(-0.2,0.1)}]$L_2$}] at (t2s) {};
%	\node [label={[shift={(-0.05,0.1)}]$r$}] at (C) {};
%	\node at (165:0.35) {$\theta$};
%	\node [label={[shift={(0.05,-0.28)}]$I$}] at (I) {};
	
% VWF_FIGB	
%	\node [label={[shift={(0.0,-0.04)}]$P$}] at (t1) {};
%	\node [label={[shift={(0.65,-0.25)}]$\Delta x$}] at (I) {};	
%	\node [label={[shift={(0.33,0.15)}]$d$}] at (I) {};	
%	\node at (15:0.35) {$\alpha$};
%


\end{tikzpicture}
\caption{Arrangement of Inner Tangents}
\label{fig:fig_c}
\end{figure}

When $I$ lays between two circles, the arrangement is as if circle $A$ were reflected about
point $I$ to a new circle $A'$.  Let the center location of $A'$ be named $C_{A'}$. Meanwhile radius of the reflection, $A'$,
remains equal to $r_A$, the radius of $A$.

%%%%%%%
\begin{table*}[ht]
\centering
\begin{tabular}{|c||c|c|c|c|c|c|} 
 \hline
  \multirow{3}{*}{Case} & \multicolumn{5}{c|}{Number of Tangents} & \multirow{3}{*}{Description} \\
\cline{2-6}
   & \makecell{(A)\\Null}  & \makecell{(B)\\Single} & \makecell{(C)\\Outer} & \makecell{(D)\\Inner} & Total & \\
  \hline
  0-3 & 0 & - & - & - & 0 & No tangents possible\\ 
  \hline
  4 & - & 1 & - & - & 1 & Circles tangent to each other\\ 
  \hline
  5-7 & - & - & 2 & - & 2 & Two outer tangents\\ 
  \hline
  8 & - & 1 & 2 & - & 3 & Circles tangent to each other, and 2 outer tangents\\ 
  \hline
  9 & - & - & 2 & 2 & 4 & Two outer and two inner tangents\\ 
  \hline
\end{tabular}
\caption{Classification of Coplanar Circle Cases}
\label{table:tab_tangents}
\end{table*}
%%%%%%%

The location $C_{A'}$ is simply the interval between $C_{A'}$ and $I$ added to the location of $I$, as follows:
\begin{align}
	x_{A'} &= (x_I - x_A) + x_I \\
	x_{A'} &= 2x_I - x_A \label{eq:xaprime}
\end{align}

Given two circles, the strategy of finding $I$ for a pair of inner tangents is as follows.  
\begin{enumerate}
\item Let $B$ be the larger circle.
\item Let $A'$ be the smaller circle.  
\item Assume $A'$ has a reflection thru $I$ named $A$.
\item Use Equation \ref{eq:xaprime} to find the location of reflection $A$.
\item Substitute the location of $A$ into Equation \ref{eq:xI}.
\item Rearrange terms and solve for $x_I$.
\end{enumerate}

Solving Equation \ref{eq:xaprime} for $x_A$ yields:

\begin{equation}
\boxed
{
	x_A = 2x_I - x_{A'} \label{eq:xA}
}
\end{equation}

Substituting Equation \ref{eq:xA} into Equation \ref{eq:xI} yields:
\begin{align*}
	x_I &= \frac{r_B \cdot (2 x_I - x_{A'})} {r_B - r_A} \\
	x_I &= \frac{2 r_B \cdot x_I -  r_B \cdot x_{A'}  }{r_B - r_A}
\end{align*}
\begin{equation*}
	x_I \cdot (r_B - r_A) = 2 r_B \cdot x_I - r_B \cdot x_{A'} 
\end{equation*}
\begin{align*}
	x_I \cdot r_B - x_I \cdot r_A - 2  r_B \cdot x_I &=  - r_B \cdot x_{A'}\\
	x_I \cdot (r_B - r_A - 2  r_B ) &= - r_B \cdot x_{A'}\\
	x_I \cdot (-r_B - r_A) &= - r_B \cdot x_{A'} 
\end{align*}

Finally $x_I$ is found to be:
\begin{equation}
\boxed
{
	x_I = \frac{r_B \cdot x_{A'}}{ r_B + r_A} \label{eq:xI2}
}
\end{equation}

Substituting Equation \ref{eq:xI2} into Equation \ref{eq:pointI} gives the location of $I$.  Because the location
of point $I$ has been found, Equation \ref{eq:thetaarcsin} can be used to find the angle $\theta$.  Equations 7 and 8 from 
%article\footnotemark[\ref{noteA002}] can be used to find the tangent points.

%Case 9 from article\footnotemark[\ref{noteA003}] is an example of this class.

%%%
\paragraph{Classification\\}
Now that four tangent classes have been defined, classifying the ten coplanar circle-pairs (identified by 
%article\footnotemark[\ref{noteA003}])
is desired.  Table \ref{table:tab_tangents} summarizes which tangent classes apply to each of the ten circle-pair cases.

Either one or two tangent classes many apply to a circle-pair case.  Generally, the number of total tangents increases
as a circle $A$ retreats from circle $B$. Depending upon which tangent classes apply, each case may have a total of zero 
to four possible tangents.  

The classification of coplanar circle-pairs described here provides a convenient framework for
future planar-geometry discussion.

\begin{flushright}   
Q.E.D.
\end{flushright}