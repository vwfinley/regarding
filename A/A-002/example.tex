\section{Example}

\paragraph{Given:}
\begin{itemize}
\item A fillet whose radius is 3.5 units,
\item \emph{and} whose legs intersect at a $45\degree$ included angle. 
\end{itemize}

\paragraph{Find:}
\begin{itemize}
\item The center of the fillet relative to the point where the fillet legs intersect,
\item \emph{and} the location where the fillet legs are tangent to the fillet arc,
\item \emph{and} the point on the fillet nearest to where the legs intersect.
\end{itemize}

\paragraph{Steps:}
\begin{enumerate}
\item
First, from equation \ref{eqn:alpha}, $\alpha$ is found to be $22.5\degree$

\item
Converting $\alpha$ to radians, and using equation \ref{eqn:deltax}, $\Delta x$ is calculated to be:
\begin{align*}
\Delta x &= \frac{3.5}{\sin{0.3927}} \\
\Delta x &= 9.1459
\end{align*}

\item
Converting $\alpha$ to radians, and applying equation \ref{eqn:d}, $d$ is found:
\begin{align*}
d &=  \frac{3.5}{\tan{0.3927}} \\
d &= 8.4497
\end{align*}

\item
Substituting $r$ and $\alpha$ into equation \ref{eqn:Cxy}, the
center of $C$ is located:
\begin{equation*}
C_{x,y} = \left(9.1459, 0 \right)
\end{equation*}

\item
The coordinates where $L_1$ and $L_2$ are tangent to the curved part of the fillet are discovered
by substituting into equations \ref{eqn:Lpolar} and \ref{eqn:Lcartesian}:
\begin{align*}
P_{d,\alpha} &= (8.4497, \pm22.5\degree) \\
P_{x,y} &= (7.8065, \pm 3.2336) 
\end{align*}

\item
Finally, from \ref{eqn:Ci} the point on $C$ nearest to where the fillet legs intersect is located:
\begin{align*}
C_{I} &= (9.1459 - 3.5, 0)  \\
C_{I} &= (5.6459, 0)  \\
\end{align*}

\end{enumerate}