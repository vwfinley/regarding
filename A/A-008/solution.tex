\section{Solution}
Two solutions are presented here.  Both are useful and address different design needs.

When drafting a design, either manually or with CAD software, graphical solutions save time.  
The convenient \emph{geometric construction} described below will speed progress.

When creating: programs, scripts, spreadsheets or tabular data;  equations are preferred.
The \emph{trigonometric calculation} detailed below will eliminate any guesswork.

\subsection{Geometric Construction}
It is desirable to find a simple geometric construction to locate the problem statement's hypothetical circle.

Neither the hypothetical circle's center nor radius is known.
This makes locating it difficult, as knowing one can usually be used to find the other, with simple drafting tools: compass, ruler and square.

Refer to figure \ref{fig:fig_a} for the following discussion.  Let's define points of interest:
\begin{itemize}
\item Let point $A$ be the unknown center of the hypothetical circle that lies somewhere on sloped line $L_s$.
\item Let point $B$ an alias for the the origin.  It lies on the circle since the circle passes through it.  Furthermore it also lies on sloped line $L_s$ since $L_s$ passes through the origin.
\item Let point $C$ be a location on the circle where it is tangent to line $L_h$.  In other words, $C$ is the tangent point.
\item Let point $D$ be the location where line $\overline{AC}$ intersects the $x$-axis at right angles.
\item Let point $E$ be the location where line $L_h$ intersects the $y$-axis at right angles.
\end{itemize}

%It is identical to that found in stage 3 of figure 1 in the previous article\footnote{\label{noteA006}A-006, Regarding Tangent Ogives, Vincent W. Finley, September 2022}.

Any point that is distance $r$, from a point $A$, lies on the circle of radius $r$ centered at $A$. 
Therefore to be on the circle of radius $r$ centered at $A$, points $B$ and $C$ must be a distance $r$ from $A$.  
This implies that the line from $A$ to $B$ is of length $r$.  
Also implied is the line from $A$ to $C$ is of length $r$, and it is equal to the length of the line from $A$ to $B$, as follows:
\begin{equation}
\overline{AB} = \overline{AC} = r
\end{equation}

Since $\overline{AB} = \overline{AC}$ two sides of the hypothetical triangle $\triangle{ABC}$ are equal; therfore $\triangle{ABC}$ an isosceles triangle.

The radius and center of $\triangle{ABC}$ is still unknown.  The points $A$, $B$ and $C$ are also still unknown.  To find them some algebra will be applied to triangles.

Let $\alpha$ be the angle $L_s$ makes with the negative $x$-axis.  
Also, let $\beta$ be the angle complimentary to $\alpha$ such that:
\begin{align}
\alpha &= \angle{ABD} \label{eq:abd}\\
\beta &= 90^{\circ} - \alpha  \label{eq:beta}
\end{align}

Now $\beta$ forms an alternate interior angle with $\angle{CAB}$ so that:
\begin{equation}
\angle{CAB} = \beta
\end{equation}

Since $\triangle{ABC}$ is isosceles the other two angles, $\angle{ACB}$ and $\angle{ABC}$, must be equal.
Name these other angles $\gamma$ so that.
\begin{equation}
\angle{ACB} = \angle{ABC}  = \gamma = \frac{180^{\circ} - \beta}{2}  \label{eq:gamma}
\end{equation}

Substituting equation \ref{eq:beta} into equation \ref{eq:gamma} it is found:
\begin{equation}
\gamma = \frac{90^{\circ} + \alpha}{2} \label{eq:gamma2}
\end{equation}

By inspecting equation \ref{eq:gamma2} it can be seen that angle $\gamma$ on the $\triangle{ABC}$ is half of some angle
whose value is $90^{\circ} + \alpha$.  Let's call the angle $\delta$ so that:
\begin{equation}
\delta = 90^{\circ} + \alpha
\end{equation}
and,
\begin{equation}
\gamma = \frac{\delta}{2}
\end{equation}

$\gamma$ on the triangle is just half some angle $\delta = 90^{\circ} + \alpha$, but:
\begin{equation}
\delta = 90^{\circ} + \alpha = \angle{ABE}
\end{equation}
and, it is readily observed angle $\angle{ABE}$ is just the angle between sloped line $L_s$ and the $y$-axis.

\noindent\fbox{%
    \parbox{\linewidth}{%
\textbf{Therefore, geometric construction procedure to find circle center and radius is:}
\begin{enumerate}
\item Bisect angle $\angle{ABE}$ (see figure \ref{fig:fig_a} dashed line),
\item \emph{next}, extend the bisection line from $B$, at the origin, and intersect line $L_h$,
\item \emph{next}, label the intersection of the bisection line and $L_h$ as point $C$, ($C$ is the tangent point)
\item \emph{next}, draw a vertical line down from $C$, 
\item \emph{next}, label the intersection of the vertical line with $L_s$ as point $A$, ($A$ is the circle center)
\item \emph{next}, the value of radius $r$ can be found by measuring length of either $\overline{AB}$ or $\overline{AC}$,
\item \emph{next}, verfify $\overline{AB} = \overline{AC} = r$.  They should be equal, otherwise an error occurred while applying the construction procedure.
\end{enumerate}
	}%
}

\subsection{Trigonometric Calculation}
Trigonometry makes discovery of the circle radius and center a trivial exercise.

Figure \ref{fig:fig_a} shows radius $r$ as the hypotonuese of $\triangle{ABD}$.
Recall from equation \ref{eq:abd} the value of angle of $\angle{ABD}$ is $\alpha$.

The sine of $\alpha$ is the ratio of some unknown length $i$, shown on the figure, over radius $r$.
However $i$ is just the vertical radius distance $r$ less the distance $h$, like so $i = r - h$
\begin{align*}
\sin{\alpha} &= \frac{i}{r} \\
i &= r \cdot \sin{\alpha} \\
r - h &= r \cdot \sin{\alpha} \\
r - r \cdot \sin{\alpha} &= h \\
r(1 - \sin{\alpha}) &= h 
\end{align*}
\begin{equation}
\boxed{
r = \frac{h}{ 1 - \sin{\alpha}}
}
\end{equation}



\subsection{Notes on general application}


\begin{flushright}   
Q.E.D.
\end{flushright}