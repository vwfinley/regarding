\section{Solution}
Two solutions are presented here.  Both are useful and address different design needs.

When drafting a design, either manually or with CAD software, graphical solutions save time.  
The convenient \emph{geometric construction} described below will speed progress.

When creating: programs, scripts, spreadsheets or tabular data;  equations are preferred.
The \emph{trigonometric calculation} detailed below will eliminate any guesswork.

\subsection{Geometric Construction}
It is desirable to find a simple geometric construction to locate the problem statement's hypothetical circle.

Neither the hypothetical circle's center nor radius is known.
This makes locating it difficult, as knowing one can usually be used to find the other, with simple drafting tools: compass, ruler and square.

Refer to figure \ref{fig:fig_a} for the following discussion.  Let's define points of interest:
\begin{itemize}
\item Let point $A$ be the unknown center of the hypothetical circle that lies somewhere on sloped line $L_s$.
\item Let point $B$ an alias for the the origin.  It lies on the circle since the circle passes through it.  Furthermore it also lies on sloped line $L_s$ since $L_s$ passes through the origin.
\item Let point $C$ be a location on the circle where it is tangent to line $L_h$.  In other words, $C$ is the tangent point.
\item Let point $D$ be the location where line $\overline{AC}$ intersects the $x$-axis at right angles.
\item Let point $E$ be the location where line $L_h$ intersects the $y$-axis at right angles.
\end{itemize}

%It is identical to that found in stage 3 of figure 1 in the previous article\footnote{\label{noteA006}A-006, Regarding Tangent Ogives, Vincent W. Finley, September 2022}.

Any point that is distance $r$, from a point $A$, lies on the circle of radius $r$ centered at $A$. 
Therefore to be on the circle of radius $r$ centered at $A$, points $B$ and $C$ must be a distance $r$ from $A$.  
This implies that the line from $A$ to $B$ is of length $r$.  
Also implied is the line from $A$ to $C$ is of length $r$, and it is equal to the length of the line from $A$ to $B$, as follows:
\begin{equation}
\overline{AB} = \overline{AC} = r
\end{equation}

Since $\overline{AB} = \overline{AC}$ two sides of the hypothetical triangle $\triangle{ABC}$ are equal; therfore $\triangle{ABC}$ an isosceles triangle.

The radius and center of $\triangle{ABC}$ is still unknown.  The points $A$, $B$ and $C$ are also still unknown.  To find them some algebra will be applied to triangles.

Let $\alpha$ be the angle $L_s$ makes with the negative $x$-axis.  
Also, let $\beta$ be the angle complimentary to $\alpha$ such that:
\begin{align}
\alpha &= \angle{ABD}\\
\beta &= 90^{\circ} - \alpha  \label{eq:beta}
\end{align}

Now $\beta$ forms an alternate interior angle with $\angle{CAB}$ so that:
\begin{equation}
\angle{CAB} = \beta
\end{equation}

Since $\triangle{ABC}$ is isosceles the other two angles, $\angle{ACB}$ and $\angle{ABC}$, must be equal.
Name these other angles $\gamma$ so that.
\begin{equation}
\angle{ACB} = \angle{ABC}  = \gamma = \frac{180^{\circ} - \beta}{2}  \label{eq:gamma}
\end{equation}

Substituting equation \ref{eq:beta} into equation \ref{eq:gamma} it is found:
\begin{equation}
\gamma = \frac{90^{\circ} + \alpha}{2} \label{eq:gamma2}
\end{equation}

By inspecting equation \ref{eq:gamma2} it can be seen that angle $\gamma$ on the $\triangle{ABC}$ is half of some angle
whose value is $90^{\circ} + \alpha$.  Let's call the angle $\delta$ so that:
\begin{equation}
\delta = 90^{\circ} + \alpha
\end{equation}
and,
\begin{equation}
\gamma = \frac{\delta}{2}
\end{equation}

$\gamma$ on the triangle is just half some angle $\delta = 90^{\circ} + \alpha$, but:
\begin{equation}
\delta = 90^{\circ} + \alpha = \angle{ABE}
\end{equation}
and, it is readily observed angle $\angle{ABE}$ is just the angle between sloped line $L_s$ and the $y$-axis.

\noindent\fbox{%
    \parbox{\linewidth}{%
\textbf{Therefore, geometric construction procedure to find circle center and radius is:}
\begin{enumerate}
\item Bisect angle $\angle{ABE}$ (see figure \ref{fig:fig_a} dashed line),
\item \emph{next}, extend the bisection line from $B$, at the origin, and intersect line $L_h$,
\item \emph{next}, label the intersection of the bisection line and $L_h$ as point $C$, ($C$ is the tangent point)
\item \emph{next}, draw a vertical line down from $C$, 
\item \emph{next}, label the intersection of the vertical line with $L_s$ as point $A$, ($A$ is the circle center)
\item \emph{next}, the value of radius $r$ can be found by measuring length of either $\overline{AB}$ or $\overline{AC}$,
\item \emph{next}, verfify $\overline{AB} = \overline{AC} = r$.  They should be equal, otherwise an error occurred while applying the construction procedure.
\end{enumerate}
	}%
}

\subsection{Trigonometric Calculation}
Figure 2 of article A-005, shows cases where equal coplanar circles will have regions of intersection.
These are cases 5 through 7 specifically, with cases 1 and 2 being redundant.
It is worth noting, for all cases resulting in regions of intersection, only a pair of Class C outer tangent lines is possible.

Recall the conditions for cases 5 through 7:
\begin{itemize}
\item Case 5: $0 < x_A < r_A$
\item Case 6: $x_A = r_A$
\item Case 7: $r_A < x_A < 2 \cdot r_A$
\end{itemize}

Combining the three conditions yields a general condition, which equal circle pairs must satisfy, to have a region of intersection:
\begin{equation}
\boxed{
0 < x_A  < 2 \cdot r_A
}
\end{equation}

Since circles $A$ and $B$ are equal, their radii $r_A = r_B = r$ are equal. Furthermore, the value $x_A$ is simply the
distance between the centers of circles $C_A$ and $C_B$ as described by A-005.
In that article $C_B$ was fixed at the origin and $C_A$ was permitted to vary some distance $x_A$ along the non-negative $x$-axis.

In Stage 3 on Figure 1 from article A-006, the two circles are centered about points $P'$ and $P''$.
These center points lie along the x-axis some distance $a$ from the origin.
This is confirmed by equations 8a and 9a in A-006, where these are given as $P'=(-a,0)$ and $P''=(a,0)$ respectively.

Therefore the relationship between variable $x_A$ in A-005 and variable $a$ in A-006 can be described as:
\begin{equation}
x_A = 2 \cdot a
\end{equation}
where $x_A$ is the distance between centers, and:
\begin{equation}
\boxed{
a = \frac{x_A}{2}\label{eq:a}
}
\end{equation}

This relationship bridges articles A-005 and A-006 together.
Bridging of these articles is expected since both merely deal with overlapping equal coplanar circles whose centers are separated by some distance. 

To fortify the bridge between articles, substitute equation \ref{eq:a} above into the ogive equations: 2, 3 and 4; from A-006.
The substitution yields the respective ogive equation in terms of circle separation distance, as follows:
\begin{equation}
\boxed{
r = \frac{x_A}{2} + b
}
\end{equation}
\begin{equation}
\boxed{
r = \sqrt{\left(\frac{x_A}{2}\right)^2+c^2}
}
\end{equation}
\begin{equation}
\boxed{
x_A = \frac{c^2 - b^2}{b}
}
\end{equation}
Note: equations: 6 and 7; from A-006 are unaffected since they are independent of $a$, and are therefore independent of $x_A$.

\subsection{Finding area of ogive cross-section} 
The area of the ogive cross-section is found by doubling the area under the circle between points $P_a$ and $P_b$ in figure \ref{fig:fig_a}.

By remembering figure \ref{fig:fig_a} here is just Stage 3 from Figure 1 in article A-006, one can work backward to Stage 1.
Referring to the half-ogive shown in Stage 1 makes the following integration more apparent.

From a table of integrals\footnote{ISBN 0-471-85045-4, Calculus, 3rd Edition, Howard Anton, Drexel University, Copyright 1988 Anton Textbooks, Inc., John Wiley \& Sons, See integral \#40 inside front cover.} the 
definite integral: with respect to $x$, of the circle with having radius $r$, over limits $a$ and $r$; is found to be:
\begin{align*}
\left. F(x) \right]^r_a &= \int^r_a \sqrt{r^2 - x^2}\,dx \\
&=\left. \frac{x}{2}\sqrt{r^2 - x^2} + \frac{r^2}{2}sin^{-1}\left(\frac{x}{r}\right)\right]^r_a \\
&=\frac{r}{2}\sqrt{r^2 - r^2} + \frac{r^2}{2}sin^{-1}\left(\frac{r}{r}\right) \\ 
&\phantom{{}=1}- \frac{a}{2}\sqrt{r^2 - a^2} - \frac{r^2}{2}sin^{-1}\left(\frac{a}{r}\right) \\
&= \frac{\pi r^2}{4} - \frac{a}{2}\sqrt{r^2 - a^2} - \frac{r^2}{2}sin^{-1}\left(\frac{a}{r}\right)
\end{align*}
The area $A_{to}$ of the tangent ogive cross-section is twice the area of definite integral.
\begin{equation*}
A_{to} = 2 \left[\frac{\pi r^2}{4} - \frac{a}{2}\sqrt{r^2 - a^2} - \frac{r^2}{2}sin^{-1}\left(\frac{a}{r}\right)\right]
\end{equation*}
\begin{equation}
A_{to} = \frac{\pi r^2}{2} - a \sqrt{r^2 - a^2} - r^2 sin^{-1}\left(\frac{a}{r}\right)
\end{equation}
Recognizing $\sqrt{r^2 - a^2}=c$, from A-006 equation 3, yields $A_{to}$ in a more convenient form.
\begin{equation}
\boxed{
A_{to} = \frac{\pi r^2}{2} - ac - r^2 sin^{-1}\left(\frac{a}{r}\right) \label{eq:ato}
}
\end{equation}

\subsection{Finding volume of tangent ogive} 
Starting with the standard form of the equation of a circle\footnote{ISBN 0-471-85045-4, Anton, page 49, equation \#1.}, substitute
in $P'=(-a, 0)$ in place of $(x_0, y_0)$.
Doing so will translate the circle center to $P'$ as shown in figure \ref{fig:fig_a} and in A-006, Figure 1, Stage 2.
Solve the equation for $x$, thereby yielding a function of $y$.
\begin{align*}
r^2 &= (x-x_0)^2 + (y-y_0)^2 \\
r^2 &= (x-(-a))^2 + (y-0)^2\\
r^2 &= (x+a)^2 + y^2\\
(x+a)^2 &= r^2 - y^2 \\
 x+a &= \sqrt{r^2 - y^2}
\end{align*}
\begin{equation}
u(y) = x = -a + \sqrt{r^2 - y^2} \label{eq:uy}
\end{equation}
Use the ``Volumes by Disks Perpendicular to
the y-Axis''\footnote{ISBN 0-471-85045-4, Anton, page 380, equation \#7.}
method to integrate equation \ref{eq:uy}.
\begin{equation}
V = \int^d_c\pi[u(y)]^2\,dy \label{eq:V}
\end{equation}
Substituting equation \ref{eq:uy} into \ref{eq:V}, integrating from $y=0$ to $y=c$ (A-006, Figure 1, Stage 2), and recognizing $\sqrt{r^2 - c^2} = a$ (A-006, Equation 3); yields:
\begin{align*}
V_{to} &= \int^c_0\pi\left[-a + \sqrt{r^2 - y^2}\right]^2\,dy\\
&= \pi \int^c_0\left[a^2 - 2a\sqrt{r^2 - y^2} + r^2 - y^2\right]\,dy\\
&= \pi \left[a^2y - ay\sqrt{r^2 - y^2} - ar^2 sin^{-1}\left(\frac{y}{r}\right)+ r^2y - \frac{y^3}{3}\right]^c_0\\
&= \pi \left[a^2c - ac\sqrt{r^2 - c^2} - ar^2 sin^{-1}\left(\frac{c}{r}\right)+ r^2c - \frac{c^3}{3}\right]\\
&= \pi \left[a^2c - a^2c - ar^2 sin^{-1}\left(\frac{c}{r}\right)+ r^2c - \frac{c^3}{3}\right]
\end{align*}
Further reduction yields the volume $V_{to}$ of the tangent ogive in terms of $a$, $c$ and $r$:
\begin{equation}
\boxed{
V_{to} = \pi \left[r^2c - \frac{c^3}{3} - ar^2 sin^{-1}\left(\frac{c}{r}\right)\right] \label{eq:vto_acr}
}
\end{equation}
Substituting in $c = \sqrt{r^2 - a^2}$ and multiplying through by $\pi$; gives the ogive volume $V$ only in terms of distance $a$ and radius $r$.
\begin{equation} \label{eq:vto_ar}
\begin{split}
V_{to} = \pi r^2\sqrt{r^2 - a^2} - \frac{\pi (r^2 - a^2)^{3/2}}{3} \\
- \pi ar^2 sin^{-1}\left(\frac{\sqrt{r^2 - a^2}}{r}\right)
\end{split}
\end{equation}
However, describing $V_{to}$ in terms of the two variables in equation \ref{eq:vto_ar} is computationally intensive.  
In practice, first calculating $c$ from $a$ and $r$, then substituting the value of $c$ into equation \ref{eq:vto_acr} requires less redundant computation.

\subsection{Area of circle intersection}
Upon inspecting figure \ref{fig:fig_a} one can see similarities between intersecting circles and Venn diagrams.
While Venn diagrams concern themselves with set membership, figure \ref{fig:fig_a} is concerned with regions on a plane.
However, if one considers all points in some bounded region to be members of an infinite point set for that region, then figure \ref{fig:fig_a} becomes a Venn diagram representing point membership.
Thinking of figure \ref{fig:fig_a} in this manner relates: tangent ogive area and intersectional circle area; to set theory and set operations.

Start by realizing the intersectional circle area $A_{ci}$ of two equal coplanar circles is simply twice the area of the tangent ogive cross-section.
This is due to the reflection of the tangent ogive cross-section about the x-axis.
\begin{equation}
\boxed{
A_{ci} = 2 \cdot A_{to} \label{eq:aci}
}
\end{equation}
Now let $Q$ and $R$ be two equal overlapping circles, such that the area of each circle is:
\begin{equation*}
A_Q = A_R = \pi r^2
\end{equation*}
Set operators could be defined as operations upon areas of $Q$ and $R$.   For example, define the: intersection, union, set difference and symmetric difference; of $Q$ and $R$ as:
\begin{subequations}
\begin{align}
\begin{split}
Q \cap R &= A_{ci}
\end{split}\\
\begin{split}
Q \cup R &= A_Q + A_R - A_{ci} \\
  &= \pi r^2 + \pi r^2 - A_{ci} \\
  &= 2\pi r^2 - A_{ci}
\end{split}\\
\begin{split}
Q - R &= A_Q - A_{ci} \\
  &= \pi r^2 - A_{ci}
\end{split}\\
\begin{split}
Q \ominus R &= A_Q + A_R - 2 \cdot A_{ci} \\
  &= \pi r^2 + \pi r^2 - 2 \cdot A_{ci} \\
  &= 2\pi r^2 - 2 \cdot A_{ci}
\end{split}
\end{align}
\label{eq:setops}
\end{subequations}

A complete collection of set operators could be defined in a similar manner.
Mapping set operators to Boolean operators could also be useful.

\subsection{Volume of circle intersection}
The volume, $V_{ci}$, of the circle intersection area revolved around the y-axis is simply twice the volume of the solid tangent ogive.
This is due to the fact that the revolved circle intersection is just the solid tangent ogive and its reflection about the x-axis.
\begin{equation}
\boxed{
V_{ci} = 2 \cdot V_{to} \label{eq:vci}
}
\end{equation}
Note: $V_{ci}$ is not the same as the volume of the intersection of 2 equal spheres.
\begin{flushright}   
Q.E.D.
\end{flushright}