\section{Example}

\paragraph{Given:}
\begin{itemize}
\item Line $L_s$, with slope $m = 0.268$, passing through the origin,
\item \emph{and} horizontal line $L_h$ passing distance $h = 13 \: units$ above the origin. 
\end{itemize}

\paragraph{Find:}
\begin{itemize}
\item The circle that:
\begin{itemize}
\item passes through the origin,
\item and is tangent to horizontal line $L_h$,
\item and has its center somewhere on sloped line $L_s$.
\end{itemize}
\item A geometric construction locating the circle on $L_s$, tangent to $L_h$, and through the origin.
\item The radius of the circle.
\item Numeric values for all angles and critical points.
\end{itemize}

\begin{figure}[ht!]
\centering

\begin{tikzpicture}[x=1in, y=1in]
	\begin{scope}[xscale=.10, yscale=.10]	

		\clip (-24, -10) rectangle (7, 17);

		% axes
		\draw[lightgray, -{Latex[scale=1.5]}] (-40.0, 0) -- (5.0, 0)  node [right] {$x$};		
		\draw[lightgray, -{Latex[scale=1.5]}] (0, -30.0) -- (0, 15.0)  node [above] {$y$};	

		% vertical
		\draw[lightgray] (-16.942, -35) -- (-16.942, 35); % left

		% horizontal
		\draw[lightgray] (-50, 13.0) -- (90, 13.0) node[shift={(179.4 : 8.5)}]{$L_h$};    % node[shift={(179.3 : 8.2)}]{$L_h$}; % L_H
%		\draw[lightgray] (-19.0097, -4.0024) -- (-9.0097, -4.0024);	 % A center mark

%		\draw[lightgray] (-16.0097, -0.75) -- ++(0  : 0.75) -- ++(90 : 0.75) ;	 % A center mark

		% sloped
		\draw[lightgray] (195 : 100) -- (15 : 100) node[shift={(-165.6 : 9.45)}]{$L_s$}; % L_S

		% circle
		\draw[lightgray] ([shift=(135 :  21.355)] 0, 0)  arc (135 : 150: 21.355); % mark
		
		% bisector
		\draw[lightgray, dashed] (142.50 : -3) -- (142.50 : 25); % dashed line

		% angle labels
%		\draw[black] ([shift=(180 : 5)] 0, 0)  arc (180 : 194.04 : 5) node[shift={(-0.10, 0.05)}]{$\alpha$} ; % alpha
%		\draw[black] ([shift=(194.04 : 2)] 0, 0)  arc (194.04 : 270 : 2) node[shift={(-0.17, -0.05)}]{$\beta$}; % beta
%		\draw[black] ([shift=(142.02 : 4)] 0, 0)  arc (142.02 : 194.04 : 4) node[shift={(-0.05, 0.23)}]{$\gamma$}; % gamma
%		\draw[black] ([shift=(194.04 : 3)] 0, 0)  arc (194.04 : 90 : 3) node[shift={(-0.21, 0.003)}]{$\delta$}; % delta

%		\draw[black] ([shift=(0 : 5)] 194.04 : 16.5024)  arc (0 : 14.04 : 5) node[shift={(0.09, -0.05)}]{$\alpha$}; % alpha
%		\draw[black] ([shift=(14.04 : 2)] 194.04 : 16.5024)  arc (14.04 : 90 : 2) node[shift={(0.17, 0.05)}]{$\beta$}; % beta		
%		\draw[black] ([shift=(142.02 : 20.3116)] 270 : 4)  arc (270 : 322.02 : 4) node[shift={(-0.09, -0.19)}]{$\gamma$}; % gamma

		% dimensions
%		\draw[lightgray, latex - latex] (0 : -22.5) -- +(90 : 12.5) node[shift={(-84 : 0.65)}]{$h$};
%		\draw[lightgray, latex - latex] (0 : -22.5) -- +(-90 : 4.0024) node[shift={(75 : 0.24)}]{$i$};
%		\draw[lightgray] (-23.5, -4.0024) -- (-21.5, -4.0024);	 % A center mark

		% locate arc
		\draw[black, thick] (-16.942, -4.54) circle (17.54); % The arc
		
		% radius lines
%		\draw[black, -latex] (194.04 : 16.5024) -- +(90 : 16.5024) node[shift={(-94 : 0.9)}]{$r$};
%		\draw[black, -latex] (194.04 : 16.5024) -- +(14.04 : 16.5024) node[shift={(200 : 0.75)}]{$r$};




		% point labels
		\draw[black] (192 : 18.2)  node{$A$};
		\draw[black] (50 : 1.4)  node{$B$};
		\draw[black] (140.2 : 23)  node{$C$};
%		\draw[black] (183 : 16.8)  node{$D$};
		\draw[black] (85 : 12.1)  node{$E$};

	\end{scope}
		





\end{tikzpicture}
\caption{Example Circle}
\label{fig:fig_b}
\end{figure}

\paragraph{Steps:}
\begin{enumerate}
\item
Follow the ``Geometric Construction Procedure'' from the end of subsection \ref{gConst}.
Then, on figure \ref{fig:fig_b}:
\begin{itemize}
\item bisect $\angle{ABE}$,
\item label the intersection of the bisection line and $L_h$,
\item draw a vertical line downward to intersect with $L_s$,
\item draw a circle: centered at $A$ the intersection of the vertical line and $L_s$, through point $B$ the origin, and through $C$ the intersection of the bisection line with $L_h$.
\end{itemize}

\item
To get the angle $\alpha$ from slope $m$:
\begin{align}
\alpha &= \tan^{-1} \left( m \right) \label{eq:m_alpha}\\
\alpha &= \tan^{-1} \left( 0.268 \right) \nonumber \\
\alpha &= 15^\circ   \nonumber
\end{align}

\item
From the given values, $m$ and $h$, all other values can be calculated using the numbered equations as follows:
\begin{center}
\begin{tabular}{ |c|c|c| } 
\hline
Variable & Value & Equation \\ [0.5ex] 
\hline
$m$ & 0.268 & \\ 
$h$ & 13.0 &  \\
\hline
$\alpha$ & $15^\circ$ & \ref{eq:m_alpha}\\ 
$r$ & 17.540 & \ref{eq:eq_r}\\ 
$x_A$ & -16.942 & \ref{eq:eq_xa} \\
\hline
$A$ & (-16.942, -4.540) & \ref{eq:eq_A}\\ 
$B$ & (0.0, 0.0) & \ref{eq:eq_B}\\ 
$C$ & (-16.942, 13.0) & \ref{eq:eq_C}\\ 
$D$ & (-16.942, 0.0) & \ref{eq:eq_D}\\ 
$E$ & (0.0, 13.0) & \ref{eq:eq_E}\\
\hline
$\beta$ & $75.0^\circ$ & \ref{eq:beta}\\ 
$\gamma$ & $52.5^\circ$ & \ref{eq:gamma3}\\ 
$\delta$ & $105.0^\circ$ & \ref{eq:eq_delta}\\
\hline
\end{tabular}
\end{center}
\end{enumerate}





