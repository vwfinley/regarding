\section{Solution}

%% List of Steps

\begin{enumerate}
\item
Draw axes, placing the origin at center of $C$, such that point $C=(0,0)$.
\item
Project point $P$ vertically downwards to a point $I$ on the x-axis.  Points: $C$, $I$ and $P$; form a right-triangle $CIP$
having sides: $\Delta x$, $\Delta y$ and $r$.

\begin{figure}[h]
\centering
\begin{tikzpicture}[x=1in,y=1in]
	\coordinate (origin) at (0,0);

	\def \axisLen{0.3};
	\def \L{0.75};
	\def \D{0.75};
	\def \h{1.0}
	\def \cornerlen{0.1};

	\coordinate (xaxisStart) at (-\axisLen, 0);
	\coordinate (yaxisStart) at (0, -\axisLen);

	\coordinate (p1) at (-0.5 * \L, 0.5 * \L);
	\coordinate (p2) at (-0.5 * \L, -0.5 * \L);
	\coordinate (p3) at (0.5 * \L, -0.5 * \L);

	\coordinate (cornertop) at (-0.5 * \L, -0.5 * \L + \cornerlen);

	\coordinate (milltooltop) at (0, \h);
	\coordinate (milltooldim) at (0, \h + 0.1);

	% axes
	\draw[lightgray, {Latex[scale=1.0]}-] (xaxisStart) node [below] {\small $x$} -- (origin) ; % x-axis
	\draw[lightgray, {Latex[scale=1.0]}-] (yaxisStart) node [left] {\small $z$} -- (origin) ; % z-axis	

	% origin
	\draw[fill=black] (origin) circle (0.02);

	% v-groove triangle
	\draw[black] (p1) node[shift={(260 : 0.45)}] {$a$} node[shift={(292 : 0.2)}] {$\theta$} -- (p2) node[shift={(345 : 0.35)}] {$b$} -- (p3) node[shift={(130 : 0.75)}] {$c$} node[shift={(155 : 0.2)}] {$\theta$} -- (p1);

    \draw[black] ([shift=(p1)] 0, -0.1)  arc (270 : 315 : 0.1);
    \draw[black] ([shift=(p3)] -0.1, 0)  arc (180 : 135 : 0.1);

	% corner
	\draw[lightgray] (cornertop) -- ++ (\cornerlen, 0) -- ++ (0, -\cornerlen);

	% milling tool
	\draw[black] (milltooltop) -- ++ (0, -\h) -- ++ (\D, 0) -- ++ (0, \h);

	% milling dimension
	\draw[black, {Latex[scale=0.8]}-{Latex[scale=0.8]}] (milltooldim) node[shift={(15 : 0.4)}] {\small $D$} -- ++ (\D, 0);

	% hdim
	\def \hxstart{-0.5 * \L};
	\def \hystart{-0.5 * \L -0.3};
	\coordinate (hdimstart) at (\hxstart, \hystart);

	\draw[black, {Latex[scale=0.8]}-] (hdimstart) node[shift={(330 : 0.2)}] {\small $L/2$} -- ++ (0.5 * \L, 0);
	%\draw[black] (hdimstart) -- ++ (0.5 * \L, 0);

	\draw[black] (\hxstart, \hystart + 0.05) -- ++ (0, -0.1);
	\draw[black] (0, \hystart + 0.05) -- ++ (0, -0.1);


	% vdim
	\def \vxstart{-0.5 * \L - 0.3};
	\def \vystart{-0.5 * \L};
	\coordinate (vdimstart) at (\vxstart, \vystart);

	\draw[black, {Latex[scale=0.8]}-] (vdimstart) node[shift={(130 : 0.25)}] {\small $L/2$} -- ++ (0, 0.5 * \L);
	%\draw[black] (vdimstart) -- ++ (0.5 * \L, 0);

	\draw[black] (\vxstart + 0.05, \vystart) -- ++ (-0.1, 0);
	\draw[black] (\vxstart + 0.05, \vystart + 0.5 * \L) -- ++ (-0.1, 0);


\end{tikzpicture}
\caption{Point $P$ projected onto x-axis}
\label{fig:fig_b}
\end{figure}

\item
All points on a circle lie a distance $r$ from the center, the distance $r$ being
the radius of the circle.  Furthermore, a point on the circle will lie at some angle $\theta$
from the positive x-axis.  In other words, point $P$ on the circle will be located at polar coordinate $P=(r, \theta)$.
\item
A line tangent to a circle will always intersect radius line $r$ at right
angles.\footnote{ISBN 0-8311-2575-6, Machinery's Handbook, page 46}.
Which means, $L$ and $r$ will intersect at right angles at point $P = (r, \theta)$.
\item
Since $L$ is at an angle $\alpha$ with respect to the x-axis (or parallel to x-axis), \emph{and} $r$ is right
angles to $L$, then $r$ will be at an angle $\alpha$ with respect to the y-axis (or parallel to y-axis).
\item
Aside from having a right angle, triangle $CIP$ has angles $\theta$ and $\alpha$.  The sum of $\theta$ and
$\lvert \alpha \rvert$ must be $\pi/2$.  However since the actual value of $\alpha$ can be negative with respect to the horizontal,
$\theta - \alpha = \pi/2$ is used here.  Furthermore $\theta = \pi/2 + \alpha$ .

\item
Using the fundamental trigonometric functions, polar coordinates can be easily converted to Cartesian
coordinates, such that $P=P_{r,\theta}=(r, \theta)$ and $P=P_{x,y}=(\Delta x, \Delta y)$, as follows:
\end{enumerate}

%% Equations 
From:
\begin{align*}
\cos(\theta) &= \frac{\Delta x}{r} \\
\sin(\theta) &= \frac{\Delta y}{r}
\end{align*}

it is found,
\begin{align}
\Delta x &= r \cdot \cos(\theta)\label{eqn:deltax} \\
\Delta y &= r \cdot \sin(\theta)\label{eqn:deltay}
\end{align}

and since:
\begin{equation*}
\theta-\alpha=\frac{\pi}{2} \\
\end{equation*}
\begin{equation}\label{eqn:90degoffset}
\theta = \frac{\pi}{2}+\alpha
\end{equation}

substituting equation \ref{eqn:90degoffset} into equations \ref{eqn:deltax} and \ref{eqn:deltay} yields:
\begin{align*}
\Delta x &= r \cdot \cos\left(\frac{\pi}{2} + \alpha\right) \\
\Delta y &= r \cdot \sin\left(\frac{\pi}{2} + \alpha\right)
\end{align*}

Therefore, from above:
\begin{equation*}
  P = (\Delta x, \Delta y) \\
\end{equation*}
\begin{equation}\label{eqn:tangent_point}
\boxed
{
P = \left(r \cdot \cos\left(\frac{\pi}{2} + \alpha\right),  r \cdot \sin\left(\frac{\pi}{2} + \alpha\right)\right)
}
\end{equation}

\begin{flushright}   
Q.E.D.
\end{flushright}   


