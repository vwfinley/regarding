\section{Network Setup}
Section 6 of the previous article provided instructions for setting up the network.
Those steps assumed a GUI since we were working with a Desktop version of Ubuntu.

This time around we are working with a headless server version of Ubuntu.
So, we will need to configure the network manually by editing text files.

Let's start by doing section 6.1 to setup a second virtual ethernet adapter as a bridged network adapter.

You will recall previously on Ubuntu Desktop the second virtual ethernet adapter was automatically setup to receive an IP address from your local DHCP server.
However, in Ubuntu Server we need to manually setup this second adapter with an IP address.

We'll manually setup the second adapter to receive a dynamic IP address from the local DHCP server.
We will do this to determine which network on the host (our laptop) to use for our bravo VM. 

Once we know which network to use we'll manually set a static IP address.
Setting a static IP address makes sense on a server because any clients need a reliable unchanging address to locate the server.

Let's start by discovering the virtual network adapters available to our VM.
At the commandline on the VM enter: 
\verbatim{begin} 
\$ ip a

1: lo: <LOOPBACK,UP,LOWER_UP> mtu 65536 qdisc noqueue state UNKNOWN group default qlen 1000
    link/loopback 00:00:00:00:00:00 brd 00:00:00:00:00:00
    inet 127.0.0.1/8 scope host lo
    ...
    ...
    ...
2: enp0s3: <BROADCAST,MULTICAST,UP,LOWER_UP> mtu 1500 qdisc fq_codel state UP group default qlen 1000
    link/ether 08:00:27:31:cd:75 brd ff:ff:ff:ff:ff:ff
    inet 10.0.2.15/24 metric 100 brd 10.0.2.255 scope global dynamic enp0s3
       valid_lft 79566sec preferred_lft 79566sec
    ...
    ...
    ...
3: enp0s8: <BROADCAST,MULTICAST,UP,LOWER_UP> mtu 1500 qdisc fq_codel state UP group default qlen 1000
    link/ether 08:00:27:c3:8e:b1 brd ff:ff:ff:ff:ff:ff
\verbatim{end} 

You should see output that looks similar to what is shown above.
We are interested in the two vitual adapters created by VirtualBox.
Above they start with the prefix "enp0s".

The vitual network adapter with ID=enp0s3 should have an IP4 address already set by VirtualBox.
This adapter was defined by the first adapter tab in VirtualBox.
In this case the address is 10.0.2.15 and the address was automatically assigned by DNS.
VirtualBox has it setup for NAT.
This adapter will be primarilly used for outbound connections to the internet from the VM, and inbound responses back to the VM.

You defined the second virtual network adapter with ID=enp0s8 on the second adapter tab in VirtualBox.
Remember you had defined this adapter in VirtualBox with Bridge Host configruation.  
You'll notice that adapter enp0s8 has no IP address assigned, yet.

\section{Netplan}

https://linuxconfig.org/how-to-configure-static-ip-address-on-ubuntu-18-04-bionic-beaver-linux

The network manager used by Ubuntu is called "Netplan".
Netplan is responsible for configuring any network adapters and getting an IP address assigned to them.

Netplan uses one or more yaml files to configure how it will setup adapters.

Open the file /etc/netplan/50-cloud-init.yaml in an editor:
\verbatim{begin}
\$ sudo vi /etc/netplan/50-cloud-init.yaml
\verbatim{end}

Edit the file to add configuration for adapter enp0s8.
Set it up to get its IP address from the DHCP server.
Save the file and quit the editor.
\verbatim{begin}
network:
    ethernets:
        enp0s3:
            dhcp4: true
        enp0s8:
            dhcp4: true
    version: 2
\verbatim{end}
sudo netplan apply

Now restart netplan to see the changes applied.
\verbatim{begin}
\$ sudo netplan apply
\$ ip a

1: lo: <LOOPBACK,UP,LOWER_UP> mtu 65536 qdisc noqueue state UNKNOWN group default qlen 1000
    link/loopback 00:00:00:00:00:00 brd 00:00:00:00:00:00
    inet 127.0.0.1/8 scope host lo
    ...
    ...
    ...
2: enp0s3: <BROADCAST,MULTICAST,UP,LOWER_UP> mtu 1500 qdisc fq_codel state UP group default qlen 1000
    link/ether 08:00:27:31:cd:75 brd ff:ff:ff:ff:ff:ff
    inet 10.0.2.15/24 metric 100 brd 10.0.2.255 scope global dynamic enp0s3
    ...
    ...
    ...
3: enp0s8: <BROADCAST,MULTICAST,UP,LOWER_UP> mtu 1500 qdisc fq_codel state UP group default qlen 1000
    link/ether 08:00:27:c3:8e:b1 brd ff:ff:ff:ff:ff:ff
    inet 10.0.0.117/24 metric 100 brd 10.0.0.255 scope global dynamic enp0s8
    ...
    ...
    ...
\verbatim{end}

Notice that adapter enp0s8 has an IP address assigned?
In the example above the IP address and subnet mask (CIDR block) is 10.0.0.117/24.
This tells us that our bravo VM is connected on host (VirtualBox on the laptop) network 10.0.0.X via adapter enp0s8.

To test everything is working okay, ping address 10.0.0.117 from the Windows host on the laptop.

In Windows open a CMD shell session and enter:
\verbatim{begin}
C:\> ping 10.0.0.117
\verbatim{end}
You should see a reply from the bravo VM.
\verbatim{begin}

