\section{Cloud Init}
Reboot our VM named "bravo".
Log back in and view the /etc/netplan/50-cloud-init.yaml file.

You will see the change we made in the previous section was removed and now a notice appears at the top of the file.
The notice says "...To disable cloud-init's network configuration capabilities..."

Cloud init assumes Ubuntu Server will be running on a cloud platform.
It wants to setup the server's network to align with a cloud hosting enviroment.

We are not interested in running on a cloud platform, so we will manually setup the network.
To manually setup the network, we need to disable Cloud Init's network setup.

\subsection{Disable Cloud Init network setup}
To disable Cloud Init's network setup, let's create a file at /etc/cloud/cloud.cfg.d/99-disable-network-config.cfg containing the following line:
"network: {config: disabled}"

Better yet, run the following script at the command line:

\begin{verbatim} 
echo "network: {config: disabled}" \
    | sudo tee /etc/cloud/cloud.cfg.d/99-disable-network-config.cfg > /dev/null
\end{verbatim} 

\subsection{Set Static IP}
Okay, again edit the /etc/netplan/50-cloud-init.yaml file as sudo.

This time change it so that virtual adapter enp0s8 has:
\begin{itemize}
  \item dhcp4 set to false.
  \item the static IP address, you wrote down in the previous section (10.0.0.117/24), explicitly set.
\end{itemize}

Save the the /etc/netplan/50-cloud-init.yaml file file.
It should appear as below.
\begin{verbatim} 
network:
    ethernets:
        enp0s3:
            dhcp4: true
        enp0s8:
            dhcp4: false
            addresses:
                - 10.0.0.117/24
    version: 2
\end{verbatim} 

Now reboot the VM.
When the VM is back up, ping the static IP address (10.0.0.117) from the laptop Windows host CMD shell.
You should see a reply to the ping.