\section{Microsoft Visual Studio Code (VSCode)}
Now that the SSH keys and connection have been established, it is a convenient time to setup VSCode.
Sure there are other IDEs you could install, but we'll be installing VSCode as the baseline example.
VSCode has a couple advantages: it is widely used by developers, and it is freely available.

\subsection{Setup VSCode}
To setup VSCode:
\begin{enumerate}
  \item On your your laptop, browse to the Microsoft App Store at https://apps.microsoft.com/
  \item Search for VSCode or go here: \begin{verbatim}https://apps.microsoft.com/detail/xp9khm4bk9fz7q?hl=en-US&gl=US \end{verbatim}
  \item Download and Install VSCode on your laptop.
  \item After VSCode has installed, launch it.
  \item The current VSCode version is Version 1.100.2, you version may be newer.
\end{enumerate}


\subsection{Install Remote Development extension pack}
Once VSCode is up and running, we need to install the Remote Development extension pack.
The pack contains 4 extensions that will help us connect VSCode to the alpha VM.

\begin{enumerate}
  \item In VSCode goto View $>$ Extensions or type (Ctrl+Shift+X)
  \item In the ``Search Extensions in Marketplace'' search bar, type ``Remote Development''.
  \item The Remote Development pack icon (has number 4) should appear at the top of the search results.
  \item Click the blue install button under Remote Development.
  \item After the Remote Development pack installs you may be asked to restart VSCode.
\end{enumerate}


\subsection{Connect VSCode to the alpha VM}
Let's point VSCode to our alpha VM.
Doing so will let us use VSCode to develop code on alpha.

\begin{enumerate}
  \item First we need to verify our alpha VM is running in VirtualBox.
  \item Next, in VSCode click Remote Explorer icon on the left vertical panel.
  \item Select Remotes(Tunnels/SSH) from drop down at top of Remote Explorer panel.
  \item Hover over the SSH section on the panel.
  \item Click the Plus (New Remote) to the right of SSH.
  \item In the ``Enter SSH Connection Command'' box, enter ``ssh your_username@remotemachine -A'', for example: ssh vfinley@alpha -A
  \item In the ``Select SSH configuration file to update'' box enter the first file in the list, for example C:/Users/vfinl/.ssh/config
\end{enumerate}

\subsection{Initial Connection}
Let's use VSCode to connect to the alpha VM.

\begin{enumerate}
  \item In Remoter Explorer SSH list, find the host (alpha) you just added.
  \item Hover over alpha
  \item Click the right arrow (Connect in current window) button.
  \item Wait about 30 sec for VSCode server to install.
  \item Click Explorer icon in left bar (Ctrl+Shift+E), click ``Open Folder'' button.
  \item Select default path, click OK
  \item You will see dialog box ``Do you trust the authors of the files in this folder?''
  \item Click ``Yes, I trust the authors''
\end{enumerate}


\subsection{Testing the terminal}
Once we've got a connection to our alpha VM, we need to check the terminal is available.

\begin{enumerate}
  \item In VSCode goto View $>$ Terminal
  \item Click in the terminal panel that appears at bottom and enter: ping google
\end{enumerate}


