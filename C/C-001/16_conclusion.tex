\section{Conclusion}
This article described the setup of a basic remote containerized Golang development environment.
While you may not be writing Golang code, you can follow the general steps to setup an environment for your project.

Every project is different.
Your software development project will be different than mine.
I guarantee it!

\begin{itemize}
  \item Instead of writing Golang code, you may be writing Java code.
  \item Instead of hosting a remote VM in VirtualBox, you remote VM maybe running in an AWS EC2 instance.
  \item Instead of running Ubuntu in your VM, you may be running RHEL.
  \item Instead of containers in Docker you may have containers in Podman.
  \item Instead of VSCode on a Windows Laptop, you may have Emacs on a Linux desktop box.
\end{itemize}

Yes, each of these scenarios are different.
However, despite the differences from project to project, the general approach to setting up a development environment is similar.
You can take the steps that are laid out in this article and substitute whichever technologies you are using for your project.

What is important is the process.
Recognize the patterns!

Best wishes on your software development project.