\section{Solution}
As with most developers, you are probably working on either a laptop or desktop PC.
You are likely running a version of Microsoft Windows.
Perhaps you are reluctant to uninstall that Windows distribution that came with your machine?
Or perhaps your employer's IT policy prevents you?

Good news, there's a solution to develop in a more friendly open-source Linux environment.
Docker/Podman containers accommodate hosting of guest environments on your available hardware.
Your compiler and development libraries could live in a VSCode developer Docker container, while your code could reside on the hosting system.

If the hosting system were some remote Linux machine with a Docker/Podman installation, then you would have a pretty good development environment.
Sure, you could install Docker/Podman directly onto your Windows laptop and run your containers there.
But wouldn't you rather develop in a true Linux environment?
Especially when you are writing cloud deployable microservices and web-apps?

The advantage of remoting into the development hosting environment is that you may be working with multiple teams on multiple projects.
Each project or team could have their own development hosting environment.
In the morning, you could point the IDE on your laptop to project A.
After lunch, you could point it to project B.
You could also work on different applications within the same project environment, by simply running different containers.
All this is possible without reconfiguring or disturbing your laptop.

\subsection{Strategy}
Ok, so what do we want to accomplish?
Let's assume we want to write Golang code for some project.

\begin{itemize}
  \item We will need a container setup to build/run our Golang program.
  \item We will need Docker/Podman installed on a remote Linux machine, to run the container.
  \item We will also need a client to manage code in our source control repository.
  \item We will need an IDE installed on our Windows laptop, talking to the remote Linux machine.
\end{itemize}

Our development environment will roughly look like Figure \ref{fig:fig1}.

\begin{figure}[!h]
\centering
\begin{tikzpicture}
  \node (r1) at (0,0)  [draw, align=center, label={[align=center]above:Windows\\Laptop}, minimum height=2cm,minimum width=2cm] {VSCode};
  \node (r2) at (5,0)  [draw, align=center, label={[align=center]above:Ubuntu\\Remote VM}, minimum height=2cm,minimum width=2cm] {Docker\\+ Git};
  \node (r3) at (10,0) [draw, align=center, label={[align=center]above:Dev\\Container}, minimum height=2cm,minimum width=2cm] {Golang};
  \node (r4) at (2.5,-5) [draw, align=center, minimum height=2cm,minimum width=2cm] {Git\\Repo};
  \node (r5) at (7.5,-5) [draw, align=center, minimum height=2cm,minimum width=2cm] {Container\\Repo};
  \draw[arrows={-Latex[length=6pt]}] (r1.east) -- (r2.west);
  \draw[arrows={-Latex[length=6pt]}] (r2.east) -- (r3.west);
  \draw[arrows={-Latex[length=6pt]}] (r2.south) -- (r4.north);
  \draw[arrows={-Latex[length=6pt]}] (r2.south) -- (r5.north);
\end{tikzpicture}
\caption{The Development Environment}
\label{fig:fig1}
\end{figure}


\subsection{Remote Dev Machine}

For this exercise, many of the technology choices will have already been made for us.
For example, our Windows 11 laptop is a given.

Other technology choices are open-ended.
The biggest choice we need to make is related to the remote development machine.
\begin{itemize}
  \item Which Linux distribution will we run?
  \item Will the machine be physical or virtual?
  \item What hardware are we running on?
\end{itemize}

If you have a remote physical machine that you can dedicate to running a Linux distribution, you are indeed lucky.
In most cases you will be interested in running a Linux distribution as a Virtual Machine inside a hypervisor.

Fortunately, there are many hypervisors available.
WSL, HyperV, vSphere, ESXi, ProxMox, VirtualBox: are some of the hypervisors you could use to run the Linux instance in a VM.
Many can be installed either on a networked server or locally on your laptop.
Which you choose depends upon budget, licensing and compute hardware.

Another popular solution is to run the Linux instance in a Cloud VM service.
For example AWS EC2 or Google Compute Engine are popular choices.

For my environment I will be installing Oracle VirtualBox hypervisor on my laptop.
Yes, the Linux instance can run in a VirtualBox hosted VM on the same Windows laptop where my IDE runs.
From the IDE's point of view, the Linux instance will appear to be on a remote network machine.
The IDE doesn't care if the remote machine lives in an EC2 instance or in a hypervisor 

\subsection{Linux Distribution}

The best place to start searching for a Linux distribution is DistroWatch\footnote{https://distrowatch.com}.
There you will find hundreds of Linux (and BSD) operating system distributions.

Linux distributions fall into two main families that differ by package manager and available packages.
Each has its pros and cons.
The RHEL/Fedora family uses the YUM\footnote{Yellowdog Updater, Modified} package manager.
The Debian/Ubuntu family uses the APT\footnote{Advanced Package Tool} package manager.

RHEL/Fedora has packages to install Podman and Buildah.
Podman increases security by enforcing rootless containers.

Debian/Ubuntu has packages to install Docker.
Docker is easier to work with since much existing tooling had been developed for it.
Sharing project files with developer containers is also easier in Docker.

For my development environment I will choose an Ubuntu\footnote{https://ubuntu.com/} Desktop distribution since I want the ease of working with Docker.

\subsection{Source Control}

Git is the de facto standard for modern source control.
Choosing Git is a no-brainer since I will be using a repository in GitHub to keep my source code.

\subsection{IDE}

Installing VSCode on my laptop is a matter of convenience.
Better IDEs probably exist, but I am familiar with VSCode.

VSCode has the advantage of developer container support.
It also has extensions that integrate remote SSH connections.

During routine usage VSCode acts almost like a thin-client terminal. 
Once VSCode is setup, developing code in a dev container on a remote machine appears similar to developing code on your local laptop.
This makes for a nice developer experience, to the point where one forgets where the code and compiler actually lives.

\subsection{Password Manager}

As you develop complex systems, you will accumulate (and forget) multiple passwords.
It is helpful to secure your account logins and passwords in a single place.

Several systems are available to keep your passwords handy and safe.
I recommended you install one of the following password managers: Bitwarden, 1Password and KeePassXC; on your PC.

\subsection{Laptop}

VirtualBox hypervisor will be running on the same laptop where VSCode will be installed.
For performance reasons it is worth noting the laptop configuration.
Knowing the laptop configuration helps when setting expectations and making comparisons to other hardware environments.

\begin{verbatim}
Device Specifications
---------------------
Manufacturer: Lenovo
Model: IdeaPad Slim 5 16IRU9
Processor: Intel(R) Core(TM) 7 150U   1.80 GHz
Installed RAM: 16.0 GB (15.7 GB usable)
Product ID: 00342-22290-21080-AAOEM
System type: 64-bit operating system, x64-based processor
Pen and touch: No pen or touch input is available for this display


Windows Specifications
----------------------
Edition: Windows 11 Home
Version: 24H2
Installed on: 2/14/2025
OS build: 26100.3775
Experience: Windows Feature Experience Pack 1000.26100.66.0
\end{verbatim}

\subsection{Environment Summary}

After much consideration, the development below was selected.
By no means is this an optimal solution.
You can swap in/other components of this environment to suit your development scenario.

\begin{itemize}
  \item Host OS: Windows 11 Home Edition
  \item Password Manager: KeePassXC 2.7.9
  \item Hypervisor: Oracle VirtualBox Version 7.1.6 r167084 (Qt6.5.3)
  \item IDE: Visual Studio Code 1.100.0 (user setup)
  \item Guest OS: Ubuntu 22.04.5 LTS Desktop
  \item Container runtime: Docker 28.8.1, build 4eba377
  \item Source code repository: GitHub
  \item Container image repository: DockerHub
  \item Golang container: 
  \item Golang version
\end{itemize}

