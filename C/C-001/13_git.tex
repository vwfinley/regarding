\section{git}

Every development project, big and small, needs source code control.
Git is the de facto standard for source control in modern development projects.
We will setting up our project to use a Git client with GitHub as our online repository.
In practice you may substitue online and onprem Git servers, including GitLab, Azure DevOps, etc.

\subsection{Git Install on Remote VM}
Connect to the alpha remote VM with VSCode.
Open the terminal in VSCode, it should be connected to the alpha VM.
Do the following:

\begin{verbatim}
# Install Git client on the alpha VM.
$ sudo apt install git
\end{verbatim}

\subsection{Initalize user information}
Setup your user information.
It will be used to identify you when reviewing Git change history.

\begin{verbatim}
# Register your name and email with Git.
# Be sure to change user name and email to YOUR name and email address
$ git config --global user.name "John Doe"
$ git config --global user.email "john.doe@youremail.com"
\end{verbatim}

\subsection{Personal Access Token (PAT)}
Personal Access Tokens (PATs) provide more security and flexibility than user passwords.
Many Git servers are configured to use PATs exclusively.
You will need to generate a PAT at least once per year.
Don't worry if you forget or lose your PAT, you can always generate a new one.

To create a PAT do the following:
\begin{enumerate}
  \item Login to GitHub or other online/onprem Git service.
  \item Click on the user icon
  \item Select "Settings" from the menu.
  \item Left menu, scroll to bottom, click "Developer Settings".
  \item Left menu, under Access Tokens, select Fine-Grained Tokens
  \item Click green "Generate new token" button.
  \item Fill in the "Token name" field with a name you will remember.
  \item Set the Expiration date to a convenient date.
  \item Select "All Repositories"
  \item Under "Permissions / Repository permissions" at a minimum give yourself "Read and write" access to Contents and Pull requests.
  \item Click the green "Generate token" button at the bottom.
  \item In the popup dialog click the green "Generate token" again.
  \item In the page that appears, copy the PAT (including the github_pat_ prefix), name and expiration date.
  \item Save it somewhere for later use, like in a password manager (KeePassXC, Bitwarden, etc.).
\end{enumerate}

\subsection{Identify Git Repository}
The first thing you'll need to do is to identify a Git repository that you will be working out of.

\begin{enumerate}
  \item In browser find a Git project where you have write priviliges. In my case I have prviliges to write: https://github.com/vwfinley/regarding
  \item Click on green "Code" button.
  \item Click on the "Local" tab.
  \item Select "Https"
  \item Copy the URL in the "Clone using the web URL" field.  For example: https://github.com/vwfinley/regarding.git
\end{enumerate}

\subsection{Clone Project}
Next you'll need to clone the project.
In the VSCode BASH shell terminal that is connected to our alpha VM do the following.

\begin{verbatim}
# Create a directory for your project, then cd to go there.
$ mkdir ~/Documents/projects
$ cd ~/Documents/projects/

# Clone the repo with the URL you copied above.
~/Documents/projects$ git clone https://github.com/vwfinley/regarding.git
Cloning into 'regarding'...
remote: Enumerating objects: 2047, done.
remote: Counting objects: 100% (271/271), done.
remote: Compressing objects: 100% (179/179), done.
remote: Total 2047 (delta 188), reused 158 (delta 90), pack-reused 1776 (from 2)
Receiving objects: 100% (2047/2047), 3.91 MiB | 4.60 MiB/s, done.
Resolving deltas: 100% (1243/1243), done.
\end{verbatim}


\subsection{Git Password Saver Setup}
Typing your password/PAT over-and-over for Git(Hub/Lab) is annoying.
Fortunately there is a better way.

At the terminal Bash shell prompt enter:

\begin{verbatim}
$ git config --global credential.helper store
\end{verbatim}

Next time you enter password/PAT info for Git, you password/PAT will be remembered thereafter.
Some Git Repositories are setup to use a PAT in place of a password, when prompted for a password use the PAT instead.


\subsection{Initial Checkin}
Before changes appear in a remote Git repository, a change must be made to your working copy.
Let's make a change to the project that was previously cloned, and check-in the change.

\begin{verbatim}
# Make change to the project (for example).
~/Documents/projects$ cd regarding/
~/Documents/projects/regarding$ ls
A  B  LICENSE.md  README.md
~/Documents/projects/regarding$ mkdir C
~/Documents/projects/regarding$ ls
A  B  C  LICENSE.md  README.md
~/Documents/projects/regarding$ mkdir C/C-001
~/Documents/projects/regarding$ cd C/C-001

# Create new file
~/Documents/projects/regarding/C/C-001$ echo "Hello World" > README.md

# Initial check-in
~/Documents/projects/regarding/C/C-001$ git add README.md
~/Documents/projects/regarding/C/C-001$ git commit -m "New File"
[main 5ed57a6] New File
 1 file changed, 1 insertion(+)
 create mode 100644 C/C-001/README.md
\end{verbatim}

\subsection{Initial Push}
Now, let's push the change to the remote repository so we can see it.

\begin{verbatim}
 ~/Documents/projects/regarding/C/C-001$ git push

# When prompted for password paste your PAT token.
~/Documents/projects/regarding/C/C-001$ git push
Username for 'https://github.com': vwfinley
Password for 'https://vwfinley@github.com': <paste_PAT_token_here> 

Enumerating objects: 6, done.
Counting objects: 100% (6/6), done.
Delta compression using up to 2 threads
Compressing objects: 100% (2/2), done.
Writing objects: 100% (5/5), 365 bytes | 365.00 KiB/s, done.
Total 5 (delta 1), reused 0 (delta 0), pack-reused 0
remote: Resolving deltas: 100% (1/1), completed with 1 local object.
To https://github.com/vwfinley/regarding.git
   7ff2319..5ed57a6  main -> main
\end{verbatim}

\subsection{VSCode Git Integration}
Support for source code control is built-in to VSCode.
You will notice as you make changes to a cloned project, that the Source Control icon on the VSCode left vertical panel will update with the number of pending code changes.

To check-in and push the changes

\begin{enumerate}
  \item Click the icon or type Ctrl+Shift+G.
  \item Click the plus icon next to each file you want to commit.
  \item Type a commit message above the blue Commit button.
  \item Click the blue commit button.
  \item 

\end{enumerate}
