\section{Hosts File}

At this point we have an Ubuntu Linux desktop distribution running in a virtual machine.
Our virtual machine is named "alpha".

We want to begin establishing communication between our laptop and the "alpha" VM.
The first thing we want to do is discover which network alpha is on, and what address can be used to access it.

To accomplish this first open a terminal on alpha and do the following:
\begin{enumerate}
  \item Goto: Activities $>$ Settings $>$ Network
  \item Click the gear icon beside each ethernet connection.
  \item Write down the IPv4 Link address for each ethernet connection.
  \item For example, addresses for 2 ethernet connections: 10.0.2.15 and 192.168.56.104
\end{enumerate}

The next step is to Ping alpha with our ethernet connections addresses, and determine which alpha responds to.
On our laptop, open a Windows cmd shell and do the following:
\begin{verbatim}
C:\Users\vfinl>ping 10.0.2.15

Pinging 10.0.2.15 with 32 bytes of data:
Request timed out.

Ping statistics for 10.0.2.15:
    Packets: Sent = 4, Received = 0, Lost = 4 (100% loss),

C:\Users\vfinl>ping 192.168.56.104

Pinging 192.168.56.104 with 32 bytes of data:
Reply from 192.168.56.104: bytes=32 time=1ms TTL=64

Ping statistics for 192.168.56.104:
    Packets: Sent = 4, Received = 4, Lost = 0 (0% loss),
Approximate round trip times in milli-seconds:
    Minimum = 0ms, Maximum = 1ms, Average = 0ms
\end{verbatim}

Alpha responded to the Ping at address 192.168.56.104.
This address will be used to communicate with alpha hereafter.
However, the address is difficult to remember and tedious to type frequently.

It would be good to alias the address to a friendly name like "alpha".
Let's update the laptop's hosts file to associate the address to the friendly name.

To edit the hosts file on a Windows 11 system, run Notepad as admin.
\begin{itemize}
  \item Click Windows start icon.
  \item Type "notepad".
  \item Expand menu under Open.
  \item Select as "Run as administrator".
  \item Open file \begin{verbatim}C:\Windows\System32\drivers\etc\hosts\end{verbatim}
  \item Add following entry to the file: \begin{verbatim}192.168.56.104  alpha # Ubuntu VM\end{verbatim}
  \item Save the file.
  \item Close Notepad.
\end{itemize}

Verify that the friendly name is associated with the address.
At a Windows cmd shell prompt, do the following and verify alpha replies to the ping.

\begin{verbatim}
C:\Users\vfinl>ping alpha

Pinging alpha [192.168.56.104] with 32 bytes of data:
Reply from 192.168.56.104: bytes=32 time=1ms TTL=64
Reply from 192.168.56.104: bytes=32 time<1ms TTL=64
Reply from 192.168.56.104: bytes=32 time<1ms TTL=64
Reply from 192.168.56.104: bytes=32 time<1ms TTL=64

Ping statistics for 192.168.56.104:
    Packets: Sent = 4, Received = 4, Lost = 0 (0% loss),
Approximate round trip times in milli-seconds:
    Minimum = 0ms, Maximum = 1ms, Average = 0ms
\end{verbatim}


%%\begin{verbatim}
%%C:\Windows\System32\drivers\etc\hosts
%%- Add entry: 192.168.56.104    alpha # Ubuntu VM
%%- Save and close the hosts file
%%- In CMD shell: ping alpha
%%\end{verbatim}

