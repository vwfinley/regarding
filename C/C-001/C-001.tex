% Template based upon included TeXworks Basic LaTeX documents article.tex template.

% !TEX TS-program = pdflatex
% !TEX encoding = UTF-8 Unicode

% This is a simple template for a LaTeX document using the "article" class.
% See "book", "report", "letter" for other types of document.

\documentclass[11pt, onecolumn]{article} % use larger type; default would be 10pt

\usepackage[utf8]{inputenc} % set input encoding (not needed with XeLaTeX)

%%% Examples of Article customizations
% These packages are optional, depending whether you want the features they provide.
% See the LaTeX Companion or other references for full information.

%%% PAGE DIMENSIONS
\usepackage{geometry} % to change the page dimensions
\geometry{letterpaper} % or letterpaper (US) or a5paper or....
\geometry{margin=1in} % for example, change the margins to 2 inches all round
% \geometry{landscape} % set up the page for landscape
%   read geometry.pdf for detailed page layout information

\usepackage{graphicx} % support the \includegraphics command and options

% \usepackage[parfill]{parskip} % Activate to begin paragraphs with an empty line rather than an indent

%%% PACKAGES
\usepackage{tikz}
\usetikzlibrary{arrows.meta} % for figures
\usepackage{amsmath}
\usepackage{gensymb}
%\usepackage{unicode-math}

\usepackage{booktabs} % for much better looking tables
\usepackage{array} % for better arrays (eg matrices) in maths
\usepackage{paralist} % very flexible & customisable lists (eg. enumerate/itemize, etc.)
\usepackage{verbatim} % adds environment for commenting out blocks of text & for better verbatim
\usepackage{subfig} % make it possible to include more than one captioned figure/table in a single float
% These packages are all incorporated in the memoir class to one degree or another...

%%% HEADERS & FOOTERS
\usepackage{lastpage} % page x of y
\usepackage{fancyhdr} % This should be set AFTER setting up the page geometry

%%% Background color for commandline/code
\usepackage{color}
\let\oldv\verbatim
\let\oldendv\endverbatim
\def\verbatim{\par\setbox0\vbox\bgroup\oldv}
\def\endverbatim{\oldendv\egroup\fboxsep6pt \noindent\colorbox[gray]{0.8}{\usebox0}\par}

%%% Foce lines between paragraphs
\usepackage[indent=12pt]{parskip}

\pagestyle{fancy} % options: empty , plain , fancy
%\renewcommand{\headrulewidth}{0pt} % customise the layout...
%\lhead{}\chead{}\rhead{}
%\lfoot{}\cfoot{\thepage\ of \pageref{LastPage}}\rfoot{}

\fancypagestyle{plain}{
  \fancyhf{} 
  \fancyhead[R]{}
  \fancyfoot[C]{\thepage\ of \pageref{LastPage}\\{\scriptsize https://github.com/vwfinley/regarding/blob/main/C/C-001/C-001.pdf}}
  \renewcommand{\headrulewidth}{0pt}
}

\fancyhf{}
  \fancyhead[R]{}
  \fancyfoot[C]{\thepage\ of \pageref{LastPage}}
\renewcommand{\headrulewidth}{0pt}

%%% SECTION TITLE APPEARANCE
\usepackage{sectsty}
\allsectionsfont{\sffamily\mdseries\upshape} % (See the fntguide.pdf for font help)
% (This matches ConTeXt defaults)

%%% ToC (table of contents) APPEARANCE
\usepackage[nottoc,notlof,notlot]{tocbibind} % Put the bibliography in the ToC
\usepackage[titles,subfigure]{tocloft} % Alter the style of the Table of ContentsT
\renewcommand{\cftsecfont}{\rmfamily\mdseries\upshape}
\renewcommand{\cftsecpagefont}{\rmfamily\mdseries\upshape} % No bold!

% Package for boxes around aligned equations
\usepackage{empheq}
\newcommand*\widefbox[1]{\fbox{\hspace{2em}#1\hspace{2em}}}

% Packages for complex tables
\usepackage{multirow}
\usepackage{makecell}

% Package for long multilined equations
%\usepackage{mathtools}
\usepackage{breqn}

% For boxed multiline equations
\usepackage{empheq}

%%% END Article customizations

%%% The "real" document content comes below...

\title{C-001\\Regarding Setup of a\\Development Environment}
\author{Vincent W. Finley\thanks{\copyright 2025: Bear, DE, USA: CC BY 4.0 license\\https://creativecommons.org/licenses/by/4.0/legalcode}}
\date{Rev A, July 2025} % Activate to display a given date or no date (if empty),
         % otherwise the current date is printed 

\begin{document}
\maketitle

Software development environments hosted on a headless server increases opportunities for developer collaboration.

Setup differences of remote Ubuntu server vs. desktop are described.
Manual configuration of Netplan is investigated.


\section{Introduction}
01_intro.tex

TODO: Write Me!

Blah, Blah, Blah

Example footnote\footnote{https://github.com/vwfinley/regarding}

Example bulleted list
\begin{itemize}
  \item This is
  \item a bulleted
  \item list
\end{itemize}

Example enumerated list
\begin{enumerate}
  \item This is
  \item an enumerated bulleted
  \item list
\end{enumerate}


\section{Solution}
You are probably working on either a laptop or desktop PC running Microsoft Windows.
This was the same situation encountered in the previous article.

Microsoft Windows already has a desktop.
As in the previous article, we will use the Windows desktop the as a ``terminal'' that can connect to remote systems.

However this time instead of installing Ubuntu Desktop in a VM, we will install Ubuntu Server in the VM.
Ubuntu Server will act as our remote development system.

We can describe Ubuntu Server as ``headless'' because it does not include a GUI.
A headless server installation is more appropriate place for teams of developers to collaborate.

Okay, so what's our strategy to setup a remote development environment?
We will:
\begin{itemize}
  \item Follow many of the steps found in the previous article.
  \item Install an Ubuntu server distro rather than the Ubuntu Desktop distro.
  \item Make a few changes that are unique to the Server distro.
  \item Connect VSCode to the remote Ubuntu Server instance.
  \item Complete the environment setup.
\end{itemize}






\section{Hypervisor}

Let's begin setting up the development environment.
We will install Oracle VirtualBox onto the laptop.

Do the following:
\begin{enumerate}
  \item Browse to https://www.virtualbox.org/wiki/Downloads
  \item Select the link for ``Windows hosts''.
  \item Once the download has completed, open the download location and run the *.exe file.
  \item When prompted if you want the app to make changes to your Device, click Yes.
  \item On the welcom panel, click Next.
  \item Accept the terms of the license agreement and click Next.
  \item On the Custom Setup panel, accept all the defaults, and click Next.
  \item On the Warning: Network Interfaces panel, click Yes.
  \item After installation has completed, reboot your laptop.
  \item Once you've logged back into your laptop, launch Oracle VirtualBox from the Windows start menu.
\end{enumerate}

\section{Virtual Machine}

Setting up a Virtual Machine is not difficult, but there are many steps involved.

\subsection{Download Linux image}
To create an Ubuntu Desktop virtual machine on the VirtualBox hypervisor, we will need a distribution image.

\begin{enumerate}
  \item Browse to https://ubuntu.com/download/desktop
  \item Search the page for "check out our alternative downloads", and click it.
  \item Search the page under Past Releases for Ubuntu 22.04 LTS (Jammy Jellyfish), and click it.
  \item Click the link for "64-bit PC (AMD64) desktop image".
  \item Download of the file named ubuntu-22.04.5-desktop-amd64.iso will complete in a few minutes.
\end{enumerate}



\subsection{Create New Virtual Machine}
Let's create a new virtual machine in VirtualBox with the image we downloaded.

\begin{enumerate}
  \item From the Windows start menu, launch Oracle VirtualBox.
  \item In Oracle VirtualBox Manager application goto: Machine $>$ New... 
  \item In the Create Virtual Machine popup dialog, we'll set Name="alpha".
  \item Click the ISO Image field and search for the ubuntu-22.04.5-desktop-amd64.iso you downloaded.
  \item Set Type=Linux
  \item Set Subtype=Ubuntu
  \item Set Version=Ubuntu (64-bit)
  \item Check "Skip Unattended Installation".
  \item Expand the Hardware section.
  \item Increase Processors to 2 CPUs.
  \item Click the Finish button.
\end{enumerate}



\subsection{Start the Virtual Machine}
Okay, let's launch Ubuntu inside VirtualBox.

\begin{enumerate}
  \item In the Oracle VirtualBox Manager application, search the left panel for the machine "alpha" you just created.
  \item Right-click on "alpha" and select Start $>$ Normal Start.
  \item Once the boot screen appears in the preview pane, click "Show" on the top menu.
  \item Click inside window the boot window that appears, select "Try or Install Ubuntu", and hit enter.
\end{enumerate}

\subsection{Install Ubuntu}
Time to install Ubuntu onto the VM!

\begin{enumerate}
  \item Once the Ubuntu desktop appears, you should see a window the the title "Install".
  \item The page has two options, choose your language then select the option "Install Ubuntu".
  \item On the Keyboard Layout page, choose your keyboard layout and click "Continue".
  \item On the Updates and Other Software page, choose Minimal Installation, uncheck "Download Updates while installing Ubuntu", check "Install third-party software..."
  \item Click "Continue".
  \item On the Installation Type page, select "Erase disk and install Ubuntu".
  \item Click "Install Now".
  \item On the pop-up box that says "Write the changes to disk?", click "Continue".
  \item Select your timezone and click "Continue".
  \item On the "Who are you?" page, populate your name, set the computer name to "alpha", set your username and password.
  \item Choose "Require my password to log in".
  \item Click "Continue".
  \item Ubutnu will begin installing onto the VM, be patient it will take 10-15 minutes to complete.
  \item Once the installation has completed, click the "Restart Now" button.
  \item Wait a few moments, an Ubutnu screen will appear telling you to remove the installation media.  DO NOT Press Enter Yet!
  \item Go into the VirtualBox Manager application, select the "alpha" machine, and click Settings on the top menu.
  \item Select Storage from the Left panel.
  \item Under Controller IDE, click the disk icon.
  \item Uncheck Live CD/DVD.  Click OK.
  \item Now go back to the remove installation media screen, and press Enter.
  \item Wait for the VM to shutdown.
\end{enumerate}







\section{Initial Login}

We will restart the VM named alpha, and login for the first-time.

\subsection{First time login pages}
\begin{enumerate}
  \item In VirtualBox right-click the VM named alpha.
  \item Select Start $>$ Normal Start
  \item When the Ubuntu login window appears, login.
  \item When the desktop appears, a setup window named ``Online Accounts'' will appear.
  \item Click Skip.
  \item On the page Enable Ubutnu Pro, select Skip for Now and then click Next.
  \item On the Help Improve Ubuntu, make a choice and click Next.
  \item On the Privacy page, enable Location Services, and click Next.
  \item On the Your Ready To Go Page, click Done.
\end{enumerate}

\subsection{Screen Blank}

Until you get everything setup, it would be good to turn off the annoying screen blanking.

\begin{enumerate}
  \item In the Ubuntu desktop, got Activities $>$ Settings $>$ Privacy $>$ Screen
  \item Change Blank screen to: Never
  \item Change Automatic Screen Lock to: Disabled
  \item Change Lock Screen on Suspend to: Disabled
\end{enumerate}


\subsection{Power Auto Suspend}

Since the alpha VM is running on the laptop, it will lose connections when the laptop goes to sleep to save power.
Let's prevent this from happening.

\begin{enumerate}
  \item In the Ubuntu desktop, got Activities $>$ Settings $>$ Power
  \item Change Screen Blank to: Never
  \item Change Automatic Power Saver to: Disabled
  \item Change Automatic Suspend to: Off
\end{enumerate}


\subsection{Screen Resolution}

Finally it would be good for the alpha guest VM to make full use of the display on the laptop host.

\begin{enumerate}
  \item In the Ubuntu desktop, got Activities $>$ Settings $>$ Displays
  \item You will need to try different settings for Resolution.
  \item Click the green Apply button.
  \item If you like the resolution setting click Keep Changes, otherwise click Revert Settings.
  \item For my laptop, Resolution=1680x1050 seems to work best.
\end{enumerate}

\subsection{Shutdown}

Shutdown the alpha VM to prepare for the next section.
\section{Network Setup}

The Ubuntu desktop distribution has been installed into a VM that runs in VirtualBox on the laptop.
As far as the laptop is concerned, the VM is a wholly independent machine, even though the VM lives on the laptop.

We need a way to talk to the VM from the laptop.
To achieve this we will be making a change to the network, and then gathering some information.

Initally, VirtualBox created the VM with a single default network adapter that supports Network Address Translations (NAT).
This default adapter is fine for connecting to the outside world: browsing to an internet website, or pinging google.com.
In other words, outbound traffic is supported.

However, inbound traffic is unsupported.
If we were to ping the ``alpha'' VM from the commandline on the laptop, we would get no reply.

\subsection{Add Bridge Adapter}
To support inbound traffic, we need to add another network adapter.

\begin{enumerate}
  \item In VirtualBox: Select machine alpha (powered off).
  \item Goto Settings $>$ Network $>$ Adapter 2 tab
  \item Check Enable Network Adapter
  \item Change Attached to: = Bridged Adapter
  \item Click OK
\end{enumerate}



\subsection{Gather IP Addresses}
Let's verify the network is configured correctly, and can send/receive traffic.
While we are at it, let's find the alpha VM IP addresses.

\begin{enumerate}
  \item You should still be logged into the alpha VM from the previous section.
  \item Click Activities $>$ Settings $>$ Network
  \item For each of the two Ethernet connections:
    \begin{itemize}
      \item Click on the Gear button
      \item Click on the Details tab
      \item Write down the IPv4 addresses for later.
      \item For example: 10.0.2.15 and 10.0.0.246
    \end{itemize}
  \item Close the Settings application. 
\end{enumerate}


\subsection{Outbound connection}
We have two network adapters each with a IPv4 address associated to it.
It is unclear which address is the mapped to the NAT adapter and which is mapped to the Bridged adapter.
In other words we don't know which is being used for outbound and inbound traffic.
However we can test to verify the outbound connection is working properly.

\begin{enumerate}
  \item On the Ubuntu destop, click Activities.
  \item Search for and select ``terminal''.
  \item Once the terminal application launches, at the commandline type: \begin{verbatim}ping google.com\end{verbatim}
  \item Hit Enter
  \item Verify you get a response back.
  \item Hit Ctrl+C to stop pinging.
\end{enumerate}

\subsection{Inbound connection}
Okay, let's verify that the inbound (Bridged) connection is working.
It is also an opportunity to determine which address is the inbound address, and which is the outbound address.

\begin{enumerate}
  \item On the laptop, open a Windows CMD terminal.
  \item In at the commandline in the terminal you opened above, run ping with one of the addresses you wrote down above.
  \item For example: \begin{verbatim}ping 10.0.2.15\end{verbatim}
  \item If you get a response, it is the inbound (Bridged) traffic address, otherwise it is the outbound (NAT) address.
  \item Write either ``inbound'' or ``outbound'' next to this address.
  \item At the commandline try pinging the other address.
  \item For example: \begin{verbatim}ping 10.0.0.246\end{verbatim}
  \item If you get a response, it is the inbound (Bridged) traffic address, otherwise it is the outbound (NAT) address.
  \item Write either ``inbound'' or ``outbound'' next to this address.
\end{enumerate}





%%\begin{verbatim}
%%https://www.bing.com/videos/riverview/relatedvideo?q=virtualbox+ubuntu+22.04+cannot+ping+from+host&mid=5B1AA719E4ADB9624DB45B1AA719E4ADB9624DB4&mcid=FEEEA5610D984BF1B9DDC47C30A02B7B&FORM=VIRE
%%\end{verbatim}









\section{Software Updates}

You have verified that the network is configured and functional.
While your are still logged into the desktop, it is a great opportunity to update all the software on the system.

There are a couple ways to update.

\subsection{Method 1: Update Graphically}
Ubuntu desktop has an application named Software Updater to make system updates convenient.

You may notice a message box that periodically appears.
It will say ``Updated software is available for this computer.  Do you want to update it now?''.

You can also manually run the Software Updater application by going to: Activities $>$ Software Updater

When prompted, click Install Now.
Be sure to reboot after the update completes.

\subsection{Method 2: Update Commandline}
A better way to update is from the commandline.
This method better supports automation and will work with headless (server) Ubuntu distributions.

First launch a terminal: Activities $>$ Terminal

Then enter the following at the commandline.
\begin{verbatim}
$ sudo apt-get update
$ sudo apt-get upgrade
\end{verbatim}

You should reboot after the upgrade completes.

\input{08_hosts.tex}
\section{SSH Service}
Most Linux \textbf{server} distributions are already setup for inbound SSH connectivity.
However, Linux \textbf{desktop} distributions are setup to support outbound SSH connections only.
The Ubuntu desktop distribution is no exception.
It is setup to be an SSH client, but not a server.
Inbound SSH support must be setup manually.

\subsection{Install SSH Server}
To install the SSH service on an Ubuntu system\footnote{https://devconnected.com/how-to-install-and-enable-ssh-server-on-ubuntu-20-04/}, open a terminal on alpha and enter the following:

\begin{verbatim}
$ sudo apt-get install openssh-server
$ systemctl status ssh

# If ssh.service is not running, do the following command:
$ sudo systemctl restart ssh
\end{verbatim}


\input{10_sshclient.tex}
\section{Microsoft Visual Studio Code (VSCode)}
Now that the SSH keys and connection have been established, it is a convenient time to setup VSCode.
Sure there are other IDEs you could install, but we'll be installing VSCode as the baseline example.
VSCode has a couple advantages: it is widely used by developers, and it is freely available.

\subsection{Setup VSCode}
To setup VSCode:
\begin{enumerate}
  \item On your your laptop, browse to the Microsoft App Store at https://apps.microsoft.com/
  \item Search for VSCode or go here: \begin{verbatim}https://apps.microsoft.com/detail/xp9khm4bk9fz7q?hl=en-US&gl=US \end{verbatim}
  \item Download and Install VSCode on your laptop.
  \item After VSCode has installed, launch it.
  \item The current VSCode version is Version 1.100.2, you version may be newer.
\end{enumerate}


\subsection{Install Remote Development extension pack}
Once VSCode is up and running, we need to install the Remote Development extension pack.
The pack contains 4 extensions that will help us connect VSCode to the alpha VM.

\begin{enumerate}
  \item In VSCode goto View $>$ Extensions or type (Ctrl+Shift+X)
  \item In the ``Search Extensions in Marketplace'' search bar, type ``Remote Development''.
  \item The Remote Development pack icon (has number 4) should appear at the top of the search results.
  \item Click the blue install button under Remote Development.
  \item After the Remote Development pack installs you may be asked to restart VSCode.
\end{enumerate}


\subsection{Connect VSCode to the alpha VM}
Let's point VSCode to our alpha VM.
Doing so will let us use VSCode to develop code on alpha.

\begin{enumerate}
  \item First we need to verify our alpha VM is running in VirtualBox.
  \item Next, in VSCode click Remote Explorer icon on the left vertical panel.
  \item Select Remotes(Tunnels/SSH) from drop down at top of Remote Explorer panel.
  \item Hover over the SSH section on the panel.
  \item Click the Plus (New Remote) to the right of SSH.
  \item In the ``Enter SSH Connection Command'' box, enter ``ssh your_username@remotemachine -A'', for example: ssh vfinley@alpha -A
  \item In the ``Select SSH configuration file to update'' box enter the first file in the list, for example C:/Users/vfinl/.ssh/config
\end{enumerate}

\subsection{Initial Connection}
Let's use VSCode to connect to the alpha VM.

\begin{enumerate}
  \item In Remoter Explorer SSH list, find the host (alpha) you just added.
  \item Hover over alpha
  \item Click the right arrow (Connect in current window) button.
  \item Wait about 30 sec for VSCode server to install.
  \item Click Explorer icon in left bar (Ctrl+Shift+E), click ``Open Folder'' button.
  \item Select default path, click OK
  \item You will see dialog box ``Do you trust the authors of the files in this folder?''
  \item Click ``Yes, I trust the authors''
\end{enumerate}


\subsection{Testing the terminal}
Once we've got a connection to our alpha VM, we need to check the terminal is available.

\begin{enumerate}
  \item In VSCode goto View $>$ Terminal
  \item Click in the terminal panel that appears at bottom and enter: ping google
\end{enumerate}




\input{12_git.tex}
\section{Docker}
We eventually want to do our development in a developer container.
Before we can think about containers we need to install a container runtime on our alpha VM.

The easiest way to install a container runtime is to install either Docker or Podman.
Since much tooling was developed with Docker in-mind, Docker is the simplest to work with.
Podman is more secure, however mapping directories from the remote host to a rootless container can tricky.

When your remote host is Fedora/RHEL, you will probably use Podman.
Otherwise you will probably use Docker.

Since our alpha VM is running Ubuntu, we will install Docker.
Instructions to install Docker given by: \begin{verbatim}https://docs.docker.com/engine/install/ubuntu/#install-using-the-repository\end{verbatim} 
They are repeated here in case that webpage disappears.
To install Docker, at the VSCode Bash terminal connected to the alpha VM, do the following:
\begin{enumerate}
\item{Set up Docker's apt repository.}
\begin{verbatim}
# Add Docker's official GPG key:
sudo apt-get update
sudo apt-get install ca-certificates curl
sudo install -m 0755 -d /etc/apt/keyrings
sudo curl -fsSL https://download.docker.com/linux/ubuntu/gpg \
      -o /etc/apt/keyrings/docker.asc
sudo chmod a+r /etc/apt/keyrings/docker.asc

# Add the repository to Apt sources:
echo \
  "deb [arch=$(dpkg --print-architecture) \ 
  signed-by=/etc/apt/keyrings/docker.asc] \
  https://download.docker.com/linux/ubuntu \
  $(. /etc/os-release && \
  echo "${UBUNTU_CODENAME:-$VERSION_CODENAME}") stable" | \
  sudo tee /etc/apt/sources.list.d/docker.list > /dev/null
sudo apt-get update
\end{verbatim}

\item{Install the Docker packages.}
\begin{verbatim}
sudo apt-get install \
     docker-ce docker-ce-cli containerd.io docker-buildx-plugin \
     docker-compose-plugin
\end{verbatim}

\item{Verify that the installation is successful by running the hello-world image:}
\begin{verbatim}
sudo docker run hello-world
\end{verbatim}

\item{Add yourself or another user to the docker group}
\begin{verbatim}
# Add yourself to the docker group
# https://docs.docker.com/engine/install/linux-postinstall/

# Adding to docker group
sudo groupadd docker
sudo usermod -aG docker $USER
newgrp docker
docker run hello-world
\end{verbatim}
\end{enumerate}

\section{Developer Container}
Finally we are at the point where we can create a developer container for our development environment.
Having a dev container will insure uniformity and repeatability, and generally make life easier.
We show how to create a dev container for GoLang, however you could select/create dev containers for other languages too.
Although we won't get into customization, dev containers can be customized by modifying the devcontainer.json file.

\subsection{Create a new dev container}
Let's create a basic dev container for our GoLang development.

\begin{enumerate}
  \item In VSCode Ctrl+Shift+P
  \item Search for and select: ``Dev Containers: New Dev Container...'' from the Select Dev Container Configuration prompt.
  \item Search for an appropriate GoLang container, for example ``Go devcontainers''.
    \begin{enumerate}
      \item This item has with official checkmark certificate icon.
      \item Description starts with ``Develop Go based applications...''
      \item You could select other developer containers depending on the languages you want to develop.
    \end{enumerate}
  \item Select ``Create Dev Container go''.  You could also select ``Additional Options...'', and pick your desired options.
  \item Wait a few mins while container image downloads and installs 
  \item A new instance of VSCode wil launch.
  \item while you are waiting, you can click the box that says "Connecting to dev container (show log)" and watch the progress.
\end{enumerate}

\subsection{Container shell prompt}
Did it work?
It would be good to test out our dev container before writing any code.

\begin{enumerate}
  \item Once the container has downloaded and installed, open a new terminal in VSCode by clicking View $>$ Terminal.
  \item You should see the prompt change to ``vscode -> /workspaces/go \$'' to indicate you are now running inside the developer container.
  \item At the prompt, enter ``go version''  
\end{enumerate}

You should see something that looks like this:
\begin{verbatim}
vscode -> /workspaces/go $ go version
go version go1.23.8 linux/amd64
\end{verbatim}


\subsection{Devcontainer file}
When the dev container was created, a file was also created in a hidden directory.
Search for and open the file at .devcontainer/devcontainer.json

The contents of the will resemble the JSON object below.


\begin{verbatim}
{
  "name": "Go",
  // Or use a Dockerfile or Docker Compose file.
  // More info: https://containers.dev/guide/dockerfile
  "image": "mcr.microsoft.com/devcontainers/go:1-1.23-bookworm"

  // Features to add to the dev container.
  // More info: https://containers.dev/features.
  // "features": {},

  // Use 'forwardPorts' to make a list of ports inside the
  // container available locally.
  // "forwardPorts": [],

  // Use 'postCreateCommand' to run commands after the container is created.
  // "postCreateCommand": "go version",

  // Configure tool-specific properties.
  // "customizations": {},
  
  // Uncomment to connect as root instead.
  // More info: https://aka.ms/dev-containers-non-root.
  // "remoteUser": "root"
}
\end{verbatim}

The important part of the JSON is the value for the ``image'' field.
We see the value is ``mcr.microsoft.com/devcontainers/go:1-1.23-bookworm''.
This tells us that you will find the developer container in the Microsoft Artifact Registry located at ``https://mcr.microsoft.com''.
You can browse Microsoft's repository for other developer containers.
You can also browse other repositories for containers.


\subsection{Development and Customization}
As you add files to your project, they should be available in your dev container and appear in VSCode.
Developing in a dev container is a broad subject, beyond the scope of this article.
You should know that you can customize and fine-tune your dev container by modifying devcontainer.json file.

You are encouraged to read more here: https://code.visualstudio.com/docs/devcontainers/containers

\subsection{Ready to Develop!}
Pin your VSCode session on the Windows toolbar so you can quickly launch it next time.

\begin{enumerate}
  \item Right-click the VSCode icon on the Windows toolbar at the bottom of the screen.
  \item Under ``Recent Folders'' hover over ``go [Dev Container]''
  \item Click the pin icon.
  \item Next time you want to launch your GoLang developer container, you will find it in the Pinned.
  \item Just Right-click the VSCode icon.
\end{enumerate}

At this point your development environment is ready to use.
You are open for business!

\section{Other Considerations}
As you go, you'll no doubt want to add things to your environment to make life easier.
Read the following subsections regarding helpful additions and fine-tuning.

\subsection{VSCode Extensions}
VSCode has a nice system for adding extensions.
Earlier we used it to add the Remote Development extension pack so that we connect VSCode in to our remote alpha VM.

There are thousands of VSCode extensions.
Some useful, others not so.

Through experience and habit I have found the following extensions to help me in my work.

\begin{itemize}
  \item Code Spell Checker: by Street Side Software
  \item vscode-pdf: by tomoki1207
  \item Markdown PDF: by yzane
  \item Docker: by Microsoft
  \item Go: by Go Team at Google
\end{itemize}

To install them and other extensions goto Ctrl+Shift+X in VSCode.

\subsection{Docker repositories}
The main public repositories are not the only container repositories that exist.
Other semi-public and government sponsored repositories also exist.
Private corporate repositories may also exist, for example hosted with Artifactory.

\begin{itemize}
  \item DockerHub (https://hub.docker.com/)
  \item RedHat Quay (https://access.redhat.com/products/red-hat-quay)
  \item Microsoft Artifact Registry (https://mcr.microsoft.com/)
  \item IronBank (https://p1.dso.mil/ironbank)
\end{itemize}

You will want to setup Docker or Podman to recognize all the container repositories you plan to use for your work.
You may also want to setup Docker or Podman to automatically login to those repositories when you begin a development session.

\section{Conclusion}
This article described the setup of a basic remote containerized Golang development environment.
While you may not be writing Golang code, you can follow the general steps to setup an environment for your project.

Every project is different.
Your software development project will be different than mine.
I guarantee it!

\begin{itemize}
  \item Instead of writing Golang code, you may be writing Java code.
  \item Instead of hosting a remote VM in VirtualBox, you remote VM maybe running in an AWS EC2 instance.
  \item Instead of running Ubuntu in your VM, you may be running RHEL.
  \item Instead of containers in Docker you may have containers in Podman.
  \item Instead of VSCode on a Windows Laptop, you may have Emacs on a Linux desktop box.
\end{itemize}

Yes, each of these scenarios are different.
However, despite the differences from project to project, the general approach to setting up a development environment is similar.
You can take the steps that are laid out in this article and substitute whichever technologies you are using for your project.

What is important is the process.
Recognize the patterns!

Best wishes on your software development project.

\section{Acronymns, Abbreviations and Terminology}

\begin{itemize}
\item 1Password : A password manager (https://1password.com/)
\item APT : Advanced Package Tool (https://en.wikipedia.org/wiki/APT_(software))
\item AWS : Amazon Web Service (https://aws.amazon.com/)
\item Azure : Microsoft cloud platform (https://azure.microsoft.com/)
\item BASH : Bourne Again Shell (https://en.wikipedia.org/wiki/Bash_(Unix_shell))
\item Bitwarden : A password manager (https://bitwarden.com/)
\item BSD : Berkeley Software Distribution (https://en.wikipedia.org/wiki/Berkeley_Software_Distribution)
\item Buildah : Container image tool (https://buildah.io/)
\item CD : Compact Disk
\item CMD : Windows command shell
\item Debian : Linux Distribution (https://www.debian.org/)
\item DevOps : Developer Operations
\item Distro : Distribution
\item Docker : Container Tool (https://www.docker.com/)
\item DockerHub : Container library (https://hub.docker.com/)
\item DVD : Digital Video Disc (https://en.wikipedia.org/wiki/DVD)
\item EC2 : Amazon Elastic Compute Cloud (https://en.wikipedia.org/wiki/Amazon_Elastic_Compute_Cloud)
\item Emacs : Family of text editors (https://en.wikipedia.org/wiki/Emacs)
\item ESXi : Hypervisor by VMWare (https://en.wikipedia.org/wiki/VMware_ESXi)
\item Fedora : A Linux Distribution (https://fedoraproject.org/)
\item Git : Source version control system (https://git-scm.com/)
\item GitHub :
\item GitLab :
\item Golang : Go Programming Lanuage (by Google)
\item HTTPS :
\item HyperV :
\item IDE : Integrated Development Environment
\item IP :
\item IPv4 :
\item ISO :
\item IT : Information Technology
\item Java :
\item JSON :
\item KeePassXC :
\item Linux :
\item NAT :
\item onprem :
\item OpenSSH :
\item PAT :
\item PC : Personal Computer
\item Podman :
\item Podman :  
\item ProxMox :
\item Repo :  Repository
\item RHEL :
\item SSH :  Secure SHell protocol
\item Ubuntu :
\item URL :
\item VirualBox :
\item VM : Virtual Machine (Virtual Personal Computer)
\item VMWare :
\item VSCode : Visual Studio Code
\item vSphere :
\item WSL : Windows Subsystem for Linux
\item YUM :
\end{itemize}

\end{document}
