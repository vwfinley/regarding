\section{Introduction}
Your software development environment can either speed or hinder progress.

Recent years have seen an explosion in software developer tooling.
More power is within the reach of more developers, at the cost of more choices and increased complexity.

Aggregating a collection of tooling into an integrated whole is the key to attaining a successful developer environment.
Seldom is found comprehensive documentation describing how independent tools can be assembled into a working development environment. 

This article shows how to integrate several tools into a working development environment.

Prior to 20 years ago, software development centered around the desktop.
\begin{itemize}
  \item Software was written, compiled, tested and run on a desktop PC.
  \item The end product targeted users who installed the released software on their own desktop PCs.
  \item This system had the advantage of simplicity.
  \item Setting up the development environment was relatively straightforward.
\end{itemize}

Typically an IDE, runtime engine and supporting libraries were installed on the developer's networked desktop PC.
\begin{itemize}
  \item Everything needed to develop code was contained locally on the desktop PC.
  \item The environment was truly an integrated whole.
  \item Only an external code repository was needed to share code amongst the development team.
\end{itemize}

Software development became more complex with the rise of Web applications.
No longer was the software product intended to be run on the desktop.
The user interface now ran in a browser, and the low-level application logic ran on a server somewhere on a network.

Around 10+ years ago, when containerized microservice applications appeared on scene, development became more complex.
Developers found themselves writing code remotely.
Their working code and compiler resided on a remote machine, whereas their physical laptop PC acted as just a terminal.     
