\section{Other Considerations}
As you go, you'll no doubt want to add things to your environment to make life easier.
Read the following subsections regarding helpful additions and fine-tuning.

\subsection{VSCode Extensions}
VSCode has a nice system for adding extensions.
Earlier we used it to add the Remote Development extension pack so that we connect VSCode in to our remote alpha VM.

There are thousands of VSCode extensions.
Some useful, others not so.

Through experience and habit I have found the following extensions to help me in my work.

\begin{itemize}
  \item Code Spell Checker: by Street Side Software
  \item vscode-pdf: by tomoki1207
  \item Markdown PDF: by yzane
  \item Docker: by Microsoft
  \item Go: by Go Team at Google
\end{itemize}

To install them and other extensions goto Ctrl+Shift+X in VSCode.

\subsection{Docker repositories}
The main public repositories are not the only container repositories that exist.
Other semi-public and government sponsored repositories also exist.
Private corporate repositories may also exist, for example hosted with Artifactory.

\begin{itemize}
  \item DockerHub (https://hub.docker.com/)
  \item RedHat Quay (https://access.redhat.com/products/red-hat-quay)
  \item Microsoft Artifact Registry (https://mcr.microsoft.com/)
  \item IronBank (https://p1.dso.mil/ironbank)
\end{itemize}

You will want to setup Docker or Podman to recognize all the container repositories you plan to use for your work.
You may also want to setup Docker or Podman to automatically login to those repositories when you begin a development session.
