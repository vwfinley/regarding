\section{Docker}
We eventually want to do are development in a developer container.
Before we can think about containers we need to install a container runtime on our alpha VM.

The easiest way to install a container runtime is to install either Docker or Podman.
Since much tooling was developed with Docker in-mind, Docker is the simplest to work with.
Podman is more secure, however mapping directories from the remote host to a rootless container can tricky.

When your remote host is Fedora/RHEL, you will probably use Podman.
Otherwise you will probably use Docker.

Since our alpha VM is running Ubuntu, we will install Docker.
To install Docker, at the VSCode BASH terminal connected to the alpha VM, do the following:

\begin{verbatim}
# Docker install/setup
# https://docs.docker.com/engine/install/ubuntu/

# Add yourself to the docker group
# https://docs.docker.com/engine/install/linux-postinstall/

# Adding to docker group
$ sudo groupadd docker
$ sudo usermod -aG docker $USER
$ newgrp docker
$ docker run hello-world
\end{verbatim}
