\section{Network Setup}

The Ubuntu desktop distribution has been installed into a VM that runs in VirtualBox on the laptop.
As far as the laptop is concerned, the VM is a wholly independent machine, even though the VM lives on the laptop.

We need a way to talk to the VM from the laptop.
To achieve this we will be making a change to the network, and then gathering some information.

Initally, VirtualBox created the VM with a single default network adapter that supports Network Address Translations (NAT).
This default adapter is fine for connecting to the outside world: browsing to an internet website, or pinging google.com.
In other words, outbound traffic is supported.

However, inbound traffic is unsupported.
If we were to ping the ``alpha'' VM from the commandline on the laptop, we would get no reply.

\subsection{Add Bridge Adapter}
To support inbound traffic, we need to add another network adapter.

\begin{enumerate}
  \item In VirtualBox: Select machine alpha (powered off).
  \item Goto Settings $>$ Network $>$ Adapter 2 tab
  \item Check Enable Network Adapter
  \item Change Attached to: = Bridged Adapter
  \item Click OK
\end{enumerate}



\subsection{Gather IP Addresses}
Let's verify the network is configured correctly, and can send/receive traffic.
While we are at it, let's find the alpha VM IP addresses.

\begin{enumerate}
  \item You should still be logged into the alpha VM from the previous section.
  \item Click Activities $>$ Settings $>$ Network
  \item For each of the two Ethernet connections:
    \begin{itemize}
      \item Click on the Gear button
      \item Click on the Details tab
      \item Write down the IPv4 addresses for later.
      \item For example: 10.0.2.15 and 10.0.0.246
    \end{itemize}
  \item Close the Settings application. 
\end{enumerate}


\subsection{Outbound connection}
We have two network adapters each with a IPv4 address associated to it.
It is unclear which address is the mapped to the NAT adapter and which is mapped to the Bridged adapter.
In other words we don't know which is being used for outbound and inbound traffic.
However we can test to verify the outbound connection is working properly.

\begin{enumerate}
  \item On the Ubuntu destop, click Activities.
  \item Search for and select ``terminal''.
  \item Once the terminal application launches, at the commandline type: \begin{verbatim}ping google.com\end{verbatim}
  \item Hit Enter
  \item Verify you get a response back.
  \item Hit Ctrl+C to stop pinging.
\end{enumerate}

\subsection{Inbound connection}
Okay, let's verify that the inbound (Bridged) connection is working.
It is also an opportunity to determine which address is the inbound address, and which is the outbound address.

\begin{enumerate}
  \item On the laptop, open a Windows CMD terminal.
  \item In at the commandline in the terminal you opened above, run ping with one of the addresses you wrote down above.
  \item For example: \begin{verbatim}ping 10.0.2.15\end{verbatim}
  \item If you get a response, it is the inbound (Bridged) traffic address, otherwise it is the outbound (NAT) address.
  \item Write either ``inbound'' or ``outbound'' next to this address.
  \item At the commandline try pinging the other address.
  \item For example: \begin{verbatim}ping 10.0.0.246\end{verbatim}
  \item If you get a response, it is the inbound (Bridged) traffic address, otherwise it is the outbound (NAT) address.
  \item Write either ``inbound'' or ``outbound'' next to this address.
\end{enumerate}





%%\begin{verbatim}
%%https://www.bing.com/videos/riverview/relatedvideo?q=virtualbox+ubuntu+22.04+cannot+ping+from+host&mid=5B1AA719E4ADB9624DB45B1AA719E4ADB9624DB4&mcid=FEEEA5610D984BF1B9DDC47C30A02B7B&FORM=VIRE
%%\end{verbatim}








