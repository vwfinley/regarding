\section{Network Setup}

The Ubuntu desktop distribution has been installed into a VM that runs in VirtualBox on the laptop.
As far as the laptop is concerned, the VM is a wholly independent machine, even though the VM lives on the laptop.

We need a way to talk to the VM from the laptop.
To achieve this we will be making a change to the network, and then gathering some information.

Initally, VirtualBox created the VM with a single default network adapter that supports Network Address Translations (NAT).
This default adapter is fine for connecting to the outside world: browsing to an internet website, or pinging google.com.
In other words, outbound traffic is supported.

However, inbound traffic is unsupported.
If we were to ping the "alpha" VM from the commandline on the laptop, we would get no reply.

\subsection{Add Bridge Adapter}
To support inbound traffic, we need to add another network adapter.

\begin{enumerate}
  \item In VirtualBox: Select machine alpha (powered off).
  \item Goto Settings $>$ Network $>$ Adapter 2 tab
  \item Check Enable Network Adapter
  \item Change Attached to: = Bridged Adapter
  \item Click OK
\end{enumerate}



\subsection{Gather IP Addresses}
Let's verify the network is configured correctly, and can send/receive traffic.
While we are at it, let's remember the alpha VM IP addresses.

\begin{enumerate}
  \item In VirtualBox right-click the VM named alpha.
  \item Select Start $>$ Normal Start
  \item When the Ubuntu login window appears, login.
  \item When the desktop appears, click Activities $>$ Settings $>$ Network
  \item For each of the two Ethernet connections: 
    Click on the Gear button
    Click on the Details tab
    Write down the IP4 address
    
    10.0.2.15
  \item Click in the boot window, and 
\end{enumerate}


VirtualBox > Settings > Network > Bridged adapter.



\begin{verbatim}
https://www.bing.com/videos/riverview/relatedvideo?q=virtualbox+ubuntu+22.04+cannot+ping+from+host&mid=5B1AA719E4ADB9624DB45B1AA719E4ADB9624DB4&mcid=FEEEA5610D984BF1B9DDC47C30A02B7B&FORM=VIRE
\end{verbatim}

\begin{itemize}
  \item Memory 2048
  \item Processors 2
  \item Video mem 16MB?
\end{itemize}

For Ubuntu Desktop Distro
\begin{enumerate}
  \item In Ubuntu > Settings > Network > Wired Connected Gear button > Details tab > IP4 address 10.0.2.15
  \item In Ubuntu > terminal > ping google.com
  \item In VirtualBox > Select machine (powered off) > Settings > Network > Adapter 1 tab
    \begin{itemize}
      \item (default checked) Enable Network Adapter
      \item (default) Attached to: NAT
      \item (default) Adapter type: Intel PRO/1000 MT Desktop (82540EM)
      \item MAC Address: 080027742766
      \item (default checked) Cable Connected
    \end{itemize}
  \item Adapter 2 Tab
    \begin{itemize}
      \item Check: Enable Network Adapter
      \item Select: Attached to:  Host-only Adapter
      \item Name, Adapter type, Promiscuous Mode, MAC Address, Cable   connected: Will auto populate
      \item Click OK
    \end{itemize}
  \item VirtualBox > File > Tools > Network Manager > VHost-only Networks
    \begin{itemize}
      \item VirtualBox Host-Only Ethernet Adapter should appear in list.
      \item IPV4 Prefix 192.168.56.1/24
      \item DHCP Server = Enabled
     \end{itemize}
  \item Start the Ubuntu VM 
  \item Settings > Network > Choose second adapter > Gear icon
    \begin{itemize}
      \item IPv4 Address 192.168.56.104
      \item From host CMD shell: Ping 192.168.56.104
      \item See video for Ubuntu Server distro (netplan setup?)
      \item sudo apt-get update
    \end{itemize}
  \item Edit hosts file
    \begin{itemize}
      \item On Windows11 system
      \item Run Notepad as admin
      \item Open 
    \end{itemize}
\end{enumerate}

\begin{verbatim}
C:\Windows\System32\drivers\etc\hosts
- Add entry: 192.168.56.104    alpha # Ubuntu VM
- Save and close the hosts file
- In CMD shell: ping alpha
\end{verbatim}


Screen settings:
\begin{itemize}
\item In Ubuntu Display 1680x1050 landscape
\item Privacy > Screen
  \begin{itemize} 
    \item Blank screen: Never
    \item Automatic Screen Lock: Disabled
    \item Lock Screen on Suspend: Disabled
  \end{itemize}
\end{itemize}

Settings > Power > Automatic Suspend = off


\begin{verbatim}
sudo apt-get install openssh-server
sudo systemctl status sshd
sudo systemctl restart sshd 
Edit local host file with VM name and IP address.
- C:\Windows\System32\drivers\etc\hosts
- Open notepad as admin
- Or ask admin to put in DNS
\end{verbatim}



\subsection{Inital Ubuntu Setup}


\noindent
Create VM
\begin{itemize}
  \item Initial resources (cpu, memory, storage)
  \item Install OS
  \item Minimum install
  \item Give VM a name
  \item Setup user account
  \item Ping google test
  \item Get ip address, write it down
  \item Run software update (yum, apt)
\end{itemize}

