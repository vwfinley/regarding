\section{Developer Container}
Finally we are at the point where we can create a developer container for our development environment.
Having a dev container will insure uniformity and repeatability, and generally make life easier.
We show how to create a dev container for GoLang, however you could select/create dev containers for other languages too.
Although we won't get into customization, dev containers can be customized by modifying the devcontainer.json file.

\subsection{Create a new dev container}
Let's create a basic dev container for our GoLang development.

\begin{enumerate}
  \item In VSCode Ctrl+Shift+P
  \item Search for and select: ``Dev Containers: New Dev Container...'' from the Select Dev Container Configuration prompt.
  \item Search for an appropriate GoLang container, for example ``Go devcontainers''.
    \begin{enumerate}
      \item This item has with official checkmark certificate icon.
      \item Description starts with ``Develop Go based applications...''
      \item You could select other developer containers depending on the languages you want to develop.
    \end{enumerate}
  \item Select ``Create Dev Container go''.  You could also select ``Additional Options...'', and pick your desired options.
  \item Wait a few mins while container image downloads and installs 
  \item A new instance of VSCode wil launch.
  \item while you are waiting, you can click the box that says "Connecting to dev container (show log)" and watch the progress.
\end{enumerate}

\subsection{Container shell prompt}
Did it work?
It would be good to test out our dev container before writing any code.

\begin{enumerate}
  \item Once the container has downloaded and installed, open a new terminal in VSCode by clicking View $>$ Terminal.
  \item You should see the prompt change to ``vscode -> /workspaces/go \$'' to indicate you are now running inside the developer container.
  \item At the prompt, enter ``go version''  
\end{enumerate}

You should see something that looks like this:
\begin{verbatim}
vscode -> /workspaces/go $ go version
go version go1.23.8 linux/amd64
\end{verbatim}


\subsection{Devcontainer file}
When the dev container was created, a file was also created in a hidden directory.
Search for and open the file at .devcontainer/devcontainer.json

The contents of the will resemble the JSON object below.


\begin{verbatim}
{
  "name": "Go",
  // Or use a Dockerfile or Docker Compose file.
  // More info: https://containers.dev/guide/dockerfile
  "image": "mcr.microsoft.com/devcontainers/go:1-1.23-bookworm"

  // Features to add to the dev container.
  // More info: https://containers.dev/features.
  // "features": {},

  // Use 'forwardPorts' to make a list of ports inside the
  // container available locally.
  // "forwardPorts": [],

  // Use 'postCreateCommand' to run commands after the container is created.
  // "postCreateCommand": "go version",

  // Configure tool-specific properties.
  // "customizations": {},
  
  // Uncomment to connect as root instead.
  // More info: https://aka.ms/dev-containers-non-root.
  // "remoteUser": "root"
}
\end{verbatim}

The important part of the JSON is the value for the ``image'' field.
We see the value is ``mcr.microsoft.com/devcontainers/go:1-1.23-bookworm''.
This tells us that you will find the developer container in the Microsoft Artifact Registry located at ``https://mcr.microsoft.com''.
You can browse Microsoft's repository for other developer containers.
You can also browse other repositories for containers.


\subsection{Development and Customization}
As you add files to your project, they should be available in your dev container and appear in VSCode.
Developing in a dev container is a broad subject, beyond the scope of this article.
You should know that you can customize and fine-tune your dev container by modifying devcontainer.json file.

You are encouraged to read more here: https://code.visualstudio.com/docs/devcontainers/containers

\subsection{Ready to Develop!}
Pin your VSCode session on the Windows toolbar so you can quickly launch it next time.

\begin{enumerate}
  \item Right-click the VSCode icon on the Windows toolbar at the bottom of the screen.
  \item Under ``Recent Folders'' hover over ``go [Dev Container]''
  \item Click the pin icon.
  \item Next time you want to launch your GoLang developer container, you will find it in the Pinned.
  \item Just Right-click the VSCode icon.
\end{enumerate}

At this point your development environment is ready to use.
You are open for business!
