\section{Solution C: Scratchbuilding from a Base Image}

Rather than depending upon container images that have been populated by others, a better solution is to populate your own image from a base image.

The advantage to this solution is you have total control over the contents and configuration of the image.
You don't have to troubleshoot configurations some previous author may have forgotten to document.

Also, once you know how to scratchbuild from base images, you can deploy almost any application to a container, including your own code projects.

The workflow for building images from a base image is shown in Figure \ref{fig:fig2}.
You will notice it is similar but not the same as the Solution B workflow from Figure \ref{fig:fig1}.
Why is this?
Well, since we are starting with a empty base container we will first need to populate it before testing by generating a PDF document.
Usually, several attempts will be required to get the image to build correctly.
Inevitably packages or configurations will be missing and you'll need to iterate through the cycle in \ref{fig:fig2}.

\begin{figure}[!h]
\centering
\begin{tikzpicture}
  \node [minimum width=1.25cm,draw,circle] (a) at (0,0) {\scriptsize Pull};
  \node [minimum width=1.25cm,draw,circle] (b) at ($(a)+(1.75,0)$) {\scriptsize Extend};
  \node [minimum width=1.25cm,draw,circle] (c) at ($(b)+(1.75,0)$) {\scriptsize Build};
  \node [minimum width=1.25cm,draw,circle] (d) at ($(c)+(1.75,0)$) {\scriptsize Test};
  \node [minimum width=1.25cm,draw,circle] (e) at ($(d)+(1.75,0)$) {\scriptsize Push};

  \draw[-{Latex[scale=1.0]}] (a) -- (b);
  \draw[-{Latex[scale=1.0]}] (b) -- (c);
  \draw[-{Latex[scale=1.0]}] (c) -- (d);
  \draw[-{Latex[scale=1.0]}] (d) -- (e);
  \draw[-{Latex[scale=1.0]}] (e) to[out=135,in=45] (a);
  \draw[-{Latex[scale=1.0]}] (d) to[out=135,in=45] (b);
\end{tikzpicture}
\caption{\\Solution C Customization Workflow}
\label{fig:fig2}
\end{figure}

\subsection{Strategy}
The strategy for scratchbuilding our own LaTeX container image is:
\begin{itemize}
\item Pull a suitable base devcontainer image
\item Install the TexLive LaTeX distribution into the image. 
\item Add any missing needed packages
\item Fix any warnings/errors (mktexpk for generating missing fonts)
\item Use container for a project
\end{itemize}

Furthermore, we will be rebuilding the container image frequently during development and maintenance.
It would be nice to automate the build process.
Therefore, this is a great opportunity to learn about GitHub pipelines, actions and runners.

We would like to break the problem up into two parts.
The first part is to create an image only with a basic TexLive distribution.
This second part is to add any extra packages to the first image and create another image.

The idea here is that the basic TexLive image can be reused to create other images that are customized for specific document projects.
If we pollute the basic TexLive image with unnecessary packages it won't suit most document projects anyway.
The image would need to be customized to accommodate most document projects, so why not just start with a simple image and a clear way to customize it as needed.

The code that follows is checked in at \url{https://github.com/vwfinley/texlive} and \url{https://github.com/vwfinley/texlive_dev}.

\subsection{Select Base Image}

First let's choose a suitable base image.
The mcr.microsoft.com/devcontainers/base:dev-debian-12 image is a friendly plain vanilla developer container image.
You will find it listed here: \url{The base image https://mcr.microsoft.com/en-us/artifact/mar/devcontainers/base/tag/dev-debian-12}
Sure we could find some non-developer image, but the developer image will play nice with VSCode.
If you are setting up a production environment, where hundreds of documents will be built and published, you might consider a slim bare-bones non-developer image.

\subsecion{Install TexLive}

\begin{mdframed}[backgroundcolor=black!20,leftmargin=0.0cm,skipabove=0.4cm,hidealllines=true,
  innerleftmargin=0.1cm,innerrightmargin=0.1cm,innertopmargin=0.15cm,innerbottommargin=-0.9cm]
\begin{code}[caption={Scratchbuilding the TexLive Image}, label=lst:scratchbuildTexlive]
ARG SRC_IMG=mcr.microsoft.com/devcontainers/base
ARG TAG=dev-debian-12

FROM ${SRC_IMG}:${TAG}

ARG DEST=https://github.com/vwfinley/texlive
ARG MIRROR=https://mirror.ctan.org
ARG FILE=systems/texlive/tlnet/install-tl-unx.tar.gz
ARG VER=2025
ARG SCHEME=small
ARG DIR=/usr/local/texlive/${VER}/bin/x86_64-linux

ENV PATH="$PATH:${DIR}"

LABEL org.opencontainers.image.source ${DEST}

RUN apt-get update

WORKDIR /tmp 

RUN <<EOF
  wget ${MIRROR}/${FILE}
  zcat < install-tl-unx.tar.gz | tar xf -
  cd install-tl-*
  perl ./install-tl --scheme=${SCHEME} \
    --no-interaction --no-doc-install \
    --no-src-install
  tlmgr update --all
  cd ..
  rm -r install*
EOF

WORKDIR /workspace 

USER vscode
\end{code}
\end{mdframed}


\subsection{Pull}
\subsection{Extend}
\subsection{Build}
\subsection{Test}
\subsection{Push}
