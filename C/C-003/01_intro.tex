\section{Introduction}
All articles in the Regarding Series\footnote{https://github.com/vwfinley/regarding}, including this one, have been typeset with LaTeX.

TeX is the typesetting system published by Donald Knuth\footnote{https://en.wikipedia.org/wiki/TeX} in 1978.
Whereas LaTeX is the macro library, released by Leslie Lamport\footnote{https://en.wikipedia.org/wiki/LaTeX} in 1984, that extends TeX.

For over 40 years LaTeX has been used extensively in academia and the publishing industry.
It has been used to typeset: articles, scientific journals, whitepapers, PhD theses, books, and many other printed works.

If you want clear typography for text and mathematical equations, then LaTeX is a good choice.

At university, you received: course syllabi, handout notes, homework assignments, lab instructions, and exam papers.
You'll remember these had a consistently academic look and feel.
Equations were neatly laid out on the page and easy to read.
Formatting was no frills, nothing fancy, just stark black and white.
The reason everything looked so consistently crisp and clean is because your professors probably used LaTeX, or a similar markup \footnote{https://en.wikipedia.org/wiki/List_of_document_markup_languages} system, to typeset their course materials.

Aside from clean consistent layout LaTeX offers other benefits to authors, such as:
\begin{itemize}
  \item Separation of content from presentation
  \item Plain text source input
  \item Partitioning of document structure
  \item Coexistence with platform resident tooling
  \item Cooperation with source control systems and build environments
  \item External data injection support
  \item Automation of document generation
  \item Conversion and publication of documents into myriad formats
\end{itemize}

While LaTeX can be extremely useful and powerful, it does arrive with some costs and burdens.
The primary burden is a steep learning curve for document authors and editors.
Fortunately there is no lack of online documentation and tutorials for the curious.

