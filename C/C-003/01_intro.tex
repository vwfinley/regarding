\section{Introduction}
All articles in the \footnote{Regarding Series repository, https://github.com/vwfinley/regarding}, including this one, have been typeset with LaTeX.

TeX is the typesetting system released by Donald Knuth in 1978.
LaTeX is the macro library, released by Leslie Lamport in 1984, that extends TeX.

For over 40 years LaTeX has been used extensively in the publishing industry and in academia.
It has been used to typeset books: articles, scientific journals, whitepapers, PhD theses, and many other printed works.

If you want clear typography and presentation of text and mathematical equations, then LaTeX is a good choice.

At university, you'll remember the: syllabus, handout notes, homework assignments, lab instructions, and exam papers; you received probably had a consistent academic look and feels.
Equations were neatly laid out on the page and easy to read.
The formatting was no frills, nothing fancy, just plain black and white.
The reason everything looked so consistently academic was because your professors may have used LaTeX to typeset their course materials.

Aside from clean consistent layout LaTeX offers other benefits, to document authors, such as:
\begin{itemize}
  \item Plain text source input
  \item Separation of content from presentation
  \item Partitioning of document structure
  \item Coexistence with platform resident tooling
  \item Cooperation with source control systems and build environments
  \item Injection of external data into of documents
  \item Automation of document generation process
  \item Conversion and publication of documents into myriad formats
\end{itemize}

While LaTeX is extremely useful and powerful, it arrives with some costs and burdens.
The primary burden is a steep learning curve for document authors and editors.
Fortunately there is no lack of online documentation and tutorial for the curious.

The other main burden is a cumbersome environment setup.
Yes, some easy setup options exist, but document authors will soon outgrow those environments.

This article will describe a progression of LaTeX environments from simple desktop setups to more advanced containerized installations.  





While LaTeX yields many type



LaTeX is the typesetting system develop



The previous article\footnote{C-001: Regarding Setup of a Development Environment, https://github.com/vwfinley/regarding/blob/main/C/C-001/C-001.pdf} described how to setup an Ubuntu desktop distribution as a VM (Virtual Machine) and access it from VSCode (Visual Studio Code).

While it may make sense to install a desktop distro locally on a laptop, it makes less sense to install it on a remote shared development environment.
Why?
Because desktop distributions provide a GUI (Graphical User Interface) that need either local console access or a remote desktop client.

Sure, installing a desktop distro works fine if you are using the desktop GUI on a laptop, and the laptop is being used as a terminal.
The single lucky user currently logged into the desktop GUI at the console will have benefit of the physical: keyboard, mouse and video display.
Other users, logged in via remote desktop connections, will experience a slow awkward desktop session.

So, it would make more sense for users to:
\begin{itemize}
  \item Install a desktop distro locally on their laptops.
  \item Use the local desktop GUI as only as a terminal.
  \item Connect remotely to a server running a server distro.
  \item Perform any work on the server.
\end{itemize}

Typically a server distro gets installed on fast hardware with vast resources.
When installed on powerful server hardware, users can experience fast compute and compile cycles.
The time to perform work on a powerful server can be significantly less than doing it locally on an underpowered laptop.

The other advantage of server distros is they tend to be smaller than desktop distros.
How can this be?
Server distros do not include all the graphical packages and libraries needed to display a desktop GUI.
Less packages equals smaller footprint and easier maintenance.
