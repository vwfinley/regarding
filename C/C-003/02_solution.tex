\section{Solution}
You are probably working on either a laptop or desktop PC running Microsoft Windows.
This was the same situation encountered in the previous article.

Microsoft Windows already has a desktop.
As in the previous article, we will use the Windows desktop the as a ``terminal'' that can connect to remote systems.

However this time instead of installing Ubuntu Desktop in a VM, we will install Ubuntu Server in the VM.
Ubuntu Server will act as our remote development system.

We can describe Ubuntu Server as ``headless'' because it does not include a GUI.
A headless server installation is more appropriate place for teams of developers to collaborate.

Okay, so what's our strategy to setup a remote development environment?
We will:
\begin{itemize}
  \item Follow many of the steps found in the previous article.
  \item Install an Ubuntu server distro rather than the Ubuntu Desktop distro.
  \item Make a few changes that are unique to the Server distro.
  \item Connect VSCode to the remote Ubuntu Server instance.
  \item Complete the environment setup.
\end{itemize}




