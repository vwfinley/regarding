\section{Solution}
Awkward environment setup can be another burden of using LaTeX.
At first an author using LaTeX for typesetting may want to the use a LaTeX WYSIWYG editor like TeXworks\footnote{https://tug.org/texworks}.
Installation for TeXworks is simple and easy, especially on a Microsoft Windows platform.
Choosing this option will start an author quickly down the path toward publication.
In fact, for the early articles in the Regarding series, TeXworks was selected for this very reason.

However as an author gains more experience with LaTeX typesetting, and documents grow in complexity, the author will outgrow WYSIWYG tool.
This is to be anticipated since people usually choose LaTeX vs. WYSIWYG wordprocessors as documents become complex.
The ability to automate document management and generation gives LaTeX a big advantage when publishing 500 page long technical books.

Because setting up LaTeX can be tricky, we would like to perform setup only when absolutely necessary.
We would also like to standardize the setup across hosting platforms.
The best way to achieve both aims is to containerize the LaTeX installation once and install the container image as needed.

A question then arises.  
How does one use LaTeX in a container?
There are many answers to this question, but they fall into two main categories.
The first category is to find a pre-built container image that offers a LaTeX installation.
The second category is to install a LaTeX distribution into a base container image.

The pros and cons of each category will be discussed in their respective sections to follow.

In all cases the first step to realizing a containerized LaTeX distribution is selecting the desired distribution.
 

So how best to realize the 

https://www.latex-project.org/


http://www.tikzedt.org/   
https://en.wikipedia.org/wiki/PGF/TikZ



- Old Windows environment
- New Linux environment
  - Refer to C-001 and C-002 Ubuntu
  - Remote VSCode extensions to connect to VM Bravo remotely.
- TexLive
- Need VS add on packages
- Gradual move toward automation.
- TexLive flavors, basic, etc.

- A word about Difference between packages, containers and virtual machines.
  - Packages
    - OS packages
    - Language packages (npm, pip, etc.)
  - containers
  - virtual machines


- Strategy existing container vs. from scratch
  - Pros/cons
  - Existing container is quicker to pull from Dockerhub and use
  - Existing container is in unknown state.
  - Possibly poorly documented.
  - Maybe harder to troubleshoot.
  - If existing container is missing stuff, they you'll need to figure out what's missing and why.
  - May need to try several existing containers before finding a working match.
  - Scratch container is known state, clean state, blank canvas.
  - Need now worry about some unknown publisher's mistakes.
  - Setup as you please, no need to workaround previous author's preferences.
- Pros/cons of from scratch approach containers:
Not an exhaustive list
Good, Cheap, Fast
  - Minimal (Alpine) Small footprint, consumes minimal resources during runtime
  - Robust (UBI RHEL) Well tested and supported, good for production environment.
  - Developer (Msft dev container) Excellent integration with VSCode.  
    - Facilitates "Open in developer container".
    - Run in developer container.
    - Good directory mapping between host and devcontainer.
    - Supports installation of VSCode extensions in devcontainer.
    - Container process can also be run from Host.
    - Good support for debuggers and debugging source code.
    - Can be setup in host .devcontainer/devcontainer.json or Dockerfile


