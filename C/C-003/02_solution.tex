\section{Solution}
TODO write ME

- Old Windows environment
- New Linux environment
  - Refer to C-001 and C-002 Ubuntu
  - Remote VSCode extensions to connect to VM Bravo remotely.
- TexLive
- Need VS add on packages
- Gradual move toward automation.
- TexLive flavors, basic, etc.

- A word about Difference between packages, containers and virtual machines.
  - Packages
    - OS packages
    - Language packages (npm, pip, etc.)
  - containers
  - virtual machines


- Strategy existing container vs. from scratch
  - Pros/cons
  - Existing container is quicker to pull from Dockerhub and use
  - Existing container is in unknown state.
  - Possibly poorly documented.
  - Maybe harder to troubleshoot.
  - If existing container is missing stuff, they you'll need to figure out what's missing and why.
  - May need to try several existing containers before finding a working match.
  - Scratch container is known state, clean state, blank canvas.
  - Need now worry about some unknown publisher's mistakes.
  - Setup as you please, no need to workaround previous author's preferences.
- Pros/cons of from scratch approach containers:
Not an exhaustive list
Good, Cheap, Fast
  - Minimal (Alpine) Small footprint, consumes minimal resources during runtime
  - Robust (UBI RHEL) Well tested and supported, good for production environment.
  - Developer (Msft dev container) Excellent integration with VSCode.  
    - Facilitates "Open in developer container".
    - Run in developer container.
    - Good directory mapping between host and devcontainer.
    - Supports installation of VSCode extensions in devcontainer.
    - Container process can also be run from Host.
    - Good support for debuggers and debugging source code.
    - Can be setup in host .devcontainer/devcontainer.json or Dockerfile


