\section{Solution B: Customize an Existing LaTeX Container Image}

For most documents and projects, finding an available LaTeX container image is a good strategy.

Maybe the image was written by a colleague, or maybe by some unknown external developer.
Perhaps the image has been published to: Docker Hub, GitHub registry, or a private corporate registry.

In any case you will want to find an image that provides substantial functionality.
The more completely an image compiles features from your source files into a finished document, the less work you'll need to do.

Inevitably your selected image will be incomplete.
Required packages will be missing.
So, you will want to customize the container image to add missing packages and features.

\subsection{Image Customization}
Customization of the container image will resemble the iterative process shown in Figure \ref{fig:fig1}.
As you write your document, you will use the container to generate an output file.
You will then extend the container as needed until your output file is fully typeset as desired.

\begin{figure}[!h]
\centering
\begin{tikzpicture}
  \node [minimum width=1.25cm,draw,circle] (a) at (0,0) {\scriptsize Pull};
  \node [minimum width=1.25cm,draw,circle] (b) at ($(a)+(1.75,0)$) {\scriptsize Use};
  \node [minimum width=1.25cm,draw,circle] (c) at ($(b)+(1.75,0)$) {\scriptsize Extend};
  \node [minimum width=1.25cm,draw,circle] (d) at ($(c)+(1.75,0)$) {\scriptsize Build};
  \node [minimum width=1.25cm,draw,circle] (e) at ($(d)+(1.75,0)$) {\scriptsize Push};

  \draw[-{Latex[scale=1.0]}] (a) -- (b);
  \draw[-{Latex[scale=1.0]}] (b) -- (c);
  \draw[-{Latex[scale=1.0]}] (c) -- (d);
  \draw[dashed, -{Latex[scale=1.0]}] (d) -- (e);
  \draw[dashed, -{Latex[scale=1.0]}] (e) to[out=135,in=45] (a);
  \draw[-{Latex[scale=1.0]}] (d) to[out=135,in=45] (b);
\end{tikzpicture}
\caption{Workflow for Customization}
\label{fig:fig1}
\end{figure}

Once you are satisfied with the resulting document file, you can optionally push the final container image to a registry.

\subsection{Customization Example}
As an example, let's take a prebuilt LaTeX container image and customize it until it is suitable for typesetting this article you are now reading.

Glancing at the code listing below you will see the LaTeX container image was written by user jmuchovej and is uploaded to the GHCR (GitHub container registry).  
The container image is found at \url{https://github.com/jmuchovej/devcontainers/pkgs/container/devcontainers%2Flatex}
Furthermore you can find the source code for the image at \url{https://github.com/jmuchovej/devcontainers/tree/main/images/src/latex}






\lstnewenvironment{code}[1][]
{
 \lstset{
 basicstyle=\ttfamily\scriptsize,
 columns=fullflexible,
 captionpos=b,
 #1            
 }}{}


\begin{mdframed}[backgroundcolor=black!20,leftmargin=0.0cm,skipabove=0.4cm,hidealllines=true,
  innerleftmargin=0.1cm,innerrightmargin=0.1cm,innertopmargin=0.15cm,innerbottommargin=-0.1cm]

\begin{code}[caption={Customization Example}]
#-----------
# Dockerfile
#-----------
FROM ghcr.io/jmuchovej \
  /devcontainers/latex:2025

USER root

RUN <<EOF
    tlmgr update --self

    tlmgr install \
        pgf booktabs breqn caption \
        ec fc gensymb lastpage \
        makecell mathtools multirow \
        paralist parskip pgf sectsty \
        subfig tex4ht tocbibind 
        
    mktexpk --mfmode / --bdpi 600 \
      --mag 1+0/600 --dpi 600 tcrm1095
    mktexpk --mfmode / --bdpi 600 \
      --mag 1+0/600 --dpi 600 tcrm0900
    mktexpk --mfmode / --bdpi 600 \
      --mag 1+0/600 --dpi 600 tcrm0600
    mktexpk --mfmode / --bdpi 600 \
      --mag 1+0/600 --dpi 600 tcrm0800
EOF

USER vscode
\end{code}
%\captionsetup{format=hang, justification= raggedright, font=small}
%\captionof{mdframed}{Customization Example}
\end{mdframed}



Using incomplete images situation 
However, there will be times when 

Despite your best efforts you will discover missing 


Find existing LaTeX container image
Add missing needed packages
Fix warnings/errors (mktexpk for generating missing fonts)
Integrate/Use container in project.


