\section{TeXLive}
- TexLive is a distro of LaTeX https://tug.org/texlive/
- LaTeX is language much like golang, c++, java, python, etc.
  - Has compiler
  - Add library packages
- Difference is that instead of compiling code into a executable binary file, LaTeX compiles into a finished document.

- Package managers
  - OS Package managers, Fedora/RHEL = yum/dnf, Debian/Ubuntu = apt, Alpine = apk
  - OS feature level Packages for distro, included applications and software, tooling, low level drivers, etc.  
  - Software developer kits for a Language usuallly include compiler, linker, libraries, tools, package manager.
    - Most OS kernels are written in c/c++, 
      - Since c/c++ are integral to the core of most OS kernels
      - c/c++ usually installs with an OS distro. package manager
      - GNU gcc/g++ developer tools is a good example.
    - Other languages usually have their own package managers
      - Other lanauages have their own release cycles
      - Installed version of lanauage and libraries are independend of OS.
      - examples: npm = nodejs javascript distro, pip = python, .net = Nuget
      - Languages have their own package repos: python = pypi, .net nuget, https://www.npmjs.com/
- TexLive LaTeX has it's own package manager, tlmgr and it's own repository.
- We will want to identify any missing TexLive packages and use tlmgr to install them into container.
- What kind of packages?  Tikz, etc.