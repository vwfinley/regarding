\section{TeXLive}

In all cases the first step to realizing a containerized LaTeX distribution is selecting the desired distribution.
The three main LaTeX distributions are: MacTeX, MiKTeX and TeXLive.
Of these three, the TeXLive\footnote{https://tug.org/texlive/} distribution is most appropriate for the Linux containers we will be targeting. 

When working with the TeXLive distribution there are a few important maxims to keep in mind.
\begin{enumerate} 
\item LaTeX is language much like Golang, C++, Java, Python, etc.
\item The goal of LaTeX is to compile source code into a finished document, rather than an executable binary file as with programming languages.
\item LaTeX and the TeXLive distribution includes a compiler and linker, just like the general purpose computing languages do.
\item The TexLive distribution makes optional extension libraries available, much like the way distributions for general computing languages do.
\item The TeXLive distribution includes its own package manager named "tlmgr", similar to how the pip, npm and maven package mangers are included in distributions for their respective languages. 
\item TexLive library packages are kept in a mirrored\footnote{http://mirror.ctan.org/systems/texlive/tlnet} public registry, similar to how: GO Packages Repository\footnote{https://pkg.go.dev/}, PyPI\footnote{https://pypi.org/} and Maven Repository\footnote{https://mvnrepository.com/}; keep packages for their respective languages.  
\end{enumerate}


- LaTeX is language much like golang, c++, java, python, etc.
  - Has compiler
  - Add library packages
- Difference is that instead of compiling code into a executable binary file, LaTeX compiles into a finished document.

- Package managers
  - OS Package managers, Fedora/RHEL = yum/dnf, Debian/Ubuntu = apt, Alpine = apk
  - OS feature level Packages for distro, included applications and software, tooling, low level drivers, etc.  
  - Software developer kits for a Language usuallly include compiler, linker, libraries, tools, package manager.
    - Most OS kernels are written in c/c++, 
      - Since c/c++ are integral to the core of most OS kernels
      - c/c++ usually installs with an OS distro. package manager
      - GNU gcc/g++ developer tools is a good example.
    - Other languages usually have their own package managers
      - Other lanauages have their own release cycles
      - Installed version of lanauage and libraries are independend of OS.
      - examples: npm = nodejs javascript distro, pip = python, .net = Nuget
      - Languages have their own package repos: python = pypi, .net nuget, https://www.npmjs.com/
- TexLive LaTeX has it's own package manager, tlmgr and it's own repository.
- We will want to identify any missing TexLive packages and use tlmgr to install them into container.
- What kind of packages?  Tikz, etc.